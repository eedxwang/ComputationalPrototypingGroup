This manual describes FastHenry, a three-dimensional inductance
extraction program.  FastHenry computes the frequency dependent self
and mutual inductances and resistances between conductors of complex
shape.  The algorithm used in FastHenry is an acceleration of the mesh
formulation approach.  The linear system resulting from the mesh
formulation is solved using a generalized minimal residual algorithm
with a fast multipole algorithm to efficiently compute the
iterates. See Appendix~\ref{comfas} for references.

This manual is divided into four sections.  The first section explains
the syntax for preparing input files for FastHenry.  The input files contain the
description of the conductor geometries.
The second section describes how to run the program and process the output.
The third section describes the generation of postscript images to
visualize the three dimensional geometries defined in the input file.
The fourth section shows how to use Matlab to observe current
distribution in reference plances.

The files ``nonuniform\_manual\_*.ps'' constitute  an important
supplement to this 
manual and describe the new routines for specifying a nonuniformly
discretized reference plane.

Information on compiling FastHenry, obtaining the FastHenry source
code, producing this document, and 
corresponding about FastHenry is given in
Appendix~\ref{comfas}. 

Appendix~\ref{changes} summarizes most of the  changes in this
version, 3.0, over the previous version, 2.5.


