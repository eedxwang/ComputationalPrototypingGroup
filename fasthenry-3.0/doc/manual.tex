\documentstyle[12pt,twoside,psfig]{article}

\pagestyle{headings}
\setlength{\oddsidemargin}{0.25in}      % 1.25in left margin 
\setlength{\evensidemargin}{0.00in}     % 1.00in left margin (even pages)
\setlength{\topmargin}{0.0in}           % 1in top margin
\setlength{\textwidth}{6.25in}          % 6.25in text
\setlength{\textheight}{9in}            % Body ht for 1in margins
\addtolength{\topmargin}{-\headheight}  % No header, so compensate
\addtolength{\topmargin}{-\headsep}     % for header height and separation

%%%%%%%%%%%%%%%%%%%%%%%%%%%%%%%%%%%%%%%%%%%%%%%
%% Stuff for reading a file verbatim.  From TexBook
%%   Usage: \listing{filename}
%%%%%%%%%%%%%%%%%%%%%%%%%%%%%%%%%%%%%%%%%%%%%%%

\def\uncatcodespecials{\def\do##1{\catcode`##1=12 }\dospecials}

\def\setupverbatim{\tt \def\par{\leavevmode\endgraf} \catcode`\`=\active
\obeylines \uncatcodespecials \obeyspaces}

{\obeyspaces\global\let =\ } % let active space = control space

%  {\catcode`\`=\active \gdef`{\relax\lq}}
\def\listing#1{           %\setlength{\baselineskip}{0.01in}
         \par\begingroup\setupverbatim\input#1 \endgroup}

%%%%%%%%%%%%%%%%%%%%%%%%%%%%%%%%%%%
%% Stuff for making text box, from lms/lums
%%%%%%%%%%%%%%%%%%%%%%%%%%%%%%%%%%%

\newlength{\boxwidth}
\newlength{\exfsep}
\newlength{\fboxwidth}
\newlength{\yacktemp}
\newlength{\zerolength}
\setlength{\zerolength}{0pt}

\newcommand{\ExBox}[3]{ % Title pos text
\setlength{\fboxwidth}{\linewidth}
\setlength{\boxwidth}{\linewidth}
\addtolength{\boxwidth}{-\fboxsep}
\addtolength{\boxwidth}{-\fboxsep}
\setlength{\exfsep}{\zerolength}
\setlength{\yacktemp}{\topsep}
\setlength{\topsep}{0pt}
\def\MaybeTitle{#1}
\def\MaybePos{#2}
\ifx\MaybeTitle\empty
  \vspace{-6pt}
\fi
\begin{trivlist}\item[]
  \framebox[\linewidth][t]{
    \begin{minipage}[t]{\boxwidth}{
      \ifx\MaybeTitle\empty
%
      \else
        \begin{minipage}[t]{\boxwidth}
          \if #2c \centerline{
            \rule[-6pt]{0pt}{16pt}
            {\sl \MaybeTitle}
          } \else {
            \rule[-6pt]{0pt}{16pt}
            {\sl \MaybeTitle}
          } \fi
        \end{minipage}
        \noindent {\hrule width \fboxwidth} % \vskip -0.1in
        \setlength{\exfsep}{\fboxsep}
        \addtolength{\boxwidth}{-\fboxsep}
        \addtolength{\boxwidth}{-\fboxsep}
      \fi
      \hspace*{\exfsep}
      \begin{minipage}[t]{\boxwidth}{
        \vspace*{0.1in}{
        \listing{#3}
        }\vspace*{0.1in}
      }\end{minipage}
    }\end{minipage}
  }
\end{trivlist}
\setlength{\topsep}{\yacktemp}
}



\begin{document}

\begin{titlepage}
\vskip 36pt
\begin{center}
{\bf FastHenry USER'S GUIDE}\\
Version 3.0
\end{center}

\vskip 18pt
\begin{center}
M.\ Kamon \hspace*{0.5in} L.M.\ Silveira \hspace*{0.5in} C.\ Smithhisler
\hspace*{0.5in} J.\ White
\end{center}

\vskip 18pt
\begin{center}
Research Laboratory of Electronics \\
Department of Electrical Engineering and Computer Science \\
Massachusetts Institute of Technology \\
Cambridge, MA  02139 U.S.A.
\end{center}
\vskip 18pt
\begin{center}
11 November 1996
\end{center}

\vfill
\noindent This work was supported by Defense Advanced Research Projects Agency
contract N00014-91-J-1698, a National Science Foundation Graduate
Fellowship, and grants from IBM and Digital Equipment Corporation.
\end{titlepage}

\begin{titlepage}
$\;$
\end{titlepage}

\pagenumbering{roman}

\mbox{}
\vfill
\noindent Copyright \copyright\ 1996 Massachusetts Institute of Technology,
Cambridge, MA.  All rights reserved.

\vspace{\baselineskip}
\noindent This Agreement gives you, the LICENSEE, certain rights 
and obligations.
By using the software, you indicate that you have read, understood, and
will comply with the terms.

%\vspace{\baselineskip}
%\noindent Permission to use, copy and modify for internal, 
%noncommercial purposes
%is hereby granted.  Any distribution of this program or any part thereof
%is strictly prohibited without prior written consent of M.I.T.
%
%\vspace{\baselineskip}
%\noindent Title to copyright to this software and 
%to any associated documentation
%shall at all times remain with M.I.T.\ and LICENSEE agrees to preserve
%same.  LICENSEE agrees not to make any copies except for LICENSEE'S
%internal noncommercial use, or to use separately any portion of this
%software without prior written consent of M.I.T.\  LICENSEE agrees to
%place the appropriate copyright notice on any such copies.
%
%\vspace{\baselineskip}
%\noindent Nothing in this Agreement shall be 
%construed as conferring rights to use
%in advertising, publicity or otherwise any trademark or the name of
%``Massachusetts Institute of Technology'' or ``M.I.T.''

\vspace{\baselineskip}
\noindent M.I.T.\ MAKES NO REPRESENTATIONS OR WARRANTIES, 
EXPRESS OR IMPLIED.  By
way of example, but not limitation, M.I.T.\ MAKES NO REPRESENTATIONS OR
WARRANTIES OF MERCHANTABILITY OR FITNESS FOR ANY PARTICULAR PURPOSE OR
THAT THE USE OF THE LICENSED SOFTWARE COMPONENTS OR DOCUMENTATION WILL
NOT INFRINGE ANY PATENTS, COPYRIGHTS, TRADEMARKS OR OTHER RIGHTS.
M.I.T.\ shall not be held liable for any liability nor for any direct,
indirect or consequential damages with respect to any claim by LICENSEE
or any third party on account of or arising from this Agreement or use
of this software.

%\cleardoublepage

\newpage

\tableofcontents
\newpage

\pagenumbering{arabic}
\setcounter{page}{1}

This manual describes FastHenry, a three-dimensional inductance
extraction program.  FastHenry computes the frequency dependent self
and mutual inductances and resistances between conductors of complex
shape.  The algorithm used in FastHenry is an acceleration of the mesh
formulation approach.  The linear system resulting from the mesh
formulation is solved using a generalized minimal residual algorithm
with a fast multipole algorithm to efficiently compute the
iterates. See Appendix~\ref{comfas} for references.

This manual is divided into four sections.  The first section explains
the syntax for preparing input files for FastHenry.  The input files contain the
description of the conductor geometries.
The second section describes how to run the program and process the output.
The third section describes the generation of postscript images to
visualize the three dimensional geometries defined in the input file.
The fourth section shows how to use Matlab to observe current
distribution in reference plances.

The files ``nonuniform\_manual\_*.ps'' constitute  an important
supplement to this 
manual and describe the new routines for specifying a nonuniformly
discretized reference plane.

Information on compiling FastHenry, obtaining the FastHenry source
code, producing this document, and 
corresponding about FastHenry is given in
Appendix~\ref{comfas}. 

Appendix~\ref{changes} summarizes most of the  changes in this
version, 3.0, over the previous version, 2.5.




\section{How to Prepare Input Files}
\label{all}

This section of the manual describes how to prepare input files for
FastHenry.  The input files specify the \underline{discretization} of
\underline{conductor volumes} into \underline{filaments}.  
The input file specifies each conductor as a sequence of straight
\underline{segments}, or elements, connected together at
\underline{nodes}.  Each segment has a finite conductivity and its
shape is a cylinder of rectangular cross
section of some width and height. A node is simply a point in
3-space.  The cross section of each segment can then be broken into a 
number of parallel, thin filaments, each of which will be assumed to
carry a uniform cross section of current along its length.
The first two parts of this section describe the file format through
simple examples. A detailed description is in the third part, and more
complex examples can be found in the last section and later in the manual.

\subsection{A Simple Example}

The following is an input file
which calculates the loop inductance of four segments 
nearly tracing the perimeter of a square.  It is described more
thoroughly on the next page.

\begin{quote}
\begin{verbatim}  
**This is the title line. It will always be ignored**. 
* Everything is case INsensitive
* An asterisk starts a comment line.

* The following line names millimeters as the length units for the rest 
* of the file.
.Units MM

* Make z=0 the default z coordinate and copper the default conductivity.
* Note that the conductivity is in units 1/(mm*Ohms), not 1/(m*Ohms)
* since the default units are millimeters.
.Default z=0 sigma=5.8e4

* The nodes of a square (z=0 is the default)
N1 x=0 y=0
N2 x=1 y=0
N3 x=1 y=1
N4 x=0 y=1
N5 x=0 y=0.01

* The segments connecting the nodes
E1 N1 N2 w=0.2 h=0.1
E2 N2 N3 w=0.2 h=0.1
E3 N3 N4 w=0.2 h=0.1
E4 N4 N5 w=0.2 h=0.1

* define one 'port' of the network
.external N1 N5

* Frequency range of interest.
.freq fmin=1e4 fmax=1e8 ndec=1

* All input files must end with:
.end
\end{verbatim}
\end{quote}

As described in the comments, {\tt .Units MM} defines all
coordinates and lengths to be in millimeters.  All lines with an
{\tt N} in the first column define nodes, and all lines starting
with {\tt E} define segments.  In particular, the line
\begin{verbatim}
E1 N1 N2 w=0.2 h=0.1
\end{verbatim}
defines segment E1 to extend from node {\tt N1} to {\tt N2} and have a
width of 0.2 mm and height of 0.1 mm as drawn in Figure \ref{E1_2}.
If the $n \times n$ impedance matrix, $Z(\omega)$, 
for an $n$-conductor problem is
thought of as the parameters describing an $n$-port network, then the line 
\begin{verbatim}
.external N1 N5
\end{verbatim}
defines {\tt N1} and {\tt N5} as one port of the network.  In this
example, only one port is specified, so the output will be a $ 1
\times 1 $ matrix containing the value of the impedance looking into
this one port.

\begin{figure}
\centerline{
\psfig{figure=/u0/matt/induct/manual/E1_2.ps,width=3.0in}
}
\caption{Example Segment for Sample Input File}
\label{E1_2}
\end{figure}

FastHenry
calculates $Z(\omega)$ at the discrete frequencies described by the
line
\begin{verbatim}
.freq fmin=1e4 fmax=1e8 ndec=1
\end{verbatim}
where {\tt fmin} and {\tt fmax} are the minimum and maximum
frequencies of interest, and {\tt ndec} is the number of desired frequency
points per decade.  In this case, $Z(\omega)$ will be calculated at
$10^4, 10^5, 10^6, 10^7,$ and $10^8$ Hz.   All input files must end with
{\tt .end}.

In the above example, FastHenry created one filament per segment since
no discretization of the segments into filaments
was specified.  In order to properly model non-uniform cross sectional
current due to skin and proximity effects, a finer discretization
must be used.  Finer filaments are easily specified in the segment
definition.  For example, replacing the definition for {\tt E1}
with
\begin{verbatim}
E1 N1 N2 w=0.2 h=0.1 nhinc=5 nwinc=7
\end{verbatim}
specifies that {\tt E1} is to be broken up into thirty-five filaments:
five along its height ({\tt nhinc=5}) and seven along its width ({\tt
nwinc=7}).  See Figure \ref{E1}.

\begin{figure}
\centerline{
\psfig{figure=/u0/matt/induct/manual/E1.ps,width=3.0in}
}
\caption{Segment discretized into 35 filaments}
\label{E1}
\end{figure}

\subsection{Another Simple Example}
\label{another}

To continue teaching by example, what follows is an example of
computing the loop inductance of an L
shaped trace over a ground plane with the trace's return path through
the plane as shown in 
Figure~\ref{simple_gp}.  Note that a line beginning with `+' is a
continuation of the previous line.

\begin{figure}
\centerline{
\psfig{figure=/u0/matt/induct/examples/formanual/simple_gp_bb.ps,width=3.2in}
}
\caption{L-shaped trace over ground plane}
\label{simple_gp}
\end{figure}

\noindent
\listing{/u0/matt/induct/examples/formanual/simple_gp.inp}

%\ExBox{simple\_gp.inp}{c}{/u0/matt/induct/examples/formanual/simple_gp.inp}

In this example, the ground plane is a $1000 mil \times 1000 mil$ sheet of
copper (by default) defined by three of its four corners, $(0,0,0)$,
$(1000,0,0)$, $(1000,1000,0)$.  The plane is $1.2$~mils thick and
its discretization is specified by {\tt seg1} and {\tt seg2} (see
Section~\ref{planes}).  The discretization of the plane forms a grid of
nodes and interconnecting segments.  {\tt nin} and {\tt nout} refer to
the internal nodes of the plane which are closest to $(800,800,0)$ and
$(0,200,0)$, respectively.  
 
To model skin effects on the trace, the default for the number of
filaments per segment is set to $8$.

To compute loop inductance, the node at one end of the trace is
shorted to the plane by declaring the
the node to be ``electrically equivalent'' to the node directly underneath.
\begin{verbatim}
.equiv nin n3
\end{verbatim}
and then the other end is declared as the ``port'' with the {\tt
.external} statement.

In this case, a single loop impedance will be computed.  If, however, the
{\tt .equiv} and {\tt .external} were replaced with
\begin{verbatim}
.external N1 N3
.external nout nin
\end{verbatim}
then the partial inductances and resistances of these two paths would
be computed yielding a $2 \times 2$ impedance matrix.

Computing the impedance for only two frequencies is useful for
visualization of the distribution of current in the reference plane 
as described later in Section~\ref{viewcur}.


\subsection{Input File Syntax}
\label{syntax}

The previous section described many of the basics required for an
input file.  This section gives a more complete and detailed description of the
input file format and should serve as a reference.

Some general facts about file syntax:
\begin{itemize}
        \item Lines are processed sequentially.
        \item Uppercase is converted to lower case.
        \item ``*'' at the beginning of a line marks a comment line.
        \item Lines are restricted to 1000 characters but can be
continued with a ``+'' as the first character of subsequent lines.
Intervening ``*'' lines are allowed in this release.
        \item The first line in the file is considered the title line
and is ignored. It is recommended that this line start with an ``*''
for future compatibility and file concatenation.
        \item The file must end with the {\tt .End} keyword.
        \item Names of objects are limited to 80 characters.
\end{itemize}

In general, each line of the input file will either define some
geometrical object, such as a node or segment, or it will specify some
program parameter.  
All input lines that define geometrical objects begin with a letter
defining their type, and then some unique alphanumeric string of up to
80 characters.  For
instance, all node definitions begin with the letter {\tt N}.  
This sets object lines apart
from parameter specification lines which begin with a period, ``.''.

The remainder of this section will describe all
possible input lines.  In the following description, any argument
enclosed in `[ ]' indicates an optional argument.  If not included on
the input line, the actual value used for this argument will be either
the program default, or the user default defined by the {\tt .Default}
keyword (described below). 

\subsubsection{Node Definitions}

\begin{verbatim}
Syntax:  Nstr [x = x_val] [y = y_val] [z = z_val]
\end{verbatim}

This defines a node called {\tt Nstr} where {\tt str} is any
alphanumeric string.  The first character on the line must begin with
an {\tt N} for this to be interpreted as a node definition.
The node will have location $ (x\_val, y\_val, z\_val) $ where each
coordinate has units defined by the {\tt .Units} keyword.  

Any of the coordinates can be omitted assuming that a default value
has been previously specified with the {\tt .Default} keyword.
Otherwise, an error will occur and the program will exit.

\subsubsection{Segment Definitions}
\label{segdef}

\begin{verbatim}
Syntax:  Estr node1 node2 [w = value] [h = value] [sigma, rho = value] 
                 [wx = value wy = value wz = value] 
                 [nhinc = value] [nwinc = value] [rh = value] [rw = value]
\end{verbatim}

This defines a segment called {\tt Estr} where {\tt str} is any
alphanumeric string.  The first character on the line must begin with
the letter {\tt E} for this to be interpreted as a segment definition.
The segment will extend from node {\tt node1} to node {\tt node2}
which must be previously defined node names.  {\tt h} and {\tt w} are
the segment height and width.  Either {\tt sigma}, the conductivity,
or {\tt rho}, the resistivity, can be specified for the segment.

Discretization of the segment into multiple, parallel thin filaments
is specified with the {\tt nhinc} and {\tt nwinc} arguments. {\tt
nhinc} specifies the number of filaments in the height direction, and
{\tt nwinc}, the number in the width direction.  Both must be
integers. See Figure \ref{E1}.  By default, the ratio of adjacent
filaments is 2.0 as in Figure \ref{E1}, however this can changed with
the {\tt rh} and {\tt rw} arguments which specify the ratio 
in the height and width direction, respectively.
While an automatic technique of dividing a section into filaments 
based on the skin depth may be a better approach, no such feature is
available.  See Section~\ref{discret} for guidance in choosing
filament discretizations.


To specify the orientation of the cross section, {\tt wx}, {\tt wy},
and {\tt wz} represent any vector pointing along the width of the
segment's cross section.  If these are omitted, the width vector is
assumed to lie in x-y plane perpendicular to the length.  If the
length direction is parallel to the z-axis, then the width is assumed
along the x-axis.

{\tt h} and {\tt w} can be omitted provided they are assigned a default
value in a previous {\tt .Default} line.

{\tt nhinc}, {\tt nwinc}, and {\tt sigma} or {\tt rho} can be omitted,
and if not previously given a default value, then 1, 1, and the
conductivity of copper, respectively, are used as default values. 

Note that the nodes used to define the elements must be defined
under {\bf Node Definitions} described above and cannot be reference
plane nodes (See Section~\ref{planes} for a description of reference
planes).  
To connect to a reference plane, the user must instead
create a new node at the desired location 
as described in {\bf Node Definitions} and then use the {\tt .Equiv}
keyword to equivalence the reference plane node and the new node. For
instance, the following is not allowed:
\begin{quote}
\begin{verbatim}
g1 x1 = 0     y1 = 0     z1 = 0
+  x2 = 1000  y2 = 0     z2 = 0
 .
 .
+ n_gp (5,5,0)

N1 x=5 y=5 z=10
E1 N1 n_gp 
\end{verbatim}
\end{quote}

\noindent The legal way to define the segment would be
\begin{quote}
\begin{verbatim}
g1 x1 = 0     y1 = 0     z1 = 0
+  x2 = 1000  y2 = 0     z2 = 0
 .
 .
+ n_gp (5,5,0)

N1 x=5 y=5 z=10
N2 x=5 y=5 z=0
E1 N1 N2
.equiv N2 n_gp
\end{verbatim}
\end{quote}

\subsubsection{{\tt .Units} keyword}

\begin{verbatim}
Syntax:  .Units unit-name
\end{verbatim}

This specifies the units to be used for all subsequent coordinates and lengths
until the end of file or another {\tt .Units} specification is
encountered.  Allowed units are kilometers, meters, centimeters, millimeters,
micrometers, inches, and mils with {\tt unit-name} specified as 
{\tt km, m, cm, mm, um, in,
mils}, respectively.  

Note that this keyword affects the expected units for the conductivity
and resistivity.

\subsubsection{{\tt .Default} keyword}

\begin{verbatim}
Syntax:  .Default [x = value] [y = value] [z = value] [w = value] 
                  [h = value] [sigma, rho = value]
                  [nhinc = value] [nwinc = value] [rh = value] [rw = value]
\end{verbatim}

This keyword specifies default values to be used for subsequent object
definitions.  A certain default value is used until the end of the file, or
until it is superseded by another {\tt .Default} line changing that value.

\subsubsection{{\tt .External} keyword}

\begin{verbatim}
Syntax:  .External node1 node2 [portname]
\end{verbatim}

This keyword specifies node {\tt node1} and node {\tt node2} as a
terminal pair or port whose impedance parameters should be calculated 
for the output impedance matrix.  If an input file includes $ n $ {\tt
.External} lines, then the impedance matrix will be an $ n \times n $
complex matrix.  The port can be given a name by specifying a single
word string {\tt portname}.  This name is used to reference this port
in the output file, {\tt Zc.mat}, and is also used with 
the command line {\tt -x} option described in Section~\ref{clo}.

This keyword effectively places a voltage source between these nodes
and will later use the current through that source to determine an
entry in the admittance matrix.  The first node specified is the
positive node.  Note that it is up to the user to insure that there
are NO loops of only voltage sources.  Also, a voltage source with no
possible return path will always carry zero current 
producing a row of zeros in the admittance matrix.  The output
impedance matrix will thus be nonsense (NaN).

\subsubsection{{\tt .Freq} keyword}

\begin{verbatim}
Syntax:  .Freq fmin=value fmax=value [ndec = value]
\end{verbatim}

This keyword specifies the frequency range of interest.  
{\tt fmin} and {\tt fmax} are the minimum and maximum
frequencies of interest, and {\tt ndec} is the number of desired frequency
points per decade.  FastHenry must perform the entire solution process
for each frequency.

Note that {\tt ndec} need not be an integer.  For instance, 
\begin{verbatim}
.freq fmin=1e3 fmax=1e7 ndec=0.5
\end{verbatim}
will have FastHenry calculate impedance matrices for $f = 10^3, 10^5, $
and $ 10^7$ Hz. 

If {\tt fmin} is zero, FastHenry will run only the DC case
regardless of the value of {\tt fmax}.  

\subsubsection{{\tt .Equiv} keyword}

\begin{verbatim}
Syntax:  .Equiv node1 node2 node3 node4 ...
\end{verbatim}

This keyword specifies that nodes {\tt node1, node2, node3, node4,...}
are to be considered electrically equivalent yet maintain their separate
spatial coordinates.  The nodes are effectively `shorted' together.  
If any of the node names are not previously defined, then they become
pseudonyms for those nodes in the list which are defined.
Note that the current flow between equivalenced nodes is not
electromagnetically modeled.

\subsubsection{{\tt .End} keyword}

\begin{verbatim}
Syntax:  .End
\end{verbatim}

This keyword specifies the end of the file.  All subsequent lines are
ignored.  This line must end the file.

\subsubsection{Reference Plane definitions}
\label{planes}

{\bf Note:} It is recommended that nonuniformly discretized planes be
used instead of what is described here if possible.   
See the document ``nonuniform\_manual\_*.ps''.
Uniform planes described below can still be used if they satisfy your
needs. Also, for new users, understanding uniform planes is necessary to
understand the nonuniform\_manual.ps document.

\vspace{0.1in}
\noindent {\bf Uniformly discretized planes}

\begin{verbatim}
Syntax:  Gstr x1=value y1=value z1=value x2=value y2=value z2=value 
              x3=value y3=value z3=value
              thick=value seg1=value seg2=value 
              [segwid1 = value] [segwid2 = value] 
              [sigma, rho = value] 
              [nhinc=value] [rh=value] 
              [relx=value] [rely=value] [relz=value]
              [Nstr1     (x_val,y_val,z_val) ] 
              [Nstr2     (x_val,y_val,z_val) ]
              [Nstr3     (x_val,y_val,z_val) ].....
              [hole <hole-type> (val1,val2,....)]              
              [hole <hole-type> (val1,val2,....)].....
\end{verbatim}

This defines a uniformly discretized reference plane of finite extent
and conductivity called {\tt Gstr} where {\tt str} is any alphanumeric
string.  The first character on the line must begin with the letter
{\tt G} for this to be interpreted as a reference (originally
``ground'') plane definition.  The three locations $(x1,y1,z1),
(x2,y2,z2), (x3, y3, z3)$ mark three of the four corners of the plane
in either clockwise or counterclockwise order. FastHenry will
determine the fourth corner assuming the first three are corners of a
rectangle.  The thickness of the plane is specified with the {\tt
  thick} argument and the conductivity with either the {\tt sigma} or
{\tt rho} argument.  Note that current is modeled to flow two
dimensionally, i.e., not in the thickness direction.

\begin{figure}
\centerline{
%\psfig{figure=/u0/matt/induct/manual/gp.ps,width=3.0in}
\psfig{figure=/u0/matt/induct/examples/formanual/gp_bb.ps,width=3.5in}
}
\caption{Discretization of a Reference Plane. Segments are one-third actual width.}
\label{gp}
\end{figure}

The reference plane is approximated by first laying down a
grid of nodes, and then with segments, connecting every
node to its adjacent nodes excluding diagonally adjacent nodes.  Each
segment is given a height equal to the specified thickness of the
plane and, by default, width equal to the node 
spacing.  This choice of width completely fills the
space between segments.  Figure \ref{gp} shows a sample reference plane
with segments that are one-third full width for illustration.

{\tt seg1} and {\tt seg2} specify the number of segments along each
edge of the plane.  {\tt seg1} is the number of segments along
the edge from $(x1,y1,z1) $ to $(x2,y2,z2)$ and {\tt seg2}, the number
along the edge from $(x2,y2,z2) $ to $ (x3, y3, z3)$.  Thus the total
number of nodes created will be ({\tt seg1}+1)*({\tt seg2}+1), and the
total number of segments created will be $ ({\tt seg1} + 1)*{\tt seg2}
+ {\tt seg1}*({\tt seg2} + 1)$ .  Note that the line from the node at
$(x1,y1,z1)$ to the node at
$(x2,y2,z2)$ define the edge of the plane, however segments that
run from $(x1,y1,z1) $ to $(x2,y2,z2)$ will overhang the edge by half
their width since nodes are defined at the center of segments. This
slight overapproximation to the width of the plane is avoided in the
nonuniform discretization code described in {\tt nonuniform\_manual\_*.ps}.

{\tt nhinc} and {\tt rh} can be used to specify a number of filaments
for discretization of each segment along the thickness (See
Section~\ref{segdef}). This could be used for modeling nonuniform
current across the thickness.  If {\tt nhinc} is omitted, the value
of 1 is used regardless of the {\tt .Default}
setting.  If {\tt rh} is omitted, the default value is used.

{\bf Note:} If you do not want to use segwid1 and segwid2 or holes as
described below, you can use the nonuniform plane
definition as described in ``nonuniform\_manual.ps''.

\vspace{0.3in}
{\bf \noindent Connections to the plane}

Since the reference plane nodes are generated internally, 
there is no way to
refer to them later in the input file.  The exceptions to this are the
nodes explicitly referenced in the reference plane definition.  The
argument
\begin{verbatim}
Nstr1     (x_val,y_val,z_val)
\end{verbatim}
where {\tt str1} is an alphanumeric string, will cause all subsequent
references to {\tt Nstr1} to refer to the node in the plane closest to
the point $(x\_val,y\_val,z\_val)$.  Note that {\em no spaces} are
allowed between the `()'.  The referencing is accomplished internally
by effectively doing
\begin{verbatim}
.Equiv Nstr1 <internal-node-name>
\end{verbatim}
where {\tt <internal-node-name>} is the internal node name of the
nearest reference plane node.  

If one or more of {\tt relx}, {\tt rely}, and {\tt relz} are
specified, then the above node referencing instead chooses the node closest to
$(x\_val + relx, y\_val + rely, z\_val + relz)$.  In other words, {\tt
relx}, {\tt rely}, and {\tt relz} default to $0$ if not specified.

A coarsely discretized plane (small {\tt seg1}, {\tt seg2}) may cause
two different node references to refer to 
the same reference plane node.  FastHenry will warn of such an event, but
it is not an error condition.

{\bf Note:} It is recommended that connections to the plane be done
with the ``contact equiv\_rect'' utility for nonuniformly discretized
planes as described in ``nonuniform\_manual.ps''.

\vspace{0.3in}
\noindent {\bf Meshed Planes}

By default, the width of the segments of the plane are chosen to fill
the space between adjacent segments.  However, to model meshed planes
the width of the segments can be chosen to be smaller
with the {\tt segwid1} and {\tt segwid2} arguments.  This
gives the plane the appearance of a ``mesh'' as shown is Figure
\ref{gp}.  {\tt segwid1} specifies the width of the segments that are in the
direction parallel to the plane edge from $(x1,y1,z1)$ to
$(x2,y2,z2)$.  {\tt segwid2} specifies the width along the edge from
$(x2,y2,z2)$ to $(x3,y3,z3)$.  

Note that if the section of conductor between meshes is large compared
to the size of the mesh holes, this method of discretization may be
too coarse.  
%In other words, if the proportion of area of the plane that
%is conductor is large compared to the area that is hole, the single
%segment approximation to the current around a mesh is poor.  
For instance, assume the mesh consists of only nine holes on a 3x3
grid.  In such a situation, consider using the {\tt hole} functions
described below to make the specific holes on a plane
with {\tt seg1,seg2}~$>> 3$.  See also Section~\ref{sec:holey}.

\vspace{0.3in}
\noindent {\bf Holes in the plane}

Holes can be specified in the plane with
\begin{verbatim}
hole <hole-type> (val1,val2,val3,...)
\end{verbatim}
where {\tt <hole-type>} is the hole type and the {\tt valn}'s make the list 
of arguments to be sent to the hole generating function.  Holes are
generated by first removing reference plane nodes, and then removing all 
segments connected to those nodes.
The following describes the available hole generating functions:
\begin{verbatim}
hole point (x,y,z)
\end{verbatim}
removes the node nearest to the point (x,y,z).  
\begin{verbatim}
hole rect (x1,y1,z1,x2,y2,z2)
\end{verbatim}
removes a rectangular region whose opposite corners are the
nodes in the plane nearest (x1,y1,z1) and (x2,y2,z2).
\begin{verbatim}
hole circle (x,y,z,r)
\end{verbatim}
removes the nodes contained within the circle of radius {\tt r}
centered at (x,y,z).
\begin{verbatim}
hole user1 (val1,val2,...)
\end{verbatim}
calls the user defined function hole\_user1() to remove nodes.  
user1 - user7 are available.  The user can add the functions to the
source file {\tt hole.c} contained in the release.  See the functions
hole\_rect(), hole\_point(), and hole\_circle() to see examples of the
format for
writing user hole functions.

Any shaped hole can be formed with a combination of hole directives.
A few exceptions exist, however.  Forming a hole that isolates or
nearly isolates a
section of the plane is not allowed.  FastHenry warns of this with:
\begin{verbatim}
Warning: Multiple boundaries found around one hole region
  possibly due to an isolated or nearly isolated region of conductor.
  This may lead to no unique solution.
\end{verbatim}

\noindent If a nearly isolated section of conductor is truly desired, consider
defining two separate planes and connecting them.

Holes may not be produced as expected if the discretization of the
plane is coarse.  It is recommended that the plane be viewed by using
the options ``{\tt -f simple -g on}'' options to generate a {\tt zbuf}
file which can be used to generate a postscript image of the plane as
described in Section~\ref{postscript}.

\subsection{Other Examples}

This section describes other examples using more of the features of
FastHenry.  Some examples were created by the authors and some
contributed from users.  All files are available with the FastHenry
source code.

\subsubsection{Printed Circuit Board}

The example file {\tt gpexamp\_copper.inp} shown in
Figure~\ref{gpexamp_copper} is a portion of a printed circuit board from Digital Equipment Corporation.  This structure is to
be placed underneath a PGA package.  This geometry includes 2
reference planes sandwiching a large number of copper lines which lead
to the center where the package would be placed.  The set of copper
lines are grouped into 18 groups according to functionality (see the
many .equiv statements at the bottom of the file).  

\begin{figure}
\centerline{
\psfig{figure=/u0/matt/induct/gpexamp/gpexamp.ps,width=5.0in}
}
\caption{Printed Circuit Board example}
\label{gpexamp_copper}
\end{figure}

\subsubsection{Vias and Meshed Planes}

The example file {\tt vias.inp} shown in Figure~\ref{vias} shows three
vias passing through a meshed ground plane.  To form the mesh
structure, the standard reference plane definition is used with the
{\tt segwid1} and {\tt segwid2} arguments.
Some of the dimensions for this example were taken from B.J. Rubin, ``An
electromagnetic approach for modeling high-performance computer
packages,''  {\em IBM J. Res. Dev.}, Vol.~34, No.~4, July 1990.

\begin{figure}
\centerline{
\psfig{figure=vias.ps,width=5.0in}
}
\caption{Vias through a meshed ground plane}
\label{vias}
\end{figure}

\subsubsection{Holey ground plane}
\label{sec:holey}

The example file {\tt holey\_gp.inp} shown in Figure~\ref{holey}
is a punctured version of the reference plane example of
section~\ref{another}.  The holes of the plane are formed with 
the {\tt rect} and {\tt circle} hole directives. 

\begin{figure}
\centerline{
\psfig{figure=holey_gp_bb.ps,width=5.0in}
}
\caption{Trace over Ground Plane with holes}
\label{holey}
\end{figure}

\subsubsection{Pin-connect}

The example file {\tt pin-connect.inp} shown in
Figure~\ref{pin-connect} is thirty-five pins of a 68-pin cerquad pin
package from Digital Equipment Corporation.  
%It is not clear why this contribution does not contain all 68 pins.

\begin{figure}
\centerline{
\psfig{figure=pin-connect.ps,width=5.0in}
}
\caption{Thirty-Five Pins of a 68 pin Cerquad Package}
\label{pin-connect}
\end{figure}

\subsubsection{Right Angle Connector}

The example file {\tt 30pin.inp} from Teradyne Connection Systems
Inc. shown in Figure~\ref{30pin} is a PCB right angle connector
plugged into its thirty corresponding pins.  The V-shaped clamping
portion and right angle bend make up the connector as shown in the
single pin illustration of Figure~\ref{one_of_30pin}.  In reality, the
pin protrudes through the V-shaped clamp, but that section conducts no
current and is not included.

Each of the thirty pins is a different shade of gray.  The
white portions of the V are also part of each pin but appear white due
to the limits of the shading algorithm.


\begin{figure}
\centerline{
\psfig{figure=30pin.ps,width=5.0in}
}
\caption{Thirty pin right angle connector}
\label{30pin}
\end{figure}

\begin{figure}
\centerline{
\psfig{figure=one_of_30pin_bb.ps,width=3.0in}
}
\caption{One pin of the thirty pin right angle connector}
\label{one_of_30pin}
\end{figure}

\subsubsection{Photodetector}

The example file {\tt msm.inp} shown in Figure~\ref{msm} is a
metal-semiconductor-metal photodetector from NASA Langley Research
Center and University of Virginia.  These devices consist of a set of
closely spaced, interdigitated electrodes deposited on an optically
active semiconductor layer.  Every other electrode is biased to the
same potential, so that any two adjacent electrodes are at opposite
electrostatic potentials.  Becuase the electrodes are closely spaced,
the electric fields between them are quite strong with application of
low to moderate bias voltages. As a result, any carriers generated by
absorption of light between the fingers are transported to the
electrodes extremely rapidly giving fast response to short pulses of
light.

\begin{figure}
\centerline{
\psfig{figure=msm.ps,width=4.0in}
}
\caption{Metal-semiconductor-metal Photodetector}
\label{msm}
\end{figure}

\section{Running FastHenry}

The basic form of the FastHenry program command line is
\begin{quote}
\begin{verbatim}
fasthenry [<input file>] [<Options>]
\end{verbatim}
\end{quote}
Usually only the {\tt input file} is specified.
For example, the command
\begin{quote}\tt
fasthenry pin-connect.inp
\end{quote}
runs {\tt fasthenry} on the example pin connect structure.

Information about the input file and other FastHenry information are
sent to the standard output.  The impedance matrices for the
frequencies specified in the input file will be placed in the text
file {\tt Zc.mat}.  This file also lists the correspondence between
columns in the impedance matrix and the ports specified in the input
file.  The source file {\tt ReadOutput.c} described in
Section~\ref{processing} is provided as a sample program for reading
the output file for postprocessing.

\subsection{Example Run}

%With the release are example structures {\tt 30pin.inp}, {\tt
%pin-connect.inp}, {\tt onebargp.inp}, {\tt hole.inp}, {\tt
%broken.inp}, and {\tt together.inp}. 
This section contains a sample run of FastHenry for the 30 pin
connector example, {\tt 30pin.inp} shown in Figure~\ref{30pin}.  Comments
describing the output appear along the right margin in addition to the
{\tt FastHenry} command line option which would change the described
setting.

{\small
\input{30pin.run.hilb}
}

\subsection{Processing the Output}
\label{processing}

\subsubsection{The impedance matrix}

The file {\tt Zc.mat} is a text file containing the impedance matrices
for the frequencies requested.  The file {\tt ReadOutput.c} in
directory {\tt src/misc/} is an example program for reading the text
file for whatever processing is necessary.  It contains the function
{\tt ReadZc()} which reads from a file and returns a linked list of
impedance matrices and their corresponding frequencies. See the source
file for more details.  This function can be extracted and included in
whatever program the user desires.

The function {\tt main()} is provided as an example use of
{\tt ReadZc()}.  For each of the matrices, it divides the
imaginary part by the frequency to give the matrix $R + jL$ and then
dumps the result to the standard output.

{\tt ReadOutput.c} can be compiled by typing
\begin{verbatim}
cc -o ReadOutput ReadOutput.c
\end{verbatim}

Here is a sample of its output after processing {\tt Zc.mat} produced
by running {\tt fasthenry} on example file {\tt onebargp.inp}:

{\small
\begin{verbatim}
prompt % ReadOutput Zc.mat
Not part of any matrix: Row 2 :  nodein  to  nodeout
Not part of any matrix: Row 1 :  nb1  to  nbout
Reading Frequency 10000
Reading Frequency 100000
Reading Frequency 1e+06
Reading Frequency 1e+07
Reading Frequency 1e+08
freq = 1e+08
Row 0: 0.00112838+3.50478e-08j -2.21062e-05-7.0003e-09j 
Row 1: -2.23118e-05-6.99564e-09j 4.83217e-05+2.02958e-08j 
freq = 1e+07
Row 0: 0.00112837+3.50478e-08j -2.21055e-05-7.0003e-09j 
Row 1: -2.2311e-05-6.99564e-09j 4.83217e-05+2.02958e-08j 
freq = 1e+06
Row 0: 0.0011272+3.50501e-08j -2.20335e-05-7.00045e-09j 
Row 1: -2.22379e-05-6.99578e-09j 4.83168e-05+2.02958e-08j 
freq = 100000
Row 0: 0.00106093+3.51782e-08j -1.7938e-05-7.00794e-09j 
Row 1: -1.80765e-05-7.00341e-09j 4.80425e-05+2.02964e-08j 
freq = 10000
Row 0: 0.000955377+3.58555e-08j -1.25489e-05-7.01913e-09j 
Row 1: -1.25828e-05-7.01518e-09j 4.75474e-05+2.03045e-08j 
prompt %
\end{verbatim}
}

\subsubsection{Creating Equivalent Circuits}

Two approaches for generating spice equivalent circuits are available:

\vspace{0.1in}
\noindent {\bf Approach 1: An equivalent circuit for a single frequency.}
  \begin{itemize}
    \item Run fasthenry for EXACTLY one frequency.
    \item Compile (with some C compiler) the file MakeLcircuit.c in
      the directory fasthenry-3.0/src/misc/ 
    \begin{verbatim}
      cc -o MakeLcircuit MakeLcircuit.c -lm
    \end{verbatim}
    \vspace{-0.1in}
    \item Run MakeLcircuit on the Zc.mat produced by FastHenry
    \begin{verbatim}
      MakeLcircuit Zc.mat > my_circuit.spice
    \end{verbatim}
  \end{itemize}

This produces a circuit where nodes 0 and 1 correspond to the plus and
minus nodes of the first
FastHenry port (also called .external).  2 and 3 correspond to the second port.
4 and 5 the third, etc.

\vspace{0.1in}
\noindent {\bf Approach 2: A circuit which models the frequency
  dependent resistances and   inductances}

This approach generates a reduced order model for the system using the 
   arguments ``-r n -M'' to FastHenry.  This gives an equivalent
   circuit which is valid for a range of
   frequencies.
\begin{itemize}
   \item Run FastHenry: {\tt fasthenry -r n -M myinput.inp}\\
     Where ``n'' is replaced with the desired order, say 20.
     This produces the file {\tt equiv\_circuitROM.spice} containing a single
     spice subcircuit.  
   \item Include the subcircuit in a spice input file and connect
     devices to it.  
  \end{itemize}

The subcircuit will be named {\tt ROMequiv} and
its port nodes are p0 and m0 for the plus and minus terminals of
port 0, p1 and m1 for port 1, etc. See the header of the
equiv\_circuitROM.spice file for more description.  Any suffix
specified with ``-S'' will be appended to both the subcircuit
name, {\tt ROMequiv}, and the subcircuit filename, {\tt
  equiv\_circuitROM.spice}.

Comments:  Approach 1 will not model frequency dependent resistance
and inductance since it gives an R and L at the single specified
frequency.  Approach 2 will model the full frequency dependent effects
up to some frequency 
where that frequency is higher for a higher chosen order, $n$.
The reduced order model of Approach 2 is converted to a circuit from its
state-space form through capacitors, resistors, and VCCSs so insight
can only be gained by simulating the subcircuit in spice rather than
looking inside the subcircuit file.  An example is provided below. For
a description of reduced 
order modeling for inductance computation see the paper {\tt
tchmt-epep94.ps} in the {\tt pub/fasthenry} directory at
the {\tt rle-vlsi.mit.edu} ftp site:

\begin{quote}
\noindent
 L. Miguel Silveira and Mattan Kamon and Jacob K. White, ``Efficient
 Reduced-Order Modeling of Frequency-Dependent Coupling Inductances
 associated with 3-D Interconnect Structures'', {\em IEEE Transactions on 
    Components, Hybrids, and Manufacturing Technology, Part B:
    Advanced Packaging}, Vol. 19, No. 2, May 1996, pp. 
    283-288. 
\end{quote}

\noindent {\bf An example of Approach 2}

The file {\tt examples/pin-con7.inp} is 7 pins of a larger package
shown in Figure~\ref{pin-connect}. Each conductor is discretized into
filaments to capture skin and proximity effects up to around $10^{10}$
Hz. One can
generate an equivalent circuit via Approach 2 with
\begin{verbatim}
  fasthenry -r 20 -M pin-con7.inp -S_pin_con7_r20
\end{verbatim}
which produces the file {\tt equiv\_circuitROM\_pin\_con7\_r20.spice}
which is an equivalent circuit for a 140th ($20*7$) order model. To test
its accuracy, consider computing impedance matrices at many
frequencies with
\begin{verbatim}
  fasthenry pin-con7.inp -S_pin_con7
\end{verbatim}
which produces {\tt Zc\_pin\_con7\_r20.mat} which is the impedance matrix 
for frequencies $1, 10, 10^2, \ldots, 10^{12}$.  A spice file is
included in the {\tt examples} directory to test this subcircuit and
put the results in {\tt rom\_check\_con7\_r20.out}:
\begin{verbatim}
spice3 -b -o rom_check_con7_r20.out  rom_check_con7_r20.ckt
\end{verbatim}

The spice file in {\tt rom\_check\_con7\_r20.ckt} has a spice {\tt
  .include} line to include the subcircuit of file {\tt
  equiv\_circuitROM\_pin\_con7\_r20.spice}.  It then does an AC analysis
of the response of the circuit when 1 Amp is applied to port 0 and 0
Amp to the other ports.  The results correspond to column 1 of the
impedance matrix, $Zc$. 

A comparison of the accuracy of entry (1,1) of the impedance matrix is
given in Figure~\ref{rom-fig} for both ``-r 7'' and
``-r 20''. 

\begin{figure}
\centerline{
\psfig{figure=res_rom.ps,width=3.0in}
\psfig{figure=ind_rom.ps,width=3.0in}
}
\caption{Comparison of impedance computation done via explicit
  solution at specified frequency points and reduced-order models}
\label{rom-fig}
\end{figure}

%\begin{table}
%\begin{center}
%\begin{tabular}{|r|r||r|r|}\hline
%Frequency & \multicolumn{1}{c|}{Zc\_...mat} &\multicolumn{1}{c|}{``-r 7''} & \multicolumn{1}{c|}{``-r 20''} \\ \hline
%   1.0000e+00  &  8.3303e-02  &  8.3303e-02 & 8.3303e-02\\ \hline
%   1.0000e+01  &  8.3303e-02  &  8.3303e-02 & 8.3303e-02\\ \hline
%   1.0000e+02   & 8.3303e-02   & 8.3303e-02  &8.3303e-02\\ \hline
%   1.0000e+03  &  8.3303e-02  &  8.3303e-02 & 8.3303e-02\\ \hline
%   1.0000e+04   & 8.3303e-02   & 8.3303e-02 & 8.3303e-02\\ \hline
%   1.0000e+05   & 8.3304e-02  &  8.3304e-02 & 8.3304e-02\\ \hline
%   1.0000e+06  &  8.3436e-02  &  8.3436e-02 & 8.3436e-02\\ \hline
%   1.0000e+07   & 9.3334e-02   & 9.3353e-02  &9.3350e-02\\ \hline
%   1.0000e+08   & 2.1694e-01  &  2.1018e-01 & 2.1692e-01\\ \hline
%   1.0000e+09  &  4.6670e-01  &  2.8679e-01 & 4.6431e-01\\ \hline
%   1.0000e+10   & 5.5362e-01   & 2.8836e-01  &5.3207e-01\\ \hline
%   1.0000e+11  &  5.5519e-01    &2.8837e-01  &5.3306e-01\\ \hline
%   1.0000e+12   & 5.5520e-01  &  2.8837e-01 & 5.3307e-01\\ \hline
%\end{tabular}
%\caption{Resistance as a function of frequency by computing it
%  pointwise (Zc.mat) or simulating with a state-space equivalent
%  circuit of order 7 and 20}
%\label{Res_rom}
%\end{center}
%\end{table}

%\begin{table}
%\begin{center}
%\begin{tabular}{|r|r||r|r|}\hline
%Frequency & \multicolumn{1}{c}{Zc\_...mat} &\multicolumn{1}{c}{``-r 7''} & \multicolumn{1}{c}{``-r 20''} \\ \hline
%   1.0000e+00  & 5.7310e-08   & 5.7351e-08 & 5.7351e-08\\ \hline
%   1.0000e+01   & 5.7310e-07  &  5.7351e-07 & 5.7351e-07\\ \hline
%   1.0000e+02   & 5.7310e-06   & 5.7351e-06 & 5.7351e-06\\ \hline
%   1.0000e+03   & 5.7310e-05   & 5.7351e-05  &5.7351e-05\\ \hline
%   1.0000e+04  &  5.7310e-04   & 5.7351e-04  &5.7351e-04\\ \hline
%   1.0000e+05   & 5.7310e-03    &5.7350e-03  &5.7350e-03\\ \hline
%   1.0000e+06   & 5.7303e-02   & 5.7343e-02  &5.7343e-02\\ \hline
%   1.0000e+07   & 5.6825e-01   & 5.6865e-01  &5.6864e-01\\ \hline
%   1.0000e+08   & 5.4346e+00   & 5.4353e+00 & 5.4382e+00\\ \hline
%   1.0000e+09   & 5.3167e+01    &5.3499e+01  &5.3202e+01\\ \hline 
%   1.0000e+10   & 5.3038e+02   & 5.3483e+02 & 5.3083e+02\\ \hline
%   1.0000e+11   & 5.3036e+03    &5.3483e+03  &5.3082e+03\\ \hline
%   1.0000e+12   & 5.3036e+04   & 5.3483e+04  &5.3082e+04\\ \hline 
%\end{tabular}
%\caption{$\omega$ L as a function of frequency by computing it
%  pointwise (Zc.mat) or simulating with a state-space equivalent
%  circuit of order 7 and 20}
%\label{Ind_rom}
%\end{center}
%\end{table}

As can be seen in the figures, in the order 7 reduced-order model the
resistance matches the directly computed values (Zc.mat) up to about
$10^8$ Hz. The order 20 reduced-order model is more accurate, and
represents the frequency-dependent resistance up to $10^{9}$ Hz. The
inductance in the example chosen does not have a strong frequency
dependence.

Note that if Approach 1 were used, a single frequency would have to be
chosen to generate an equivalent circuit. This would be less accurate
but would probably reduce subsequent Spice simulation time.

\subsection{Command Line Options}
\label{clo}

This section describes using the command line options to change the
defaults settings.  All arguments are case insensitive. 

{\tt -} (dash) Forces input to be read from the standard input.  

\vspace{0.2in}
{\tt -s \{ludecomp | iterative\}} - Specifies the matrix solution
method used to solve the linear system arising from the
discretization.  {\tt iterative} uses the GMRES iterative algorithm
and {\tt ludecomp} uses LU decomposition with back substitution.  In
general, GMRES is faster, however some speed up may be obtained using
LU decomposition for problems with fewer than 1000 filaments.  {\tt
iterative} is the default.

\vspace{0.2in}
{\tt -m \{direct | multi\} } - Specifies the method to use to perform
the matrix-vector product for the iterative algorithm.  {\tt direct}
forms the full matrix and performs the product directly.  {\tt multi}
uses the multipole algorithm to approximate the matrix-vector product.
For larger problems, the multipole algorithm can save both computation time and
memory. {\tt multi} is the default.

\vspace{0.2in}
{\tt -p \{on | off | loc | posdef | cube | seg | diag | shells \}} -
Specifies the method to precondition the matrix to accelerate
iteration convergence. Two classes of preconditioners are implemented
in FastHenry.  Local inversion preconditioners ({\tt loc} and {\tt
posdef}) were used in prior releases and have been replaced by the
sparsified-L preconditioners.  Three types of sparsified-L
preconditioners are available, ({\tt cube}, {\tt seg}, and {\tt
diag}).  {\tt cube} is the default and produces the best results,
however under certain conditions can consume significantly more memory
than {\tt diag}.  If there are multiple filaments in each segment,
than {\tt seg} is a middle ground between {\tt diag} and {\tt cube},
otherwise it is identical to {\tt diag}.  {\tt shells} uses a
current shell idea described in

\begin{quote}
\noindent
Mattan Kamon, Byron Krauter, Joel Phillips, Lawrence T. Pileggi, Jacob
White, 
``Two Optimizations to Accelerated
     Method-of-Moments Algorithms for Signal Integrity Analysis of
     Complicated 3-D Packages'', {\em Proceedings of the IEEE 4th
     Topical Meeting on Electrical Performance of Electronic
     Packaging}, Portland, OR, October 1995, pp. 213-216.
\end{quote}

The method is relatively new, and its performance is sensitive to the
shell radius set by the {\tt -R} option described below.

\vspace{0.2in}
{\tt -o n} - Specifies $n$ as the order of multipole expansions.
Default is 2.  Choosing values less than 2 result in faster execution
at the expense of poorer accuracy.  Changes in the order of expansions
should accompany changes in iteration error tolerance with the {\tt -t} option.

\vspace{0.2in}
{\tt -l \{n | auto\}} - Specifies $n$ as the number of partitioning
levels for the multipole algorithm. {\tt auto} chooses the level
automatically and is the default.  

\vspace{0.2in}
{\tt -f \{off | simple | refined | both | hierarchy \}} - Switches FastHenry to
visualization mode only and specifies the type of FastCap generic file
to make (for visualization ONLY!).  {\tt off} will produce no file and
is the default.  {\tt simple} will produce a file named {\tt
zbuffile} from the segments defined in the input file.  {\tt
refined} produces a similar file, named {\tt zbuffile2}, but uses
the segments after either user refinement specified with {\tt -i} or
required refinement necessary for accuracy of the multipole algorithm
({\tt -l} option).  {\tt both} produces both files.  One FastCap
``panel'' is created for each of the four sides along the length of
the each segment.  Reference planes are handled differently as
described under the {\tt -g} option below.  See
Section~\ref{postscript} for details on producing postscript.  FastHenry will
exit after creating the visualization files. {\tt hierarchy} will dump
a 2D representation of the nonuniform reference plane discretization
hierarchy to the FIXED file {\tt hier.qui} which can also be processed
like the {\tt zbuffile}.

\vspace{0.2in}
{\tt -g \{on | off | thin | thick\}} - controls appearance of the
ground plane when using the {\tt -f} option. {\tt off} is the default
and only four panels are produced for each plane, one for each of the
edges.  Thus only the outline of each plane is drawn. This makes
generation of the postscript image much faster and also makes the planes 
transparent.  No holes are visible, however.  {\tt on} or {\tt thin}
will draw all the overlapping segments of the ground plane as if
infintely thin.  Note that using this option for a finely discretized
plane may take a long time to generate a postscript image with
{\tt zbuf}. {\tt thick} will completely draw the segments of the plane
and will be even slower for generating postscript images.

\vspace{0.2in}
{\tt -a \{on | off\}} - {\tt on} allows the multipole algorithm to
automatically refine the structure as is necessary to maintain
accuracy in the approximation.  The structure will be refined whether
or not the multipole algorithm is used.  {\tt off} prevents refinement
and will produce a warning if the multipole algorithm is used and
prevented from necessary refinement. {\tt on} is the default and is
equivalent to {\tt -l auto}. 
% {\tt off} is equivalent to using {\tt -l0}.
Note that this is not refinement
to reduce discretization error.  See the {\tt -i} option. 

\vspace{0.2in}
{\tt -i n} - Specifies $n$ as the level for initial refinement.  This
option allows the user to refine the structure if the input file is
too coarse.  It will divide each segment of the geometry into multiple
segments so that no segment has a length greater than $ \frac{1}{2^n}
$ times the length of the smallest cube which contains the whole
structure.  The default is 0 (no refinement).  This option is rarely
used since the multipole algorithm will refine as it needs and
discretization errors of this sort are usually small.

\vspace{0.2in}
{\tt -d \{on | off | mrl | mzmt | grids | meshes | pre | a | m | rl |
ls\}} \\ - dump certain internal matrices to files.  The format of some of
the files can be specified with the {\tt -k} option.  {\tt on} dumps
the $M, R, L, MZM^t$ and preconditioner matrices. {\tt off} dumps none
and is the default.  {\tt mrl} dumps
the M, R, and L matrices. {\tt mzmt} dumps the $ MZM^t$ matrix for $w
= 1$.  {\tt grids} dumps matrices for viewing the current distribution
inside each reference plane (only in matlab format).  
{\tt meshes} is a matlab file for viewing
the KVL meshes chosen by FastHenry.  {\tt pre} dumps the
preconditioner in a sparse matrix text format. {\tt ls} dumps the
sparsified $L$ matrix in sparse text format.  {\tt a} is the branch
incidence matrix, $A$. {\tt m} can be used to dump only the mesh
matrix $M$.  Similarly, only R and L can be dumped with {\tt rl}.
The right-hand-side vector, $V_s$, is dumped to the file {\tt b.mat}
whenever anything but {\tt off} is specified.

\vspace{0.2in}
{\tt -k \{matlab | text | both\}} - Specifies type of file to dump
with the {\tt -d} option.  {\tt matlab} dumps the files as M.mat, MRL.mat,
MZMt.mat, and A.mat in a format readable by matlab.  
{\tt text} saves the files M.dat, L.dat, R.dat, MZMt.dat, and A.dat as text. 
{\tt both} saves files in both formats.

\vspace{0.2in}
{\tt -t rtol, -b atol} - Specifies the tolerance for iteration error.
FastHenry calculates each column of the impedance matrix separately.  The
iterative algorithm will stop iterating when both the real and
imaginary part of each element, $x_k$, of the
current column being calculated satisfy
\begin{equation}
|x^{i - 1}_k - x^i_k| < rtol*(|x^i_k| + atol*(max_j|x_j^i|)  )
\end{equation}
where $ i $ is the iteration number.
The defaults are $ rtol = 10^{-3} $ and $ atol = 10^{-2} $.
In essence, this causes GMRES iterations to stop if every element of
solution vector has changed by less than the fraction $rtol$ since the last
iteration.  If, however, element $x_k$ is smaller than $atol$ times the
largest element in the solution vector, then the stopping criteria is
less stringent.  Note that real and imaginary parts are treated separately.

%{\tt -e \{n | auto\}} - n = maximum # of filaments in a cube to be 
%                                  considered exact. auto = (order+1)^2.

\vspace{0.2in}
{\tt -c n} - n = maximum number of iterations to perform for each
solve (column).  Overrides the default of 200.  

\vspace{0.2in}
{\tt -D \{on | off\}} - Controls the printing of debugging information.
{\tt off} is the default.  {\tt on} will cause FastHenry to print more
detailed information about the automatic partitioning level selection,
memory consumption, preconditioner calculation, and convergence of the
iterates.  The Matlab file {\tt Ycond.mat} is also produced which contains the
admittance matrices at each frequency and also the norm of the
residual at each step of the GMRES algorithm.

\vspace{0.2in}
{\tt -x portname} - Specifies that only the column in the admittance
matrix specified by {\tt portname} should be computed.  Multiple {\tt
-x} specifications can be used.  In this case, the output file {\tt
Zc.mat} will contain the requested columns of the admittance matrix
instead of the impedance matrix.  The portname is specified in the
input file with the {\tt .External} keyword.  The primary uses of this
option are either to exploit symmetry or to compute single columns for
observing current distribution.  For instance, if a 10 pin package is
symmetric about some axis, then only five columns need be computed.
The user is then responsible for forming the $10\times 10$ impedance
matrix from the $10 \times 5$ result.  Also, if the {\tt -d grids}
option is used, the port which is the source of current would be
specified with this option.

\vspace{0.2in}
{\tt -S suffix} - This adds the string {\tt suffix} to all filenames
for this run.  For instance, {\tt -S \_blah} will produce the output file
{\tt Zc\_blah.mat}.

\vspace{0.2in}
{\tt -r order} - Specifies a reduced order model of the system as
output.  The size of the model will be {\tt order}*number-of-ports. A
matlab file {\tt rom.m} will contain the A,B,C,D matrices representing
the state-space reduced order model. More useful is the file {\tt
  equiv\_circuitROM.spice} which is a spice3 compatible subcircuit
which can be included for circuit simulation.

\vspace{0.2in}
{\tt -M} - If {\tt -r} is specified with a nonzero order, then this
option will cause FastHenry to exit after generating a reduced order
model.

\vspace{0.2in}
{\tt -R radius} - If {\tt -p shells} is specified, then this specifies the
radius of the shells to use.

\vspace{0.2in}
{\tt -v} - Regurgitate internal representation of the geometry to
stdout in the input file format.  
Good for debugging. User can compare output to the original input
file.  Also, by altering the file {\tt regurgitate.c}, the user can
also call functions to translate and reflect the geometry for use as
another input file.

\subsection{Discretization Error Analysis}
\label{discret}

%A common question is how many filaments per segment should be used.
This section gives some guidelines and describes an experimental
approach to determining an appropriate discretization to model skin
and proximity effects in long, thin condutors defined with
FastHenry segments.  (For a description of discretization for reference
planes, see ``nonuniform\_manual.ps''.)

To begin, assume it is desired to compute the impedance matrix for the pin
connect structure of Figure~\ref{pin-connect} for frequencies up to
$10^8$ Hz. 

\subsubsection{The DC case}
\label{DC}

If only the DC case were of desired, it could be computed rapidly
since there is no skin effect.  The number of filaments per segment
for the entire structure is set to one by setting {\tt nhinc=1} and
{\tt nwinc=1} with the line
\begin{verbatim}
.DEFAULT Z=85. H=8.5 W=24. nhinc=1 nwinc=1
\end{verbatim}

In order to extract a nonzero value for the inductance, the frequency
is set small, but nonzero
\begin{verbatim}
.freq fmin=1e0 fmax=1e0 ndec=1
\end{verbatim}

Since there are only a few hundred filaments, there is no advantage to
the multipole algorithm so LU decomposition can be used to save a few
seconds:
\begin{verbatim}
fasthenry pin-connect.inp -sludecomp -S _DC
\end{verbatim}
The output will be in file {\tt Zc\_DC.mat}

\subsubsection{The highest frequency case}

Next, consider deciding how many filaments are needed to accurately
compute the resistance and inductance at the highest frequency point
of interest, $f = 10^8$ Hz.  A good rule of thumb is to choose the
discretization such that the width of the smallest filament is roughly
equal to the skin depth.  For this geometry,  $\rho = 0.0238$~ohm-mil
or $\sigma = 1.641 \times 10^6$~mho/m 
giving a skin depth of
\begin{equation}
 \delta = \sqrt{\frac{1}{\pi f \mu \sigma}} = 1.55\; {\rm mils}.
\end{equation}
To observe the error involved with various discretizations, we will
look at the impedance matrix for two typical adjacent segments from the {\tt
pin-connect.inp} example:

\vspace{0.3in}
\listing{/u0/matt/induct/examples/pin_expers/3pin/pin-con2seg.inp}

The result will be a $2 \times 2$ complex matrix for each frequency point,
\begin{equation}
{\bf Z}(\omega) = \left[ 
\begin{array}{cc}
  R_{11}(\omega) + j\omega L_{11}(\omega) &  R_{12}(\omega) + j\omega L_{12}(\omega)  \\
  R_{21}(\omega) + j\omega L_{21}(\omega) &  R_{22} (\omega) + j\omega L_{22}(\omega) 
\end{array}
\right]
\end{equation}

If we choose $nhinc = 3$ and $nwinc = 5$ with a default ratio of
adjacent filaments equal to 2 ($rh = rw = 2$), then the width of the
smallest filament on the right and left of the segment will be 1/10
the width ($(1 + 2 + 4 + 2 + 1)^{-1}$), or $2.4$~mils.  The thickness
of the filaments on the top and bottom of the segment will be 1/4 of
the height ($(1 + 2 + 1)^{-1}$) or $2.1$ mils.  

To run fasthenry, 
\begin{verbatim}
fasthenry pin-con2seg.inp -sludecomp -aoff -S _3x5_2
\end{verbatim}
In this case, since there are so few filaments, LU decomposition is
slightly faster.  Also, automatic refinement is prevented with the
{\tt -aoff} option since the multipole algorithm is not being used
and this problem is essentially two-dimensional.

The smallest filaments are slightly above the skin depth and thus the
results are inaccurate at $f = 10^8$ as shown by the dotted lines in
Figure~\ref{fig:twoseg}.  The next logical step might be to increase
the discretization to $nhinc \times nwinc = 4 \times 6$ however using
even numbers of filaments is not efficient since the center of the
segment, where current is relatively constant, will be divided into 4
separate filaments.  For the odd case, the center is modeled by only
one filament.  Instead, consider changing the ratio of adjacent
segments to $rh = 4$, $rw = 3$.  Now, the width of the smallest
filament on the left and right sides of the segment will be 1/17 of
the total width ($(1 + 3 + 9 + 3 + 1)^{-1}$) or $1.41$~mils and the
thickness of the smallest filament on the top and bottom, $1.42$~mils.
This is much closer to the skin depth and gives better results as
shown by the dash-dot lines in Figure~\ref{fig:twoseg}.

Another possibility is to reduce the number of filaments below $3
\times 5$ but further increase the ratio of adjacent filaments.  This
change would result in faster execution since there would be fewer
filaments, however some accuracy at midrange frequencies would be
sacrificed.  For instance, consider maintaining the width of the
smallest filaments by changing the discretization above to $3 \times
3$ with $rh = 4$, $rw = 15$.  In this case there are too few filaments
to model the decay of current density from the outside edge to center
for $f \in [10^6,10^8]$ as shown by the dashed line 
in Figure~\ref{fig:twoseg}.

\begin{figure}
\begin{minipage}{0.49\textwidth}
\centerline{\psfig{figure=/u0/matt/induct/examples/pin_expers/3pin/R11.ps,width=3.0in}}
\centerline{\psfig{figure=/u0/matt/induct/examples/pin_expers/3pin/R12.ps,width=3.0in}}
\centerline{\psfig{figure=/u0/matt/induct/examples/pin_expers/3pin/L11.ps,width=3.0in}}
\centerline{\psfig{figure=/u0/matt/induct/examples/pin_expers/3pin/L12.ps,width=3.0in}}
\end{minipage}
\hfill
\begin{minipage}{0.49\textwidth}
\centerline{\psfig{figure=/u0/matt/induct/examples/pin_expers/3pin/R11e.ps,width=3.0in}}
\centerline{\psfig{figure=/u0/matt/induct/examples/pin_expers/3pin/R12e.ps,width=3.0in}}
\centerline{\psfig{figure=/u0/matt/induct/examples/pin_expers/3pin/L11e.ps,width=3.0in}}
\centerline{\psfig{figure=/u0/matt/induct/examples/pin_expers/3pin/L12e.ps,width=3.0in}}
\end{minipage}
\caption{Values and errors for the impedance matrix as a function of frequency 
for the 2 segment experiment}
\label{fig:twoseg}
\end{figure}

After the user has decided which error for the above cases is tolerable
for a given problem,
the discretization can then be used for the full {\tt pin-connect.inp}
example.  The default $nhinc$ and $nwinc$ must be set with
\begin{verbatim}
.DEFAULT Z=85. H=8.5 W=24. nhinc=3 nwinc=5 rh=4 rw=3
\end{verbatim}
and the frequencies of interest with
\begin{verbatim}
.freq fmin=1e0 fmax=1e12 ndec=3
\end{verbatim}

The frequency setting above was used to generate
Figure~\ref{fig:twoseg}.  Note that with the sparsified-L preconditioner,
used by default, the low frequency impedance matrix is
computed in very few iterations.  Thus, while the low frequency case
could have been computed with a coarser discretization as in
Section~\ref{DC}, there is not much added execution time in computing it 
at a finer discretization.

The exact values in Figure~\ref{fig:twoseg} were generated using a
$21 \times 25$ discretization with $rh = rw = 2$.  It was run with
\begin{verbatim}
fasthenry pin-con2seg.inp -aoff -mdirect
\end{verbatim}
%fasthenry pin-con2seg.inp -l0 -mdirect -pdiag -S _21x25_2
The multipole algorithm would require space be divided, and thus the
filaments also be divided, tripling the number of filaments in this
case.  Since the original number of filaments is small and, again,
this is essentially a two-dimensional problem, we can avoid both
dividing space ({\tt -aoff}) and using the multipole algorithm ({\tt
-mdirect}). The problem is small enough to fit in memory with all the
matrices formed directly however is large enough to warrant using the
iterative algorithm instead of LU decomposition.
%Since space is not divided, the cube block preconditioner would invert
%the entire problem and thus the diagonal preconditioner must be
%pecified with the {\tt -pdiag}.  

For the $21 \times 25$ problem, the time spent forming the
preconditioner is significant.  The {\tt -pdiag} option could have
been used to reduce the preconditioner time, however this would come
at the expense of slower convergence especially at the higher
frequencies.

\section{Geometry Postscript Pictures}
\label{postscript}

The ability to see the geometry under analysis is an important tool for
debugging input files.  This section describes how to generate
postscript files for visualizing the three dimensional structures
defined by a given input file.  The process involves first running
FastHenry to generate a file of ``panels'' and then running {\tt zbuf}
to generate the postscript file.

\subsection{Creating the panel file}

When FastHenry is given the {\tt -f} option, it changes from
calculation to visualization mode.  It will produce a file of
panels readable by the {\tt zbuf} program.  Each panel is a
quadrilateral representing one of the faces of a segment in the input
geometry.  This file of panels can either be formed from the segments
of the geometry before FastHenry's refinement 
with the {\tt simple} argument, or after refinement with the
{\tt refined} argument.  The basenames for the
files produced are {\tt zbuffile} for simple geometry and 
{\tt zbuffile2} for refined unless the {\tt -S} is used.  

In addition, the file {\tt zbuffile\_shadings} 
%(if no suffix is specified) 
or {\tt zbuffile2\_shadings} is produced which specifies
a shade of gray to assign to each of the panels.  {\tt
zbuffile\_shadings} is produced for {\tt -f simple} and shades each
of the reference planes (if drawn with {\tt -g on}) of the structure
differently, and leaves all other structures white.  {\tt
zbuffile2\_shadings}, produced with {\tt -f refined}, does the same,
however it attempts to shade differently each of the conductors specified
with a {\tt .external} statement.

\subsection{Creating the postscript file}

The panel file described in the previous section can be rendered in
postsciprt with the {\tt zbuf} program.  This program is a modified
version of the algorithms used for visualization in the capacitance
extraction program, FastCap, developed by Keith Nabors.

\begin{minipage}{\textwidth}
The complete command line format of the {\tt zbuf} program is
\begin{verbatim}
zbuf [-a<azimuth>] [-e<elevation>] [<input-file>]
        [-r<rotation>] [-h<distance>] [-s<scale>] [-w<linewidth>]
        [-u<upaxis>] [-q] [-x<axeslength>]
        [-b<.figfile>] [-c] [-v] [-n] [-f] [-g] [-m]
\end{verbatim}
\end{minipage}

Table~\ref{quiref} itemizes the options and Table~\ref{dashmuse} gives
several examples of their use.

Usually the default settings
produce an acceptable plot, but many adjustments are possible,
the most important being the  view angles (relative to a coordinate
system parallel to the input coordinates and centered on the 
center of the object) which are adjusted using the
{\tt -a} and {\tt -e} options.  Other options control view distance
({\tt -h}), two-dimensional projection scale ({\tt -s}), rotation
({\tt -r} and {\tt -u}) and line width ({\tt -w}). Axes of any length
may be included  with the {\tt -x} option and lines, arrows and dots may be
added using {\tt -b}.  Shading can be accomplished with the {\tt -q} option.
Table~\ref{dashmuse} gives the commands used to produce the line
drawings in this guide as examples. 

In version 3.0, the {\tt zbuf} program now takes the ``-m'' argument
to produce a 
Matlab file for faster visualization in matlab.  This is very beneficial for
large files since producing the postscript file with {\tt zbuf} can
take $n^2$ time. 
The matlab file can be viewed within matlab with the
fasthenry-3.0/bin/plotfastH.m matlab function.  The file {\tt
  zbuffile.mat} would be produced with ``zbuf -m zbuffile'' which can
then be viewed in matlab with ``plotfastH('zbuffile.mat')''.
Also, you can modify the file {\tt src/zbuf/dump\_struct.c} to output
in YOUR own 
format instead of matlab.

\subsection{Tricks}
For Figure~\ref{holey}, the shading of the trace and ground plane were
swapped by negating all the shading values with the {\tt awk} file
{\tt invert.awk} provided in the {\tt zbuf} source directory.  Also,
Figure~\ref{one_of_30pin}, since it is only one pin, would normally
appear white and all panels would have shade $0$ in the shading file.
By randomly changing one panel to $-5$ and another to $10$, then $-5$
becomes white, $10$ becomes black, and $0$, a shade of gray.
The single black panel is seen on the left end of the figure and the
white panel is obscured by other panels.

\begin{table}
\begin{center}
\begin{tabular}{cl}\hline
\multicolumn{1}{c}{Figure} & \multicolumn{1}{c}{{\tt FastHenry/Zbuf} Usage} \\ \hline
\protect\ref{gpexamp_copper} & \tt fasthenry gpexamp\_copper.inp -f simple -S \_pcb \\
& {\tt zbuf zbuffile\_pcb} \\
\hline
\protect\ref{vias} & {\tt fasthenry vias.inp -f refined -g on} \\
& {\tt zbuf -a60 -e60 zbuffile2 -q} \\
\hline
\protect\ref{holey} & {\tt fasthenry holey\_gp.inp -f simple -g on} \\
& {\tt mv zbuffile holey\_gp} \\
& {\tt awk -f src/zbuf/invert.awk zbuffile\_shadings > holey\_gp\_shadings} \\
& {\tt zbuf holey\_gp -e0 -q -uy} \\
\hline
\protect\ref{pin-connect} & {\tt fasthenry pin-connect.inp -f simple -S pin} \\
& {\tt zbuf zbuffilepin -a25} \\
\hline
\protect\ref{30pin} & {\tt fasthenry 30pin -f refined} \\
& {\tt zbuf zbuffile -a30 -q} \\
\hline
\protect\ref{one_of_30pin} & {\tt fasthenry one\_of\_30pin -f simple} \\
& {\tt zbuf zbuffile -q -e20 -a0} \\
\hline
\end{tabular}
\caption{Commands used to generate some representative figures in this guide.}
\label{dashmuse}
\end{center}
\end{table}



\begin{table}
\begin{center}
\begin{tabular}{ccll}
\multicolumn{4}{c}{\mbox{\tt zbuf} Options}\\\hline
\multicolumn{1}{c}{Option}&\multicolumn{1}{c}{Default}&
\multicolumn{1}{c}{Range}&
\multicolumn{1}{c}{Function}\\\hline
\mbox{\tt -a}& \mbox{\tt 50.0} & \dag
&Specifies azimuth view angle in degrees.\\
\mbox{\tt -e}& \mbox{\tt 50.0} & \dag
&Specifies elevation view angle in degrees.\\
\mbox{\tt -r}& \mbox{\tt 0.0} & \dag
&Specifies final rotation of 2-D image in degrees.\\
\mbox{\tt -h}& \mbox{\tt 2.0} & $\geq$ \mbox{\tt 0.0}
&Specifies distance from surface of object in object radii.\\
\mbox{\tt -s}& \mbox{\tt 1.0} & $>$ \mbox{\tt 0.0}
&Specifies final scaling of 2-D image.\\
\mbox{\tt -w}& \mbox{\tt 1.0} & $\geq$ \mbox{\tt 0.0}
&Specifies postscript file line width.\\
\mbox{\tt -u}& \mbox{\tt z} & \mbox{\tt x}, \mbox{\tt y}, or \mbox{\tt z}
&Specifies which 3-D axis is mapped to y-axis in 2-D image.\\
\mbox{\tt -q}& --- & ---
&\parbox{4in}{Uses zbuffile\_shading shadings file to give grayscale shading to conductors.}\\
\mbox{\tt -x}& \mbox{\tt 1.0} & $>$ \mbox{\tt 0.0}
&Includes axes of length \mbox{\tt axeslength} in picture.\\
\mbox{\tt -b}& $\diamond $ & $\diamond $
&Specifies lines, dots and arrows to superimpose on picture.\\
\mbox{\tt -c}& --- & ---
&Puts the command line in postscript file pictures.\\
\mbox{\tt -v}& --- & ---
&Removes {\tt showpage} from postscript file.\\
\mbox{\tt -n}& --- & ---
&Numbers faces in order input.\\
\mbox{\tt -f}& --- & ---
&Suppresses hidden line removal.\\
\mbox{\tt -g}& --- & ---
&Prints the graph used to order the panels in postscript file.\\
\mbox{\tt -m}& --- & ---
&Produces a matlab file rather than a postscript file\\
\hline
\end{tabular}
\caption{Zbuf Options}
\label{quiref}
\end{center}
\noindent\dag Range is unrestricted. \\
\noindent$\diamond $ See function {\tt readLines()} in {\tt src/zbuf/zbufInOut.c}
for a description of the {\tt .fig} file format.
\end{table}

\subsection{Visualization artifacts}

This section discribes possible problems with producing pictures.
Many of the listed items come directly from the FastCap documentation.

\begin{enumerate}
\item To minimize the number of panels that {\tt zbuf} must deal with,
FastHenry does not output panels for the end faces of a segment.
Thus, segments often appear hollow.  Uncomment the appropriate
regions in {\tt writefastcap.c} to change this. 
\item {\tt zbuf} sometimes produces panel ordering errors (and
usually print warning messages) when panels intersect. However, often
the postscript file is correct.
\item Occasionally the postscript pictures are incorrect even
for legal discretizations. There are two known bugs leading
to this problem.  The first is often avoided by changing
the view angle slightly. The second is caused by problems with
dimensions on the order of $10^{-4}$ or less. This problem can
only be avoided by rescaling the input by changing the {\tt .Units}
to {\tt km}.
\item The panel sorting algorithm used to write out the postscript files
takes time proportional to the number of panels squared.  This limits
its usefulness to problems with fewer than a few thousand panels. 
%For
%larger problems the paneling should be done hierarchically so that
%a coarser discretization can be used for pictures. For problems built
%using the generic input file generators, the {\tt -d} option provides
%this function.
\item The shading algorithm uses one shade of gray to shade 
all the segments along only one path between the two nodes of a {\tt
.external} statement. Thus, if there are multiple paths, such as in
Figure~\ref{30pin}, the other paths will appear white.
\item When axes are included in postscript files using the
{\tt -x<axes length>} option, the axes' two-dimensional projections
appear in the postscript file before the panel fills. This means
that the object should be between the view point and the axes, otherwise
the axes can be obscured strangely. 
Thus
the option works best when viewing objects that lie entirely in
the positive orthant from a view point in the positive orthant.

\end{enumerate}


\section{Reference Plane Current Visualization}
\label{viewcur}

When FastHenry is given the {\tt -d grids} option, it dumps Matlab
readable files of the current distribution in each reference plane of
the geometry under analysis.  One file is dumped for each of the
columns computed for the admittance matrix.  The file holds the
current distribution produced by setting the source across the port
corresponding to the current column to one volt and all others ports
to zero.  This process is then repeated for each freqeuncy point.

{\bf Note:} For nonuniformly discretized planes, the filename is the
same, but the file format is very different. See {\tt
  nonuniform\_manual.ps}. This document also describes viewing the current
distribution in other than reference planes.

\subsection{Current Files}
The files are named according to the column in {\tt Zc.mat} and frequency
point.  They have the form Grid$n$\_$m$.mat where $n$ is the column
(and row) in {\tt Zc.mat} of the port which has the one volt source
and $m$ is the frequency point number.  The frequency points start at
$0$ and increment by one for each new frequency.

Inside each file are matrices for each reference plane named {\tt
grid1g<name>} and \\ 
{\tt grid2g<name>} where {\em name} is the name of
the reference plane defined in the input file.  Each entry in {\tt
grid1g<name>} is the complex valued current phasor for a filament in
the {\tt seg1} direction.  {\tt grid2g<name>} corresponds to filament
currents in the {\tt seg2} direction.  In general it is difficult to
visualize complex current, but when the current is purely real (DC) or
purely imaginary (high frequency), these two matrices can be thought
of as the current in two orthogonal directions.

\begin{figure}
\centerline{
\psfig{figure=together.ps,width=4.0in}
}
\caption{Two traces over a ground plane}
\label{together}
\end{figure}

\subsection{Examples}
        
\subsubsection{Traces over a solid plane}
Consider the example file {\tt together.inp} shown in
Figure~\ref{together}.  This structure is simply two traces passing
over a ground plane with one end of each trace shorted to the plane.
Thus, when one volt is applied to the other end of one of the traces,
current travels down the trace and returns through the plane.  To
generate the current distribution files:
\begin{verbatim}
fasthenry together.inp -d grids -x trace1
\end{verbatim}
The files {\tt Grid1\_0.mat} and {\tt Grid1\_1.mat} are produced in
addition to FastHenry's normal output files. By
default, FastHenry would compute the current distribution for the
source voltage placed one trace and then again for the other trace.
Since the current distribution is nearly identical for both cases, we
need only compute one of these cases and thus the {\tt -x} option is
used to specify that only the column of the admittance matrix
corresponding to port {\tt trace1} is to be computed (see input file).
Also, since visualization is the goal, {\tt together.inp} specifies
that only a low frequency case, $f = 10^{-1} Hz$, and a high frequency
case, $f = 10^{19} Hz$ are to be computed. In this example, the
discretization of the plane is coarse for visualization purposes.  

\begin{figure}
\centerline{
\psfig{figure=together_DC.ps,width=4.0in}
}
\caption{}
\label{together_DC}
\end{figure}

\begin{figure}
\centerline{
\psfig{figure=together_highf.ps,width=4.0in}
}
\caption{}
\label{together_highf}
\end{figure}

The current distribution can then be viewed using the following
Matlab commands to produce Figures~\ref{together_DC} and
\ref{together_highf}.
\begin{verbatim}

                            < M A T L A B (R) >
                (c) Copyright 1984-93 The MathWorks, Inc.
                            All Rights Reserved
                                Version 4.1
                                Jun 15 1993

Commands to get started: intro, demo, help help
Commands for more information: help, whatsnew, info, subscribe

>> load Grid1_0.mat
>> who

Your variables are:

grid1g1   grid2g1

>> quiver(real(grid1g1),real(grid2g1),2)
>> title('DC current distribution for together.inp')
>> print together_DC.ps
>> load Grid1_1.mat
>> who

Your variables are:

grid1g1   grid2g1

>> quiver(imag(grid1g1),imag(grid2g1),2)
>> title('High Frequency current distribution for together.inp')
>> print together_highf.ps

\end{verbatim}

Note that for the DC case in Figure~\ref{together_DC} the current
spreads across the plane as it travels from its source to sink points
but at high frequency the current is focused underneath the trace.
Note also that the high frequency current travels in the 
opposite direction as the current at DC since $I = V/(j w L) = - j V/(w L)$.

\subsubsection{Traces over a divided plane}
Next consider the example file {\tt broken.inp} shown in
Figure~\ref{broken}.  This is identical to {\tt together.inp} except
that the plane is now broken in two pieces and connected by copper
``tethers'' as shown.  

\begin{figure}
\centerline{
\psfig{figure=broken.ps,width=4.0in}
}
\caption{Two traces over a divided ground plane}
\label{broken}
\end{figure}

\begin{figure}
\centerline{
\psfig{figure=broken_DC.ps,width=4.0in}
}
\caption{}
\label{broken_DC}
\end{figure}

\begin{figure}
\centerline{
\psfig{figure=broken_highf.ps,width=4.0in}
}
\caption{}
\label{broken_highf}
\end{figure}

After running FastHenry with
\begin{verbatim}
fasthenry broken.inp -d grids -x trace1
\end{verbatim}
the two matrices can be concatenated in matlab to produce
Figures~\ref{broken_DC} and \ref{broken_highf} with the following
commands:
\begin{verbatim}
>> load Grid1_0
>> who

Your variables are:

grid1g_one     grid2g_one
grid1g_two     grid2g_two

>> size(grid1g_one)

ans =

    11    11

>> space = zeros(11,3);
>> new1 = [-grid1g_two(:,11:-1:1) space grid1g_one];
>> new2 = [grid2g_two(:,11:-1:1) space grid2g_one];
>> quiver(real(new1),real(new2),2);

(similarly for Grid1_1.mat)
\end{verbatim}

Note that some difficulty is involved in combining the matrices
because the {\tt seg1} direction in space is different for the two
planes since the corners of the plane are defined in counter-clockwise
order for one plane, and clockwise for the other.

From Figure~\ref{broken_DC} it may seem unusual that current can point
into the empty space between the planes.  This, however, is only an
artifact of the method of visualization
since the current vectors do not precisely represent the
current flow at a particular point.  Since the ground plane is
discretized as a grid of segments (see Figure~\ref{gp}), one could
attempt to represent the current as that passing through each node.
But each node on the plane has four segments attached to it while only
two are needed to define a vector direction.  FastHenry consistently
picks the segments with the smaller indices.  This choice can be altered in
the function {\tt makegrids()} in {\tt fillM.c} if the user
wishes to average the currents, for instance.

\begin{figure}
\centerline{
\psfig{figure=holey_gp_highf.ps,width=4.0in}
}
\caption{}
\label{holey_gp_highf}
\end{figure}

\subsubsection{Trace over a plane with holes}
As a final, more interesting example, consider observing the current for
example file {\tt holey\_gp.inp} shown previously in
Figure~\ref{holey}.  To produce Figure~\ref{holey_gp_highf},
\begin{verbatim}
fasthenry holey_gp.inp -d grids
\end{verbatim}
and then in Matlab,
\begin{verbatim}
>> load Grid1_1
>> quiver(imag(grid1g1),imag(grid2g1),3)
\end{verbatim}

Again in this high frequency case, the current is observed to try its
best to stay underneath the trace, outlining the holes it must circumvent.


\appendix

%
% Description of output compile flag effects and compile procedure
%
\section{Compiling FastHenry}
\label{comfas}

%\noindent{\sc Check particulars w/Miguel.}

A tar file containing the source files for {\tt fasthenry} and this
guide may be obtained on tape by sending a written request to
\begin{quote}
Prof.\ Jacob White\\
Massachusetts Institute of Technology\\
Department of Electrical Engineering and Computer Science\\
Room 36-880\\
Cambridge, MA 02139 U.S.A.
\end{quote}
This address may also be used for  general correspondence regarding 
{\tt fasthenry}, although electronic mail may be sent to
{\tt fasthenry-bug@rle-vlsi.mit.edu}, 
for bug reports, and to {\tt fasthenry@rle-vlsi.mit.edu}, for
questions or comments, if it is more convenient.  Comments on
FastHenry are encouraged as well as comments on unclear portions of
this manual.

The tar file has the form
\begin{quote}
\begin{verbatim}
fasthenry-3.0-9Oct96.tar.Z
\end{verbatim}
\end{quote}
and yields a one level directory when untarred with the commands
\begin{quote}
\begin{verbatim}
uncompress fasthenry-3.0-9Oct96.tar.Z
tar xvf fasthenry-3.0-9Oct96.tar
\end{verbatim}
\end{quote}
It will create a directory tree underneath {\tt fasthenry-3.0} which 
contains all the C source files underneath {\tt src/}, this manual in
{\tt doc/}, and the example files in {\tt examples/}.

The tar files may
also be obtained via anonymous ftp to {\tt rle-vlsi.mit.edu}.  Use
username {\tt anonymous} with your email address as the password.  The
FastHenry tar files are contained in directory {\tt pub/fasthenry}.
This directory also contains various publications on multiplole
accelerated inductance extraction.  The most complete description is 
the compressed postscript file {\tt ms\_thesis.ps.Z}:

\begin{quote}
\noindent
Mattan Kamon, {\em Efficient Techniques for Inductance Extraction of
Complex 3-D Geometries}, M.S. thesis, Massachusetts Institute of
Technology, Cambridge, MA., February 1994.
\end{quote}

\noindent {\tt mtt.ps.Z} is a preprint of 

\begin{quote}
\noindent
M. Kamon, M.J. Tsuk, and J. White, ``FASTHENRY: A
Multipole-Accelerated 3-D Inductance Extraction Program'', {\em IEEE
Transactions on Microwave Theory and Techniques}, Vol.~42, 
September 1994.
\end{quote}

\noindent See the {\tt README} file at the ftp site for update information on
releases, manuals, publications.

\subsection{Compilation Procedure}

FastHenry is compiled by changing to the {\tt fasthenry-3.0} directory, 
and typing
\begin{quote}
\begin{verbatim}
make all
\end{verbatim}
\end{quote}
to create the executables {\tt fasthenry} and {\tt zbuf} in the {\tt
bin/} directory.
%This will use the file {\tt Makefile} to make {\tt fasthenry}. 

If you are compiling on a DEC 5000, DEC 3000 (Alpha), SGI, or a system
running System V (HP and Suns running Solaris, perhaps), 
other flags are required for compilation.  In this
case, before compilation as instructed above, give the command:
\begin{quote}
\begin{verbatim}
config <name>
\end{verbatim}
\end{quote}
where {\tt <name>} is one of {\tt dec}, {\tt alpha}, or {\tt sgi}, or
{\tt solaris} for
compilation on a DEC 5000, DEC 3000 (Alpha), and Silicon Graphics
workstation, respectively.  Additional steps for compiling on an SGI
are given in the file {\tt README.sgi} in the {\tt fasthenry-3.0}
directory. {\tt config sysV} is also provided for
compilation on HP-UX, and other machines running System V,
but has not been thoroughly tested.  For more details on compilation,
see the {\tt fasthenry-3.0/README} file.

\subsection{Producing this Guide}

This guide is available in postscript as three separate files: {\tt
manual\_001.ps, manual\_002.ps}, and {\tt manual\_003.ps} in
the {\tt doc/} directory.  {\tt manual\_001.ps} contains up through
page 13, {\tt manual\_002.ps} is pages 14--31, and {\tt manual\_003.ps}
is pages 32--36.  Note that these files contain many detailed
postscript images and may take significant time to print.  The
\LaTeX\ version of the manual without the figures is also available.

Also, the nonuniform plane discretization manual is in the files {\tt
  nonuniform\_manual\_1.ps} and {\tt nonuniform\_manual\_2.ps }


%\appendix 

\section{Changes in Version 3.0}
\label{changes}

\begin{itemize}

\item Specify a nonuniform discretization of a reference plane to
  capture small features in fewer elements.

\item
Two approaches for generating spice equivalent circuits are available:

1. An equivalent circuit for a single frequency.

2. A circuit which models the frequency
  dependent resistances and   inductances through a reduced state-space
  representation

 Comments:  Method 1 will not model frequency dependent resistance
 and inductance since it gives and R and L at the single specified
 frequency.  Method 2 will model the full effects up to some frequency.

\item Major Bug fix:
  For reference planes which form segments with different
  widths in the x-direction versus the y-direction, the sizes weren't
  computed correctly in version 2.0 and later.  This has been fixed.

\item
The {\tt zbuf} program now takes the ``-m'' argument to produce a
Matlab file for faster visualization in matlab.  This is very beneficial for
large files since producing the postscript file can take $n^2$ time.
The matlab file can be viewed within matlab with the
fasthenry-3.0/bin/plotfastH.m matlab function.  The file {\tt
  zbuffile.mat} would be produced with ``zbuf -m zbuffile'' which can
then be viewed in matlab with ``{\tt >> plotfastH('zbuffile.mat')}''.
Also, you can modify the file \\
{\tt src/zbuf/dump\_struct.c} to output in YOUR own
format instead of matlab.
\item Sparse preconditioner.  Specify {\tt -p shells} to use a
  preconditioner based on current shell sparsification.
\item Regurgitate the input file with {\tt -v} to see what FastHenry
  thinks it has read.  Also can translate and reflect geometry before output.
\end{itemize}

\end{document}


