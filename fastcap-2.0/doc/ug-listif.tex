%
% description of the list file interface
%
\subsection{Using a List File to Specify Complicated Geometries}
\label{listif}

Both generic and neutral format files may be combined using an
auxiliary file called a list file.  The list file lists the file names
that are to be included in the composite problem and allows for
different permittivities to be associated with either side of
the surfaces in each of the files.  Thus the list file interface
provides both a means of entering geometries with multiple dielectrics
and, in general,
a way to build complicated geometries out of simpler structures.

For example, the two-surface geometry of
the coated sphere in Figure~\ref{cut_sph} can be specified using
%
% cut away of coated sphere
%
% for fig2ug.awk
%%%&^ cut_sph.ps
\begin{figure}
\centerline{
\psfig{figure=\figuredir/cut_sph.ps,width=3.0in}
}
\caption{The discretization used to compute the capacitance of a dielectric-coated sphere in free space.  Some of the outer dielectric-boundary panels
have been removed to show the inner conductor surface
panels.}
\label{cut_sph}
\end{figure}
the list file \verb~coated_sph.lst~,
\begin{verbatim}
*
* fastcap surface list file for two concentric spherical shells
* - inner is a conductor surface
* - outer is a dielectric/dielectric (coating/air) interface
*
* radius 1 shell is the conductor
C sphere2.neu 2.0  0.0 0.0 0.0
*
* radius 2 shell is the dielectric interface - air on the outside
D big_sphere1.neu 1.0 2.0  0.0 0.0 0.0  0.0 0.0 0.0 -
\end{verbatim}
The first non-comment line specifies the inner, radius-one conductor,
surrounded by a dielectric coating with relative permittivity {\tt 2.0}.
The last line defines the outer surface of the coating, a dielectric
interface.  

In general a list file has lines of the form given below. 
{\small
\begin{verbatim}
* <comment>
G <group name>
C <file> <outperm> <xtran> <ytran> <ztran> [+]
D <file> <outperm> <inperm> <xtran> <ytran> <ztran> <xref> <yref> <zref> [-]
B <file> <outperm> <inperm> <xtran> <ytran> <ztran> <xref> <yref> <zref> [-][+]
\end{verbatim}
}
\noindent The {\tt C} line is used to include the panels in {\tt file}
as parts of conductor surfaces surrounded by
dielectric with relative permittivity {\tt outperm}. The entire contents
of {\tt file} may be translated by specifying non-zero values for
{\tt <xtran> <ytran> <ztran>}. The {\tt D} and {\tt B} lines specify
dielectric interfaces along two regions with permittivities {\tt outperm}
and {\tt inperm}. The {\tt B} (or ``both'') specifies that the dielectric
interface is coincident with an infinitesimally thin conductor, while
{\tt D} surfaces are dielectric interfaces alone. In both cases, translations
are possible as for {\tt C} surfaces. In all cases {\tt file} can be
in either PATRAN-generated neutral or generic format.

The reference point
{\tt <xref> <yref> <zref>} and optional ``{\tt -}'' 
for {\tt D} and {\tt B} surfaces are used to specify
which side of the interface has which permittivity. In particular,
the reference point {\tt <xref> <yref> <zref>} is assumed to lie
on the {\tt outperm} side of all the panels in {\tt file}. The 
optional ``{\tt -}'' indicates that {\tt <xref> <yref> <zref>} instead
lies on the {\tt inperm} side. It is the user's responsibility to make
sure the reference point is on the same side of the dielectric interface
for all panels. By partitioning surfaces into convex subsurfaces, a
complete specification of any dielectric interface can always be made
using multiple files, each with its own reference point.

The optional ``{\tt +}'' argument is used to control conductor naming.
When a {\tt C} or {\tt B} line is terminated with a ``{\tt +},'' any
panels with the same conductor name in that line's {\tt file} and the
next {\tt C} or {\tt B} line's {\tt file} are considered to be part of
the same conductor.  For example, consider the first few lines
from the example list file
{\tt 1x1bus.lst},
\begin{verbatim}
*
* 1x1 bus crossing problem with dielectric on lower conductors
*
* conductor to air interfaces
C cond_air_1x1.qui 1.0 0.0 0.0 0.0 +
*
* conductor to dielectric interfaces
C cond_dielec_1x1.qui 7.5 0.0 0.0 0.0
\end{verbatim}
The two files \verb~cond_air_1x1.qui~ and \verb~cond_dielec_1x1.qui~ represent
the parts of the conductors that border the air and passivation
dielectrics respectively (see Figure~\ref{1x1bus}). 
%
% 1x1 bus showing breakdown into conductor/air, conductor/dielectric
%   and dielectric/dielectric
%
\begin{figure}
\centerline{
\psfig{figure=\figuredir/1x1bus_ca.ps,width=1.9in}\rule{0.1in}{0in}\psfig{figure=\figuredir/1x1bus_cd.ps,width=1.9in}\rule{0.1in}{0in}\psfig{figure=\figuredir/1x1bus_da.ps,width=1.9in}\rule{0.1in}{0in}
}
\caption{The parts of the 1x1 coated bus crossing problem: 
conductor-air interfaces, 
left (from {\tt cond\_air\_1x1.qui}), conductor-passivation
layer interfaces, center (from {\tt cond\_dielec\_1x1.qui}) and
passivation layer-air interfaces, right (from remaining files in 
{\tt 1x1bus.lst}). Axes are three units long.}
\label{1x1bus}
\end{figure}
%
% these things are here so awking with fig2ug.awk (after tex2ptug.awk)
% will generate copies of all the ps files used in the above figure
%%%&^ 1x1bus_ca.ps
%%%&^ 1x1bus_cd.ps
%%%&^ 1x1bus_da.ps
%
Both files contain
panels corresponding to the
lower conductor with conductor name ``{\tt 1}.'' The ``{\tt +}'' 
causes all these
panels to be grouped together and treated as a  single conductor 
named {\tt 1\%GROUP1}, as is required for the calculation.
Without the ``{\tt +}'' {\tt fastcap} would associate the 
``{\tt 1}'' panels in \verb~cond_air_1x1.qui~ with a conductor named
{\tt 1\%GROUP1},
and those in \verb~cond_dielec_1x1.qui~ with another conductor
named {\tt 1\%GROUP2}.

If
files are in a group (that is they are listed consecutively in the list file
and have ``{\tt +}'' symbols linking them together), all the generic
format files must be listed before any PATRAN-generated neutral files. 
This restriction
is necessary due to the ability to rename conductors in generic format files.
Also, {\tt G} lines used to rename groups to be something other than
the default {\tt GROUP<num>}, must occur immediately before the group they
rename.

All the features of the list file interface are further illustrated
by the example list file {\tt test.lst} which is generated by
typing 
\begin{quote}\tt
testgen.sh
\end{quote}
The script {\tt testgen.sh} uses {\tt cubegen} and {\tt busgen} to
create all the input files required by {\tt test.lst}.
Section~\ref{listeg} describes the geometry
corresponding to {\tt test.lst}.
