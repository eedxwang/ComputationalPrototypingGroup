\section{Implementation in FastCap}
\label{ressec}

Our implementation of the multipole-accelerated capacitance extraction
algorithm uses an optimization which exploits the fact that the
conversion and shift operations are linear functions of the charges or
the expansion coefficients, when the geometry is fixed.  That is, the
complicated evaluations involved in converting charges to potentials
or multipole coefficients, shifting multipole coefficients, converting
multipole coefficients to local coefficients, shifting local
coefficients, and converting local coefficients to potentials, are all
computed once, and stored as matrices which operate on charges or
coefficients.  

As an example, consider forming a $1^{st}$-order multipole
expansion
for a collection of $k$ charges.
Following (\ref{eq:multi}), a  $1^{st}$-order multipole expansion has the form
\begin{equation}
\psi(r, \theta , \phi ) \approx
M^0_0\frac{Y^0_0(\theta, \phi)}{r} 
+M^{-1}_1\frac{Y^{-1}_1(\theta, \phi)}{r^2} 
+M^0_1\frac{Y^0_1(\theta, \phi)}{r^2} 
+M^{1}_1\frac{Y^1_1(\theta, \phi)}{r^2},
\label{nmulti}
\end{equation}
where $M^0_0$, \ldots, $M^1_1$ are complex multipole coefficients.
Since $M_n^0$ is real for all $n$, and 
$M_n^{-m}$ is always  the complex conjugate of $M_n^m$,
the multipole expansion can be written in terms of real coefficients as
\begin{equation}
\psi(r, \theta, \phi) \approx
\Mb{0}{0}\frac{P^0_0(\cos\theta)}{r}
+\Mb{1}{0}\frac{P^0_1(\cos\theta)}{r^2}
+\Mb{1}{1}\frac{P^1_1(\cos\theta)\cos\phi}{2r^2}
+\Mbb{1}{1}\frac{P^1_1(\cos\theta)\sin\phi}{2r^2},
\label{sphmul}
\end{equation}
where $\Mb{0}{0}$, $\Mb{1}{0}$, $\Mb{1}{1}$ and $\Mbb{1}{1}$ are real 
coefficients and $P_n^m(\cos\theta)$ is the associated
Legendre function of degree $n$ and order $m$.
This equation appears as (\ref{remult}) in  %Appendix~\ref{appfor}, 
the appendix,
where it is discussed
in more detail. This low order expansion can be more simply represented as
\begin{equation}
\psi(x, y, z)\approx
\Mb{0}{0}\frac{1}{r}
+\Mb{1}{0}\frac{z}{r^3}
-\Mb{1}{1}\frac{x}{2r^3}
-\Mbb{1}{1}\frac{y}{2r^3},
\end{equation}
where $x$, $y$ and $z$ are the evaluation point's cartesian coordinates
and $r = \sqrt{x^2+y^2+z^2}$, as usual.

The real coefficients are calculated using the %Appendix~\ref{appfor} formulas
appendix formulas
(\ref{barmul}) and (\ref{dbrmul}) which
are analogous to (\ref{eq:mulmom}). Writing the four equations for the 
four real coefficients as one matrix equation yields the $4\times k$ linear system
\begin{equation}
\left[
\begin{array}{lcl}
P_0^0(\cos\alpha_1)&
\cdots &
P_0^0(\cos\alpha_k)
\\
\rho_1 P_1^0(\cos\alpha_1)&
\cdots &
\rho_k P_1^0(\cos\alpha_k)
\\
2 \rho_1 P_1^1(\cos\alpha_1) \cos\beta_1&
\cdots &
2 \rho_k P_1^1(\cos\alpha_k) \cos\beta_k
\\
2 \rho_1 P_1^1(\cos\alpha_1) \sin\beta_1&
\cdots &
2 \rho_k P_1^1(\cos\alpha_k) \sin\beta_k
\end{array}
\right]
\left[
\begin{array}{c}
q_1\\
\vdots\\q_k
\end{array}
\right]
=
\left[
\begin{array}{c}
\Mb{0}{0}\\ \Mb{1}{0}\\ \Mb{1}{1}\\ \Mbb{1}{1}
\end{array}
\right],
\end{equation}
where $q_1$, \ldots, $q_k$ are the values of the $k$ charges.
The $4\times k$ matrix is called the Q2M conversion matrix. Its $i^{th}$
column depends only on the coordinates of the $i^{th}$ charge.
Substituting for the associated Legendre functions using (\ref{mmlegn})
and (\ref{mm1leg}) from  %Appendix~\ref{appfor} 
the appendix and switching to
rectangular coordinates simplifies the matrix to
\begin{equation}
\left[
\begin{array}{ccc}
1&
\cdots &
1
\\
z_1&
\cdots &
z_k
\\
-2 x_1&
\cdots &
-2 x_k
\\
-2 y_1&
\cdots &
-2 y_k
\end{array}
\right]
\left[
\begin{array}{c}
q_1\\
\vdots\\q_k
\end{array}
\right]
=
\left[
\begin{array}{c}
\Mb{0}{0}\\ \Mb{1}{0}\\ \Mb{1}{1}\\ \Mbb{1}{1}
\end{array}
\right].
\end{equation}
Note that the first row of the matrix implies that $\Mb{0}{0}$ is
the sum of all the charge strengths, making it identical to the
coefficient $M_0^0$ in (\ref{nmulti}).

Since the Q2M matrix is a function of the charge positions alone, its 
entries need be
calculated only once if several multipole algorithm potential evaluations 
are required for the same charge geometry.
In the notation of Algorithm~2, this amounts to using
the multipole algorithm to compute several $Pw$ products with the same $P$ but
with different $w$ vectors.  Each time the multipole algorithm
is used to form a different $Pw$ product, a new vector of charge strengths
is multiplied by the Q2M matrix yielding a vector of updated 
multipole expansion coefficients. In a similar way, geometry dependent matrices
for all the other multipole algorithm conversions and shifts can be constructed
and used repeatedly in subsequent $Pw$ product calculations.

In our implementation of
the complete multipole-accelerated capacitance extraction
algorithm, given below, the shift and conversion coefficients are computed once
and stored.

%\begin{table}
%\begin{singlespace}
\begin{tabbing}
{\bf Algorithm 3: Multipole-Accelerated Capacitance Extraction Algorithm}\\[0.2in]
T \= \kill
\>T \= \kill
\>\>T \= \kill
\>\>\>T \= \kill
\> /* Setup Phase. */ \\
\> Divide the $ m $ conductors into a total of $ n $ panels.\\
\> Divide the problem domain into a hierarchy of cubes, so that each \\
\> \> of the finest level cubes has a maximum of $ 4 $ panels.\\
\> Compute the conversion and shifting matrices from the topology.\\
\> /* Loop Through all the Conductors. */\\
\> For $ j = 1 $ to $ m $ \{ \\
\> \> /* Set the Potential of the Panels on Conductor $ j $ to one. */ \\
\> \> For $ k = 1 $ to $ n $ \{ \\
\> \> \> If panel $k$ is on conductor $ j $, set $p_k $ = 1.\\
\> \> \> Else $ p_k = 0 $.\\
\> \> \} \\
\> \> /* Solve for the Panel Charges using MGCR. */\\
\> \> Use GCR (Algorithm 1) to solve $ P q = p $, using \\
\> \> \> Multipole (Algorithm 2) to compute $ P w $.\\
\> \> /* Sum the Charges on Conductor $ i $ to Compute $ C_{ij} $ */\\
\> \> For $ i = 1 $ to $ m $\rule{0.2in}{0in} $ C_{ij} = \sum_{k \in conductor_i} q_k$.\\
\> \}\\[0.2in]
\end{tabbing}
%\end{singlespace}
%\end{table}

%\begin{figure}
%\vspace{2.7in}
%V\special{psfile=/cheetah/ksn/drawings/finals1.ps}
%\caption{Bus Structure Example with Six Conductors}
%\end{figure}

\newcommand{\croscap}{ Bus structure test problem with $2\times 2$ conductors.}
% 
\fig{\figuredir/2x2.eps}{2x2}{\croscap}{187}{htbp}{83}{-106}{0.466523}{0.466523}
%
To determine the effectiveness of this approach, the multipole
accelerated algorithm was tested on the easily parameterized bus
structure given in Fig.~\ref{2x2}, for busses with $ 2 \times 2 $ conductors
through $ 6 \times 6 $ conductors. 
The conductor surfaces are discretized by
first cutting each conductor into sections based on where pairs of
conductors overlap.  In the $ 2 \times 2 $ bus example, each conductor
is cut into five sections (see Fig.~\ref{2x2discr}a), 
and in the $ 6 \times 6 $
example each conductor is divided into thirteen sections.  The
discretization is then completed by dividing each face of each section
into nine panels, as demonstrated in Fig.~\ref{2x2discr}b. 
The edge panels have widths that are 10\% of the inner panel widths
to accurately discretize the expected increased charge density near conductor 
edges \cite{rue}.
\newcommand{\discrcap}{ Conductor sections are divided into panels.}
%
\fig{\figuredir/2x2discr.eps}{2x2discr}{\discrcap}{216}{htbp}{-35}{-271}{0.84375}{0.84375}

In Table~\ref{times} below, we report the results of our experiments with the
various approaches to solving  (\ref{eq:qtop}), the matrix
problem associated with the boundary element method.  In the table,
the total number of panels resulting from the conductor surface
discretization is given, followed by the CPU times (on an I.B.M.\ 6000)
required to compute the entire $ m \times m $ capacitance matrix,
where $ m $ is the number of conductors.  Three methods for solving
  (\ref{eq:qtop}) are compared: direct or $ LU $ factorization,
GCR, and multipole accelerated GCR (MGCR).  The MGCR  algorithm's
CPU times are strongly dependent on the number of expansion terms, so
the time required when zero, first and second order expansions ($
l=0,~1,~2 $) are used is given.  Also in the table are the total
number of iterations required to reduce the max norm of the residual
in GCR and MGCR
to $ 1 \% $. 
The CPU times in parenthesis are extrapolated. They correspond
to calculations that were not possible because of the 
excessive memory required to store the entire potential coefficient matrix,
and our lack of patience. 

%\begin{singlespace}

\begin{table}
\begin{center}
\begin{tabular}{|l|c|c|c|c|c|}\hline\hline
&\multicolumn{5}{|c|}{Test Problem}\\\cline{2-6}
& $ 2 \times 2 $ & $ 3 \times 3 $ & $ 4 \times 4 $ & $ 5 \times 5 $ & $ 6 \times 6 $\\\hline\hline
panels & 792 & 1620 & 2736 & 4140 & 5832\\\hline
%direct time & 158 & 2050 & 10686 & 40369 & (144998) \\
%GCR time   & 23 & 132 & 518 & 1585 & (4150) \\
%MGCR time ($l=0$) & 12 & 34 & 86 & 169 & 294 \\
%MGCR time ($l=1$) & 18 & 48 & 141 & 278 & 443 \\
%MGCR time ($l=2$) & 22 & 68 & 181 & 409 & 668 \\\hline
%GCR iters & 28 & 48 & 72 & 100 & (132) \\
%MGCR iters ($l=0$) & 32 & 58 & 90 & 121 & 166 \\
%MGCR iters ($l=1$) & 28 & 48 & 80 & 109 & 144 \\
%MGCR iters ($l=2$) & 28 & 48 & 72 & 109 & 142 \\\hline
%
%direct time & 159 & 2068 & 10649 & 40369 & (134951)\\
%GCR time & 23 & 132 & 518 & 1585 & (4789)\\
%MGCR time ($l=2$) & 22 & 71 & 194 & 420 & 713\\
%MGCR time ($l=1$) & 17 & 48 & 151 & 287 & 468\\
%MGCR time ($l=0$) & 11 & 34 & 89 & 164 & 294\\
%\hline
%GCR iters & 28 & 48 & 72 & 100 & (132)\\
%MGCR iters ($l=2$) & 28 & 48 & 72 & 108 & 144\\
%MGCR iters ($l=1$) & 28 & 48 & 80 & 108 & 144\\
%MGCR iters ($l=0$) & 28 & 58 & 88 & 113 & 159\\
%\hline
%
direct time & 275 & 2700 & 12969 & 44345 & (141603)\\
GCR time & 121 & 570 & 2115 & 4881 & (14877)\\
MGCR time ($l=2$) & 55 & 218 & 378 & 790 & 1412\\
MGCR time ($l=1$) & 29 & 108 & 245 & 436 & 775\\
MGCR time ($l=0$) & 9 & 48 & 98 & 216 & 356\\
\hline
GCR iters & 48 & 78 & 120 & 150 & (180)\\
MGCR iters ($l=2$) & 48 & 82 & 120 & 150 & 180\\
MGCR iters ($l=1$) & 54 & 88 & 120 & 150 & 180\\
MGCR iters ($l=0$) & 58 & 90 & 120 & 150 & 180\\\hline
%
\end{tabular}
\end{center}
\caption{Comparison of Extraction Methods, CPU Times in I.B.M.\ 6000 Seconds}
\label{times}
\end{table}

%\end{singlespace}

As is clear from the results, the multipole-accelerated GCR algorithm
is very effective for the larger problems, particularly if the
expansion order is low. To examine the effect of expansion order on
accuracy, in Table~\ref{values} we compare the resulting capacitances computed
by solving   (\ref{eq:qtop}) for the $ 4 \times 4 $ conductor
problem with $LU$ factorization, GCR and  MGCR for 
expansion orders $ 0 $, $ 1 $, and
%
\newcommand{\resfoot}{In the $4\times 4$ conductor example the lengths
have been normalized so that the conductors are each five meters long,
one meter high and one meter wide, and all interconductor spaces are
one meter. The
capacitances are given in picofarads by scaling the program results by
$4\pi\epsilon_0 = 111.27$ pF/m.} 
$ 2 $.  One row of the $ 4 \times 4 $ capacitance matrix\footnote{\resfoot}
%\footnotemark[2]
is given for
the five different solution methods.  The row chosen represents the
capacitance associated with one of the conductors on the outer edge.
Taking the direct method results as exact, the data indicate that MGCR can
attain better than $ 10 \% $ accuracy in
the diagonal entry of the capacitance matrix with only a zero order ($l=0$)
expansion.  To achieve reasonable relative accuracy in the smallest coupling
capacitance, $C_{14}$, which is fifty times smaller than the
diagonal entry, $C_{11}$, the $ 2^{nd} $-order expansion is required.
In that case MGCR produces results nearly identical to GCR indicating
that further increases in accuracy would require tightening the iterative
loop tolerance as well as increasing the expansion order.

%\begin{singlespace}

\begin{table}
\begin{center}
\begin{tabular}{|l|c|c|c|c|c|c|c|c|}\hline\hline
\multicolumn{1}{|c|}{Solution}&\multicolumn{8}{|c|}{Capacitance Matrix Entry (pF)}\\\cline{2-9}
\multicolumn{1}{|c|}{Method} & $ C_{11} $ & $ C_{12} $ & $ C_{13} $ & $ C_{14} $ & $ C_{15} $ & $ C_{16} $ & $ C_{17} $ & $ C_{18} $ \\\hline\hline
%direct    & $401.0$ & $-135.3$ & $-12.05$ &$-7.872$ &$-47.92$ &$-39.75$ &$-39.75$ &$-47.92$  \\
%GCR    & $400.8$ &$-135.1$ &$-12.34$ &$-7.968$ &$-47.73$ &$-39.59$ &$-39.70$ &$-47.97$ \\
%MGCR ($l=0$) &$413.2$ &$-131.8$ &$-8.451$&$-2.395$ &$-54.54$ &$-45.77$ &$-45.64$ &$-53.34$ \\
%MGCR ($l=1$) &$401.8$ &$-137.1$ &$-8.566$ &$-5.122$ &$-48.23$ &$-39.81$ &$-40.71$ &$-49.12$ \\
%MGCR ($l=2$) &$401.0$ &$-135.0$ &$-12.44$ &$-8.077$ &$-47.71$ &$-39.65$ &$-39.83$ &$-48.03$ \\\hline
%
%direct & 401.0 & -135.3 & -12.05 & -7.872 & -47.92 & -39.75 & -39.75 & -47.92\\
%GCR & 400.8 & -135.1 & -12.33 & -7.969 & -47.74 & -39.59 & -39.70 & -47.97\\
%MGCR ($l=0$) & 409.6 & -130.9 & -9.313 & \multicolumn{1}{|r|}{2.089} & -53.37 & -44.04 & -44.99 & -51.96\\
%MGCR ($l=1$) & 401.9 & -134.5 & -13.47 & -7.524 & -48.26 & -39.84 & -39.78 & -48.29\\
%MGCR ($l=2$) & 401.1 & -135.0 & -12.44 & -8.086 & -47.72 & -39.67 & -39.80 & -48.02\\
%\hline
% data with skinny tiles below
direct &$404.6$& $-137.0$& $-12.04$ &$-7.910$& $-48.42$ &$-40.09$&$-40.09$&$-48.42$\\
GCR& $404.2$ & $-137.2$ &$-11.64$ & $-8.083$ & $-48.37$ & $-39.93$ & $-39.93$ & $-48.37$\\
MGCR ($l=2$)& $405.2$ & $-137.8$ & $-11.91$ & $-8.079$ & $-48.36$ & $-40.09$ & $-40.01$ &$-48.45$\\ 
MGCR ($l=1$)& $406.6$ & $-139.7$ & $-12.36$ & $-6.676$ & $-48.48$ & $-40.45$ & $-40.27$ & $-48.46$\\
MGCR ($l=0$)&$394.5$ & $-124.0$ & $-0.175$ & $-2.471$ & $-52.15$ & $-43.39$ & $-43.08$ & $-52.92$\\\hline
\end{tabular}
\end{center}
\caption{Comparison of Extraction Methods, $4\times 4$ Conductor Problem Capacitances}
\label{values}
\end{table}

%\end{singlespace}

As mentioned above, computing the potentials given a new charge vector
using the multipole algorithm just involves applying mapping matrices
to the changing charges or multipole and local coefficients.  Viewed
in this way, the multipole algorithm can be compared more precisely to
explicitly
computing the matrix-vector product $ P w $ by counting the
number of multiply-add operations required in each case.  
This is a way of comparing MGCR to GCR that eliminates the effects of
implementation differences.
The number of multiply-add operations required
by the four iterative methods of Tables~\ref{times} and~\ref{values} 
for each $Pw$ product
is plotted in Fig.~\ref{opplot}, as a function
of problem size.
\newcommand{\opplotcap}{ Multiply-Add operations required for a single $Pw$ calculation as a function of problem size.}
%  
\fig{\figuredir/opplot.xps}{opplot}{\opplotcap}{312}{htbp}{-49}{-364}{0.896018}{0.896018}
%
As seen in the plot,
all the MGCR methods require fewer operations than GCR
for each $Pw$ product when
the number of panels, $n$, exceeds approximately $1000$. 
Furthermore, the cost of a $Pw$ 
calculation grows as $n^2$ for GCR but roughly as $n$ for the MGCR
methods, with the expansion order affecting only the slope.

\section{Conclusions and Acknowledgments}
\label{consec}

The above results indicate that the multipole-accelerated GCR (MGCR)
algorithm is faster than both $LU$ factorization and GCR alone for
problems with more than $ 1000 $ panels. Capacitance extraction
problems with nearly $6000$ panels can be solved using MGCR in about twenty-five
minutes on a workstation, roughly ten times faster than GCR,
with comparable accuracy.
Furthermore, the multipole algorithm provides an
approach for trading accuracy for CPU time through reducing the
expansion order; if low accuracy is tolerable, another factor of three or four
speed improvement can be obtained.  Finally, MGCR does not require
explicit storage of the entire potential coefficient matrix as do GCR
and $LU$ factorization, resulting in significantly smaller memory
requirements.  Combining these features makes it feasible to perform
capacitance extraction of complicated structures as the ``inner loop''
of a design optimization procedure.

Additional work is under way to improve the efficiency of the
multipole algorithm, and we are currently working on making the
algorithm adaptive.  Future research includes investigating using
block iterative techniques to reduce the number of iterations required
and extending the approach to solving problems with piecewise-constant
dielectrics and ground planes.  In addition, we will extend our
program to handle arbitrary triangular and 
quadrilateral panels so that more esoteric
structures can be discretized.

The authors would like to thank David Ling and Albert Ruehli of the
I.B.M.\ T.\ J.\ Watson Research Center for the many discussions and
their help in understanding the integral equation method, and Albert
Ruehli for the suggestion that led to the approach presented here.  In
addition, we would like to acknowledge the helpful discussions with
Tom Simon and the members of the M.I.T.\ custom integrated circuits
group.
