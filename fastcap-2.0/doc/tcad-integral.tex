\section{The Integral Equation Approach}

Consider a system of $ m $ ideal conductors embedded in a uniform
lossless dielectric medium.  For such a system, the relation between
the $ m $ conductor potentials, denoted by $ \hat{p} \in \Re^m $, and
the $ m $ total charges on each conductor, denoted by $ \hat{q} \in
\Re^m $, is given by $ \hat{q} = C \hat{p} $, where $ C \in
\Re^{m\times m} $ is referred to as the capacitance matrix. The 
$ j^{th} $ column of $ C $ can be calculated by solving for the total
charges on each of the conductors when the $ j^{th} $ conductor is at
unit potential, and all the other conductors are at zero potential.
Then the charge on conductor $ i $, $ \hat{q_i} $, is equal to $
C_{ij} $.

To find the conductor charge distributions given the conductor potentials,
it is necessary to solve the first-kind integral equation 
\begin{equation}
%\psi (x) = \int_{ \mbox{ \em \scrpitsize surfaces} } G(x,x') \sigma (x') da'
\psi (x) = \int_{  sur\!f\!aces } G(x,x') \sigma (x') da'
\label{eq:integral}
\end{equation}
for the surface charge density $ \sigma $, where $ x$, $x' \in \Re^3 $
and are positions in 3-space,
$ da' $ is the incremental surface area, $ \psi $ is the
surface potential and is known, and $ G(x,x') $ is the
Green's function, which is 
$ \frac{1}{\| x - x' \| } $ in free space
%\footnotemark[1].
\newcommand{\integfoot}{Note that the scale factor 
$1/4\pi\epsilon_0$ can be ignored here, and
reintroduced later to give the results in units of farads.}
\footnote{\integfoot}.
Here, $ \| x - x' \| $ denotes the Euclidean distance between $ x $ 
and $ x' $.
Given the surface charge density $ \sigma $, the total charge
on the $ i^{th} $ conductor, $ Q_i $, can be computed from
\begin{equation}
%Q_i = \int_{i^{th} \mbox{ \em \scrpitsize conductor's surface} } \sigma (x') da'.
Q_i = \int_{i^{th}\; conductor's\; sur\!f\!ace } \sigma (x') da'.
\end{equation}

There are a variety of approaches for numerically computing the
conductor surface charge density given the conductor potentials, some
of which involve reformulating   (\ref{eq:integral}) as a partial
differential equation, and using finite difference methods
in three space dimensions \cite{zem,gue}.  We will focus on the
boundary-element methods applied directly to solving  
(\ref{eq:integral})  \cite{rue,rao,Nin}, as they have proved to be
efficient and accurate when applied to problems with ideal conductors
in a uniform dielectric medium.  
%It is curious to note that these
These
methods are also referred to as panel methods  \cite{hes}, or the method
of moments  \cite{har}, in other application domains.  This class of
method exploits the fact that the charge is restricted to the surface
of the conductors, and rather than discretizing all of free space,
just the surface charge on the conductors is discretized.  The surface
potential, which is known, is related to the discretized surface
charge through integrals of Green's functions.  The so-constructed
system can then be solved for the discretized surface charge.

The simplest commonly used approach to constructing a
system of equations that can be solved for the discretized surface
charge is the ``point-matching'' or collocation method.  In this
method, the surfaces of $ m $ conductors in free space are discretized
into a total of $ n $ 2-dimensional panels
(See for example Fig.~\ref{2x2discr}b).
For each
panel $ k $, an equation is written that relates the potential at the
center of that $ k^{th} $ panel to the sum of the contributions to that
potential from the charge distribution on all $ n $ panels.  That is,
\begin{equation}
p_k = 
\sum_{l=1}^n \int_{panel_l} \frac{\sigma_l(x')}{\| x' - x_k \| } \: da',
\label{eq:green}
\end{equation}
where $ x_k $ is the center of panel $ k $, 
$ x' $ is the position on the surface of panel $ l $, $ p_k $ is the
potential at the center of panel $ k $, and $ \sigma_l(x')$ is the surface
charge density on the $ l^{th} $ panel.  The
integral in (\ref{eq:green}) is the free space Green's function
multiplied by the charge density and integrated over the surface of the $
l^{th} $ panel. Note that as the distance between panel $ k $ and panel $
l $ becomes large compared to the surface area of panel $ l $, the
integral reduces to $ \frac{q_l}{\| x_l - x_k \| } $
where $ x_l $ is the center of the $ l^{th} $ panel and 
$ q_l $ is the total charge on panel $ l $.

In a first-order collocation method (higher order methods are rarely
used), it is assumed that the surface charge density on the panel
is constant  \cite{rao}.  In that case (\ref{eq:green}) can be
simplified to
\begin{equation}
p_k = \sum_{l=1}^n 
\frac{q_l}{a_l} \int_{panel_l} \frac{1}{\| x' - x_k \| } \: da',
\label{eq:sgreen}
\end{equation}
%where $ q_l $ is the total charge on panel $ l $, and 
where $ a_l $ is 
the surface area of panel $ l $.  
When applied to the collection of $ n $ panels, a dense linear system
results,
\begin{equation}
P q = p
\label{eq:qtop}
\end{equation}
where $ P \in \Re^{n\times n}$; $q$, $p\in \Re^n$ and
\begin{equation}
P_{kl} = \frac{1}{a_l} \int_{panel_l} \frac{1}{ \| x' - x_k \| }\: da'.
\label{eq:pcoeff}
\end{equation}
Note that $q$ and $p$ are the vectors of {\em panel} charges and
potentials rather than the {\em conductor} charge and potential
vectors, $\hat{q}$ and $\hat{p}$ mentioned above.  The dense linear
system of (\ref{eq:qtop}) can be solved, typically by some form of
Gaussian elimination, to compute panel charges from a given set of panel
potentials.  To compute the $ j^{th} $ column of the capacitance
matrix, (\ref{eq:qtop}) must be solved for $ q $, given a $ p $ vector
where $ p_k = 1 $ if panel $ k $ is on the $ j^{th} $ conductor, and $
p_k = 0 $ otherwise.  Then the $ ij^{th} $ term of the capacitance
matrix is computed by summing all the panel charges on the $ i^{th} $
conductor, that is
\begin{equation}
C_{ij} = \sum_{k \in conductor_i} q_k.
\end{equation}
