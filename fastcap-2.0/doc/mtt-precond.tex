\section{Preconditioning the Iterative Method}
\label{precon}

In general, the GMRES iterative method applied to solving
(\ref{eq:qtop}) can be significantly accelerated by {\it
preconditioning} if there is an easily computed good approximation to the
inverse of $ P $.  We denote the approximation to $ P^{-1} $ by $
\tilde{C} $, in which case preconditioning the GMRES algorithm is
equivalent to using GMRES to solve
\begin{equation}
P \tilde{C} x = \overline{p}.
\label{eq:pq}
\end{equation}
for the unknown vector $ x $, from which the charge density is
computed by $ q = \tilde{C} x $.  Clearly, if $ \tilde{C} $ is
precisely $ P^{-1} $, then (\ref{eq:pq}) is trivial to solve, but then
$ \tilde{C} $ will be very expensive to compute.

\subsection{A Simple Example}

%%&^^
\newcommand{\screencap}{Simple panel systems with potential coefficient 
matrices $P(7)$, (a), $P(5)$, (b), and $P(3)$, (c). The parallel, 
1m$\times $1m panels are spaced $0.5$m appart.}

\begin{figure}
\setlength{\unitlength}{1.0in}
\begin{picture}(6.5,3.7)
\put(2.5,2.6){
\special{psfile=\figuredir/cap7.ps,hoffset=-6,voffset=-6,hscale=0.204375,vscale=0.204375}
}
\put(2.67,1.4){
\special{psfile=\figuredir/cap5.ps,hoffset=-4,voffset=-4,hscale=0.15,vscale=0.15}
}
\put(2.87,0.2){
\special{psfile=\figuredir/cap3.ps,hoffset=-3,voffset=-3,hscale=0.0965625,vscale=0.0965625}
}
%%\put(0,0){\line(0,1){3}}
\put(3.2,2.4){(a)}
\put(3.2,1.2){(b)}
\put(3.2,0){(c)}
%% numbers on 7 panel
\put(2.78,3.4){1}
\put(2.97,3.378){2}
\put(3.17,3.357){3}
\put(3.39,3.335){4}
\put(3.6,3.313){5}
\put(3.8,3.292){6}
\put(4.02,3.27){7}
%% 5 panel
\put(2.97,2.148){1}
\put(3.17,2.127){2}
\put(3.35,2.105){3}
\put(3.56,2.083){4}
\put(3.78,2.062){5}
%% 3 panel
\put(3.17,0.907){1}
\put(3.36,0.885){2}
\put(3.56,0.863){3}
\end{picture}
\caption{\screencap}
\label{screen}
\end{figure}





A good approximation to $ P^{-1} $ that is easily computed, and fits
with the hierarchical multipole algorithm described previously, can be
derived by exploiting the fact that $ P^{-1} $ is approximately the
{\it detailed} capacitance matrix, by which we mean
the $ n \times n $ capacitance matrix for the problem in which
every panel or tile used to represent the conductor surfaces is
treated as an independent conductor.  To see why this point of view
leads to a preconditioner, consider the $ 7 \times 7 $ $ P $
matrix, denoted $P(7)$, for the seven panel example 
in Figure~\ref{screen}a.  The fourth 
row of $ P(7) $, which is associated with the center panel in 
Figure~\ref{screen}a, can be computed
using the definition in (\ref{eq:pcoeff}) and is, in inverse-farads,
\begin{equation}
\begin{array}{ccccccc}
P(7)_{4,1} & P(7)_{4,2} & P(7)_{4,3} &P(7)_{4,4} & 
P(7)_{4,5} & P(7)_{4,6} & P(7)_{4,7}\\
0.5785 & 0.8346 & 1.4261 & 3.1686 & 1.4261 & 0.8346 & 0.5785
\end{array}
\label{eq:panel7}
\end{equation}
where all the values have been multiplied by $10^{-10}$.
The fourth row of $ P(7)^{-1} $ is, in picofarads,
\begin{equation}
\begin{array}{ccccccc}
P(7)^{-1}_{4,1} & P(7)^{-1}_{4,2} & P(7)^{-1}_{4,3} & 
P(7)^{-1}_{4,4} & P(7)^{-1}_{4,5} & P(7)^{-1}_{4,6} & 
P(7)^{-1}_{4,7} \\ 
-1.3080 &  -1.5898 & -15.4544 &  46.7864 & -15.4544 &  -1.5898 &  -1.3080
\end{array}
\label{invpanel}
\end{equation}

From the definition in (\ref{eq:pcoeff}), the matrix elements $
P_{4,j} $ must decay as $ | j - 4 | $ grows, but notice that the terms
in $ P^{-1}_{4,j} $ decay {\it much faster} as $ | j - 4 | $
increases.  Viewing $ P^{-1} $ as an approximation to the detailed
capacitance matrix makes clear that the very fast decay of terms in $
P(7)^{-1} $ is just an example of classical electrostatic screening.  The
effect of screening can also be seen by examining the row of $ P(5)^{-1}
$ associated with the center panel in the five panel problem in 
Figure~\ref{screen}b and in the row of $ P(3)^{-1} $ associated with the center
panel in the three panel problem in Figure~\ref{screen}c, which
are, in picofarads,
\begin{equation}
\begin{array}{ccccc}
P(5)^{-1}_{3,1} & P(5)^{-1}_{3,2} & P(5)^{-1}_{3,3} & 
P(5)^{-1}_{3,4} & P(5)^{-1}_{3,5} \\
   -2.1593 & -15.5547 &  46.6990 & -15.5547  & -2.1593
\end{array}
\label{eq:panel5}
\end{equation}
and
\begin{equation}
\begin{array}{ccc}
P(3)^{-1}_{2,1} & P(3)^{-1}_{2,2} & P(3)^{-1}_{2,3} \\
  -16.5499 &  46.4573 & -16.5499 
\end{array}
\label{eq:panel3}
\end{equation}
respectively.  Comparing (\ref{invpanel}), (\ref{eq:panel5}) and
(\ref{eq:panel3}) leads to the observation that a good estimate for
the fourth row of $P(7)^{-1} $ can be
derived from the third row of $ P(5)^{-1} $, which is a {\it smaller}
problem, and that even the second row of $ P(3)^{-1} $
provides a reasonable estimate.  Specifically,
the estimate based on the five-panel problem is
\begin{equation}
\tilde{C}_{4,j} = \left\{
\begin{array}{ll}
P(5)^{-1}_{3,3 + (j - 4)}, & |j - 4| <  3;\\
0, & \mbox{otherwise};
\end{array}
\right.
\label{eq:est5}
\end{equation}
where $ \tilde{C} $ denotes the estimate to $ P(7)^{-1} $.

\subsection{Preconditioning Algorithm}

The above example suggests an approach to estimating $ P^{-1} $ for a
general configuration of panels which fits with the
hierarchical multipole algorithm in that the preconditioner $ \tilde{C} $ can
be constructed and applied in a cube-by-cube fashion.  The preconditioner
is formed by inverting a sequence of reduced $ P $ matrices, one associated
with each finest-level cube, as in Algorithm 3 below.

\begin{singlespace}
\begin{quote}
\begin{tabbing}
T \= \kill
\>T \= \kill
\>\>T \= \kill
\>\>\>T \= \kill
\>\>\>\>T \= \kill
\>\>\>\>\>T \= \kill
\>\>\>\>\>\>T \= \kill
{\bf Algorithm 3:} Forming $ \tilde{C} $.\\
\>\>{\bf For each} finest-level cube $ i = 1 $ to $ 8^L $ \{ \\
\>\>\>{\bf Form} $P^{i} $, the potential coefficient matrix for the\\
\>\>\>reduced problem considering only the panels contained\\
\>\>\>in cube $i $ and cube $i $'s neighbors.\\
\>\>\>{\bf Compute} $ \tilde{C}^i = (P^i)^{-1} $.\\
\>\>\>{\bf For each} panel $ k $ in cube $i $ or cube $i $'s neighbors \{\\
\>\>\>\>{\bf if} panel $ k $ is not in cube $ i $ \{\\
\>\>\>\>\> {\bf delete} row $ k $ from $ \tilde{C}^i $.\\
\>\>\>\>{\bf \}}\\
\>\>\>{\bf \}}\\
\>\>{\bf \}}\\
\end{tabbing}
\end{quote}
\end{singlespace}

Note that $ \tilde{C}^i $ is not a square matrix and that
\begin{equation}
\sum_{i=1}^{8^L} (\# \; \; rows \; in \; \; \tilde{C}^{i}) = n 
\end{equation}
where again $ n $ is the total number of panels.
By comparing Algorithm 3 with Algorithm 2, it is clear that
$ P^{i} $ uses only those elements of the full $ P $ matrix which are
already required in Algorithm 2, and therefore the computational cost
in computing the preconditioner is only in inverting small $ P^{i} $ 
matrices.  Then computing the product $ P \tilde{C} x^k $, 
which would be used in a GMRES algorithm applied to
solving (\ref{eq:pq}), is accomplished in two steps. First, the
preconditioner is applied to form $ q^k = \tilde{C} x^k $ using
Algorithm 4 below.  Then, $ P q^k $ is computed using Algorithm 2 in
the previous section.

\begin{singlespace}
\begin{quote}
\begin{tabbing}
T \= \kill
\>T \= \kill
\>\>T \= \kill
\>\>\>T \= \kill
\>\>\>\>T \= \kill
\>\>\>\>\>T \= \kill
\>\>\>\>\>\>T \= \kill
{\bf Algorithm 4:} Forming $ q = \tilde{C} x $.\\
\>\> {\bf For each} finest-level cube $ i = 1 $ to $ 8^L $ \{ \\
\>\>\> {\bf For each} panel $ j $ in finest-level cube $ i $ \{ \\
\>\>\>\> {\bf For each} panel $ k $ in cube $ i $ or its neighbors \{ \\
\>\>\>\>\> {\bf Add} $ \tilde{C}^i_{jk} x_k $ to $ q_j $.\\
\>\>\>\> {\bf \}}\\
\>\>\> {\bf \}}\\
\>\> {\bf \}}\\
\end{tabbing}
\end{quote}
\end{singlespace}

