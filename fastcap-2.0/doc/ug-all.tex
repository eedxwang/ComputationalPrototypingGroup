%\documentstyle[12pt]{article}
%
%%%%&^^ /nfs/sobolev/sobolev1/songmin/Manual/figure
%
%\begin{document}
\section{How to Prepare Input Files}
\label{all}

This section of the manual describes how to prepare input files for
FastCap.  The input files specify the \underline{discretization} of
\underline{conductor surfaces} into \underline{panels}.  There are
two types of input files that can be read by FastCap.  One is the
PATRAN neutral file, and the other is a generic file format designed
for FastCap.  These two have unique first line formats, so that
FastCap can internally determine the type.  Users not wishing to
use the PATRAN program to prepare an input file can skip to 
Section~\ref{generc}.
   
\subsection{PATRAN Neutral File Interface}
\label{neuif}

PATRAN is a general purpose, three-dimensional solid modeler with
interactive graphics.  FastCap includes an interpreter for PATRAN's
standard output, the neutral file.  This feature allows the user to
construct and analyze more complicated conductor structures with help
of visualization\footnote{Any geometry that can be read by
{\tt fastcap} can also be viewed using the postscript file dump options
described in Section~\ref{genpic}.}.  
This section briefly describes how to use PATRAN to
build such solid models, and how to generate the neutral file suitable
for FastCap.  For a more complete description on using PATRAN, see the
PATRAN Release 2.4 user's manual.


\subsubsection{PATRAN Data}

Before going on any further, one has to understand the hierarchy of
data for solid models in PATRAN.  There are two levels of data
relevant to FastCap, geometry data and analysis model data.

Data for geometric entities describe {\em grids, lines, patches} and
{\em hyperpatches}.  This data type is also referred to as phase 1
data in PATRAN.  The {\em grids} are simply points in space.  A {\em
patch} is a two-dimensional surface with four corner {\em grids},
while a {\em hyperpatch} defines a solid with eight corner {\em
grids}.

Analysis model data, phase 2 data, are built on top of geometric data
as they are created by meshing geometric entities.  The resulting data
consist of {\em elements} and {\em nodes}, that are also referred to
as CFEG and GFEG data.

\subsubsection{Using PATRAN for FastCap}

FastCap interprets PATRAN \underline{\em patches} \underline{as}
\underline{conductor} \underline{surfaces}.  That is, a conductor is a
collection of {\em patches} that are connected.  Discretizing the
surfaces into FastCap \underline{panels} corresponds to meshing the
{\em patches} into PATRAN's quadrilateral and triangular
\underline{\em elements}.

In building conductor models, the following steps should be taken.
\begin{enumerate}
\item  
Build \underline{conductor surfaces} using {\em patches} with corner
{\em grids}.  All other geometric entities are ignored by FastCap, and
can be used only as intermediate entities to help create {\em
patches}. Note: all the patches on a given conductor must be connected.
Patches are connected when they share a grid. If this connectivity 
constraint is ignored, FastCap will interpret the disconnected sets of
patches as separate conductors.
\item  
Name each conductor using the PATRAN NAME command.  Each conductor
can be named by naming at least one of the conductor's {\em patches}.
\item 
Mesh the {\em patches} using the PATRAN MESH command.  This
corresponds to placing {\em nodes} and {\em elements} on the {\em
patches}.  The meshing operation produces a
\underline{discretization} of the \underline{conductor surfaces} into PATRAN
{\em elements}, also referred to as FastCap panels.  One has to
remember to create only the zeroth order triangular or quadrilateral
{\em elements}.  In PATRAN they correspond TRI/3/0 or QUAD/4/0 {\em
elements}.
\item  
Finally, write out a neutral file from PATRAN.  Be sure to output both
geometric (phase 1) and analysis model (phase 2) data to the neutral
file (see the PATRAN manual for more information about creating
neutral files).
\end{enumerate}

\subsubsection{Example}

Consider computing the capacitance of the example in
Figure~\ref{pateg}(a), which consists of two conductors.  Each
conductor has six faces, and thus requires six {\em patches} to model
their surfaces.  Since each set of six {\em patches} are connected,
they will be interpreted as being from one conductor by FastCap.  The
following is the list of the PATRAN commands to build all the
geometric entities in Figure~\ref{pateg}(a).

\begin{figure}
\setlength{\unitlength}{1.0in}
\begin{picture}(6.25,7)(0,0)

\put(0.0,3.8){
\psfig{figure=\figuredir/bus_x_one.ps,width=3.1in}
%%%&^ bus_x_one.ps w3.1
%%%&^s
}

\put(3.15,3.8){
\psfig{figure=\figuredir/bus_x_one_coarse_mesh.ps,width=3.1in}
%%%&^ bus_x_one_coarse_mesh.ps w3.1
%%%&^s
}

\put(0.0,0.1){
\psfig{figure=\figuredir/bus_x_one_fine_mesh.ps,width=3.1in}
%%%&^ bus_x_one_fine_mesh.ps w3.1
%%%&^s
}

\put(3.15,0.1){
\psfig{figure=\figuredir/bus_x_one_nonunif.ps,width=3.1in}
%%%&^ bus_x_one_nonunif.ps w3.1
%%%&^s
}
\put(1.5,3.7){(a)}
\put(4.65,3.7){(b)}
\put(1.5,0.0){(c)}
\put(4.65,0.0){(d)}
\end{picture}
\caption{Example Conductor Geometry with Various Discretizations}
\label{pateg}
\end{figure}

First, put down {\em grids} to use as {\em patch} corner points by typing:
\begin{quote}
\begin{verbatim}
GRID,  1, , 0.0/0.0/0.0
GRID,  2, , 1.0/0.0/0.0
GRID,  3, , 1.0/1.0/0.0
GRID,  4, , 0.0/1.0/0.0
GRID,  5, , 0.0/0.0/3.0
GRID,  6, , 1.0/0.0/3.0
GRID,  7, , 1.0/1.0/3.0
GRID,  8, , 0.0/1.0/3.0
GRID,  9, ,-1.0/2.0/1.0
GRID, 10, ,-1.0/2.0/2.0
GRID, 11, ,-1.0/3.0/2.0
GRID, 12, ,-1.0/3.0/1.0
GRID, 13, , 2.0/2.0/1.0
GRID, 14, , 2.0/2.0/2.0
GRID, 15, , 2.0/3.0/2.0
GRID, 16, , 2.0/3.0/1.0
\end{verbatim}
\end{quote}

Secondly, construct twelve quadrilateral {\em patches} by specifying
their corner {\em grid} ID's by typing:
\begin{quote}
\begin{verbatim}
PATCH, 1, QUAD, , 1/2/3/4
PATCH, 2, QUAD, , 5/6/7/8
PATCH, 3, QUAD, , 1/4/8/5
PATCH, 4, QUAD, , 4/3/7/8
PATCH, 5, QUAD, , 3/2/6/7
PATCH, 6, QUAD, , 2/1/5/6
PATCH, 7, QUAD, , 9/10/11/12
PATCH, 8, QUAD, , 13/14/15/16
PATCH, 9, QUAD, , 9/12/16/13
PATCH,10, QUAD, , 12/11/15/16
PATCH,11, QUAD, , 11/10/14/15
PATCH,12, QUAD, , 10/9/13/14
\end{verbatim}
\end{quote}

Then, erase all the geometry entities by deleting from the active set.
This can be done by typing:
\begin{quote}
\begin{verbatim}
SET, ACTIVE, NONE
\end{verbatim}
\end{quote}

Plot one {\em patch} at a time, and use the PATRAN NAME command to
name it.  Again, only \underline{one} {\em patch} from each conductor
needs to be named.
\begin{quote}
\begin{verbatim}
PATCH, 1, PLOT
NAME, CONDUCTOR1
PATCH, 1, ERASE

PATCH, 7, PLOT
NAME, CONDUCTOR2
PATCH, 7, ERASE
\end{verbatim}
\end{quote}

Before the next step, plot the entire figure back on the screen by
typing:
\begin{quote}
\begin{verbatim}
SET, ACTIVE, ALL
\end{verbatim}
\end{quote}

After the {\em patches} are created, they must be meshed before
creating a FastCap compatible neutral file.  As shown in
Figure~\ref{pateg}(b), PATCH 1 through 12 can be discretized into
QUAD/4/0 {\em elements} by typing:
\begin{quote}
\begin{verbatim}
MESH, P1T12, QUAD/4/0, LENGTH, 1.0
\end{verbatim}
\end{quote}

Begin to write out a neutral file by typing:
\begin{quote}
\begin{verbatim}
NEUTRAL
\end{verbatim}
\end{quote}
PATRAN will produce a couple of neutral system menus.  The user should
choose CREATE OUTPUT and ENTIRE MODEL options in those menus,
respectively.  PATRAN will follow with two questions, ``Do you wish to
output Phase-I data? (Y/N)'', and ``Do you wish to output GFEG/CFEG
tables? (Y/N)''.  Y responses to both questions will finally generate
a neutral file suitable for FastCap.  

Now the user is ready to run FastCap with the neutral file by exiting
from PATRAN and by typing:
\begin{quote}
\begin{verbatim}
fastcap patran.out.1
\end{verbatim}
\end{quote}
where {\tt patran.out.1} is the name of the neutral file.  The user
can find more information on how to run FastCap in the next chapter.
Shown below is a part of FastCap output that displays the capacitance
matrix for the example in Figure~\ref{pateg}(b).
\begin{quote}
\begin{verbatim}
CAPACITANCE MATRIX, picofarads
                           1         2 
CONDUCTOR1%GROUP1 1     135.2    -56.58
CONDUCTOR2%GROUP1 2    -56.58     135.2
\end{verbatim}
\end{quote}

\subsubsection{Modifying the Discretization}

In many cases a user might need to generate several discretizations of
the conductor surfaces.  For example the user might wish to run
FastCap with gradually finer meshes to insure that the discretization
error is small.  The user can rediscretize the surfaces by:  
\begin{enumerate}
\item
Reading in the old PATRAN neutral file, if the user is starting a
new PATRAN session.
\item
Remeshing the {\em patches}.
\item
Writing out a new neutral file.
\end{enumerate}

As an example, the conductor surfaces in Figure~\ref{pateg}(b) can be
discretized into smaller panels as shown in Figure~\ref{pateg}(c).
The following is the list of PATRAN commands to accomplish this panel
refinement. 

To read in the old neutral file type
\begin{quote}
\begin{verbatim}
NEUTRAL
\end{verbatim}
\end{quote}
PATRAN will respond with a neutral system menu.  The user should
choose the INPUT MODEL option, and provide the name of the neutral
file when asked.

Then, the {\em patches} can be remeshed by typing:
\begin{quote}
\begin{verbatim}
MESH, P1T12, QUAD/4/0, LENGTH, 0.3333
\end{verbatim}
\end{quote}
PATRAN will respond by asking the user, ``WILL YOU PERMIT OVERWRITE?
(Y?N?S)'' for each patch, since the {\em patches} are already meshed.

Finally, the user can repeat the last step in the previous section to
output a new neutral file.

The following is a part of FastCap output from running the example in
Figure~\ref{pateg}(c).
\begin{quote}
\begin{verbatim}
fastcap bus_x_one_fine_mesh.out

CAPACITANCE MATRIX, picofarads
                           1         2 
CONDUCTOR1%GROUP1 1     141.8    -60.81
CONDUCTOR2%GROUP1 2    -60.81     141.8
\end{verbatim}
\end{quote}

Notice that the capacitance values are different from those reported
for the coarse discretization shown in Figure~\ref{pateg}(b).  The
difference is due to the reduced discretization error, which the user
must always be aware of when discretizing conductor surfaces.

PATRAN also provides several ways to create non-uniform meshes.  The
discretization shown in Figure~\ref{pateg}(d) is generated in PATRAN
by using interactive {\em node} and {\em element} editing
capabilities.  See PATRAN Release 2.4 user's manual for more details.

The results from running FastCap for the discretization is shown
below.  Note that the values are only slightly different from the
previous results in Figure~\ref{pateg}(d). 
\begin{quote}
\begin{verbatim}
fastcap bus_x_one_nonuniform_mesh.out

CAPACITANCE MATRIX, picofarads
                           1         2 
CONDUCTOR1%GROUP1 1       143    -61.69
CONDUCTOR2%GROUP1 2    -61.69       143
\end{verbatim}
\end{quote}


\subsection{Generic File Interface}
\label{generc}

In addition to PATRAN neutral files, FastCap can read files in
a simple generic format.
Each panel's coordinates are listed in a single line along with its
parent conductor name.

The very first line must be a title line, which begins with {\tt 0}
followed by an arbitrary title as shown below.
\begin{quote}
\begin{verbatim}
0  <title>
\end{verbatim}
\end{quote}  
Individual panels representing the discretization of a conductor
surface are specified using any mix of two types of data lines, 
for quadrilateral and triangular panels.  Each data line must start
with either a {\tt Q} or a {\tt T} followed by the conductor numbers
and x, y and z coordinates of four or three corners, respectively, as
shown below.
\begin{quote}\footnotesize
\begin{verbatim}
Q  <cond. name>  <x1> <y1> <z1>  <x2> <y2> <z2>  <x3> <y3> <z3>  <x4> <y4> <z4>  
T  <cond. name>  <x1> <y1> <z1>  <x2> <y2> <z2>  <x3> <y3> <z3> 
\end{verbatim}
\end{quote}  
The corner coordinates must be entered in clockwise or
counterclockwise order starting from any one corner.  The \verb#cond. name#
field is usually a number but may be any string in general.

For convenience, each conductor can be renamed using a line
which begins with {\tt N}. The line
\begin{quote}
\begin{verbatim}
N  <old cond. name>  <new cond. name>
\end{verbatim}
\end{quote}  
associates all the panels previously assigned to \verb#old cond. name#
with the new name \verb#new cond. name#. 

The following is an example of a generic format input file describing the
discretization of one of the two conductors shown in Figure~\ref{pateg}(a).
\begin{quote}
\begin{verbatim}
Q  1   0.0 0.0 0.0  1.0 0.0 0.0  1.0 1.0 0.0  0.0 1.0 0.0
Q  1   0.0 0.0 0.0  1.0 0.0 0.0  1.0 0.0 1.0  0.0 0.0 1.0
Q  1   0.0 0.0 1.0  1.0 0.0 1.0  1.0 0.0 2.0  0.0 0.0 2.0
Q  1   0.0 0.0 2.0  1.0 0.0 2.0  1.0 0.0 3.0  0.0 0.0 3.0
Q  1   0.0 1.0 0.0  1.0 1.0 0.0  1.0 1.0 1.0  0.0 1.0 1.0
Q  1   0.0 1.0 1.0  1.0 1.0 1.0  1.0 1.0 2.0  0.0 1.0 2.0
Q  1   0.0 1.0 2.0  1.0 1.0 2.0  1.0 1.0 3.0  0.0 1.0 3.0
Q  1   1.0 1.0 0.0  1.0 0.0 0.0  1.0 0.0 1.0  1.0 1.0 1.0
Q  1   1.0 1.0 1.0  1.0 0.0 1.0  1.0 0.0 2.0  1.0 1.0 2.0
Q  1   1.0 1.0 2.0  1.0 0.0 2.0  1.0 0.0 3.0  1.0 1.0 3.0
Q  1   0.0 1.0 0.0  0.0 0.0 0.0  0.0 0.0 1.0  0.0 1.0 1.0
Q  1   0.0 1.0 1.0  0.0 0.0 1.0  0.0 0.0 2.0  0.0 1.0 2.0
Q  1   0.0 1.0 2.0  0.0 0.0 2.0  0.0 0.0 3.0  0.0 1.0 3.0
Q  1   0.0 0.0 3.0  1.0 0.0 3.0  1.0 1.0 3.0  0.0 1.0 3.0
N  1   Big
\end{verbatim}
\end{quote}

%
% description of the list file interface
%
\subsection{Using a List File to Specify Complicated Geometries}
\label{listif}

Both generic and neutral format files may be combined using an
auxiliary file called a list file.  The list file lists the file names
that are to be included in the composite problem and allows for
different permittivities to be associated with either side of
the surfaces in each of the files.  Thus the list file interface
provides both a means of entering geometries with multiple dielectrics
and, in general,
a way to build complicated geometries out of simpler structures.

For example, the two-surface geometry of
the coated sphere in Figure~\ref{cut_sph} can be specified using
%
% cut away of coated sphere
%
% for fig2ug.awk
%%%&^ cut_sph.ps
\begin{figure}
\centerline{
\psfig{figure=\figuredir/cut_sph.ps,width=3.0in}
}
\caption{The discretization used to compute the capacitance of a dielectric-coated sphere in free space.  Some of the outer dielectric-boundary panels
have been removed to show the inner conductor surface
panels.}
\label{cut_sph}
\end{figure}
the list file \verb~coated_sph.lst~,
\begin{verbatim}
*
* fastcap surface list file for two concentric spherical shells
* - inner is a conductor surface
* - outer is a dielectric/dielectric (coating/air) interface
*
* radius 1 shell is the conductor
C sphere2.neu 2.0  0.0 0.0 0.0
*
* radius 2 shell is the dielectric interface - air on the outside
D big_sphere1.neu 1.0 2.0  0.0 0.0 0.0  0.0 0.0 0.0 -
\end{verbatim}
The first non-comment line specifies the inner, radius-one conductor,
surrounded by a dielectric coating with relative permittivity {\tt 2.0}.
The last line defines the outer surface of the coating, a dielectric
interface.  

In general a list file has lines of the form given below. 
{\small
\begin{verbatim}
* <comment>
G <group name>
C <file> <outperm> <xtran> <ytran> <ztran> [+]
D <file> <outperm> <inperm> <xtran> <ytran> <ztran> <xref> <yref> <zref> [-]
B <file> <outperm> <inperm> <xtran> <ytran> <ztran> <xref> <yref> <zref> [-][+]
\end{verbatim}
}
\noindent The {\tt C} line is used to include the panels in {\tt file}
as parts of conductor surfaces surrounded by
dielectric with relative permittivity {\tt outperm}. The entire contents
of {\tt file} may be translated by specifying non-zero values for
{\tt <xtran> <ytran> <ztran>}. The {\tt D} and {\tt B} lines specify
dielectric interfaces along two regions with permittivities {\tt outperm}
and {\tt inperm}. The {\tt B} (or ``both'') specifies that the dielectric
interface is coincident with an infinitesimally thin conductor, while
{\tt D} surfaces are dielectric interfaces alone. In both cases, translations
are possible as for {\tt C} surfaces. In all cases {\tt file} can be
in either PATRAN-generated neutral or generic format.

The reference point
{\tt <xref> <yref> <zref>} and optional ``{\tt -}'' 
for {\tt D} and {\tt B} surfaces are used to specify
which side of the interface has which permittivity. In particular,
the reference point {\tt <xref> <yref> <zref>} is assumed to lie
on the {\tt outperm} side of all the panels in {\tt file}. The 
optional ``{\tt -}'' indicates that {\tt <xref> <yref> <zref>} instead
lies on the {\tt inperm} side. It is the user's responsibility to make
sure the reference point is on the same side of the dielectric interface
for all panels. By partitioning surfaces into convex subsurfaces, a
complete specification of any dielectric interface can always be made
using multiple files, each with its own reference point.

The optional ``{\tt +}'' argument is used to control conductor naming.
When a {\tt C} or {\tt B} line is terminated with a ``{\tt +},'' any
panels with the same conductor name in that line's {\tt file} and the
next {\tt C} or {\tt B} line's {\tt file} are considered to be part of
the same conductor.  For example, consider the first few lines
from the example list file
{\tt 1x1bus.lst},
\begin{verbatim}
*
* 1x1 bus crossing problem with dielectric on lower conductors
*
* conductor to air interfaces
C cond_air_1x1.qui 1.0 0.0 0.0 0.0 +
*
* conductor to dielectric interfaces
C cond_dielec_1x1.qui 7.5 0.0 0.0 0.0
\end{verbatim}
The two files \verb~cond_air_1x1.qui~ and \verb~cond_dielec_1x1.qui~ represent
the parts of the conductors that border the air and passivation
dielectrics respectively (see Figure~\ref{1x1bus}). 
%
% 1x1 bus showing breakdown into conductor/air, conductor/dielectric
%   and dielectric/dielectric
%
\begin{figure}
\centerline{
\psfig{figure=\figuredir/1x1bus_ca.ps,width=1.9in}\rule{0.1in}{0in}\psfig{figure=\figuredir/1x1bus_cd.ps,width=1.9in}\rule{0.1in}{0in}\psfig{figure=\figuredir/1x1bus_da.ps,width=1.9in}\rule{0.1in}{0in}
}
\caption{The parts of the 1x1 coated bus crossing problem: 
conductor-air interfaces, 
left (from {\tt cond\_air\_1x1.qui}), conductor-passivation
layer interfaces, center (from {\tt cond\_dielec\_1x1.qui}) and
passivation layer-air interfaces, right (from remaining files in 
{\tt 1x1bus.lst}). Axes are three units long.}
\label{1x1bus}
\end{figure}
%
% these things are here so awking with fig2ug.awk (after tex2ptug.awk)
% will generate copies of all the ps files used in the above figure
%%%&^ 1x1bus_ca.ps
%%%&^ 1x1bus_cd.ps
%%%&^ 1x1bus_da.ps
%
Both files contain
panels corresponding to the
lower conductor with conductor name ``{\tt 1}.'' The ``{\tt +}'' 
causes all these
panels to be grouped together and treated as a  single conductor 
named {\tt 1\%GROUP1}, as is required for the calculation.
Without the ``{\tt +}'' {\tt fastcap} would associate the 
``{\tt 1}'' panels in \verb~cond_air_1x1.qui~ with a conductor named
{\tt 1\%GROUP1},
and those in \verb~cond_dielec_1x1.qui~ with another conductor
named {\tt 1\%GROUP2}.

If
files are in a group (that is they are listed consecutively in the list file
and have ``{\tt +}'' symbols linking them together), all the generic
format files must be listed before any PATRAN-generated neutral files. 
This restriction
is necessary due to the ability to rename conductors in generic format files.
Also, {\tt G} lines used to rename groups to be something other than
the default {\tt GROUP<num>}, must occur immediately before the group they
rename.

All the features of the list file interface are further illustrated
by the example list file {\tt test.lst} which is generated by
typing 
\begin{quote}\tt
testgen.sh
\end{quote}
The script {\tt testgen.sh} uses {\tt cubegen} and {\tt busgen} to
create all the input files required by {\tt test.lst}.
Section~\ref{listeg} describes the geometry
corresponding to {\tt test.lst}.


%\end{document}










