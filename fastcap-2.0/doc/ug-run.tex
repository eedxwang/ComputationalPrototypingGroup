%
% how to run Fastcap with examples
%
\section{Running FastCap}

The basic form of the FastCap program command line is
\begin{quote}
\begin{verbatim}
fastcap [-o<expansion order>] [-d<partitioning depth>] [<input file>]
        [-p<permittivity factor>] [-rs<cond list>] [-ri<cond list>]
        [-] [-l<list file>] [-t<iter tol>]
\end{verbatim}
\end{quote}
Other options having to do with producing pictures of the geometry
under analysis are discussed in Section~\ref{genpic}.

Usually only the {\tt input file}, in neutral or generic format,
or a {\tt list file}
(see Section~\ref{all}), is specified.
For example, the command
\begin{quote}\tt
fastcap sphere2.neu
\end{quote}
runs {\tt fastcap} on the example neutral file {\tt sphere2.neu}
described in Section~\ref{sphrun}.
The input file is taken from {\tt stdin} if
{\tt input file} is not given. For example
\begin{quote}\tt
busgen | fastcap
\end{quote}
runs {\tt fastcap} on the $2\times 2$ bus crossing example discussed
in Section~\ref{busxin}.
More complicated geometries,
in particular problems with multiple dielectric regions, are entered
using a {\tt list file} (see Section~\ref{listif}). Thus
\begin{quote}\tt
fastcap -l1x1bus.lst
\end{quote}
runs {\tt fastcap} on the coated $1\times 1$ bus crossing problem mentioned in
Section~\ref{listif}. The ``{\tt -}'' option forces a read from {\tt stdin}
in conjunction with a {\tt list file} or {\tt input file} read. For
example, the capacitance of a cube next to a sphere may be analyzed
using
\begin{quote}\tt
cubegen -xo2 | fastcap - sphere2.neu
\end{quote}
The {\tt cubegen} generic-format file generator is described in
Section~\ref{cubgen}. The output is written to {\tt stdout} except for
the postscript files generated using the options described in
Section~\ref{genpic}.



The program automatically chooses the expansion
order and spatial partitioning depth that maximize the efficiency of
the multipole algorithm and still maintain 1\% accuracy\footnote{The linear
system arising from the discretization is solved with 1\% accuracy. The
discretization error must be controlled by the user through a
sufficiently fine discretization.} 
in the larger capacitance
matrix entries. By default a uniform, free-space dielectric surrounding 
the conductors is assumed.

\subsection{Trading Off Accuracy, Memory Usage and Speed}
\label{trdoff}

The FastCap program's default 1\% accuracy is obtained using second
order multipole expansions and a dynamically set number of partitioning
levels. If lower accuracy is tolerable, faster calculations are possible
by forcing {\tt fastcap} to use  lower order expansions. For example
the command
\begin{quote}\tt
busgen | fastcap -o1 
\end{quote}
runs {\tt fastcap} on the $2\times 2$ bus crossing example 
(see Section~\ref{busxin}) with first order
expansions. 
The iterates are calculated about twice as fast as for the default
case but the capacitance matrix entries will have larger errors, with
the greatest percentage error in the smaller matrix entries. 
Using zero order expansions makes the
iterate calculation about nine times faster but can produce significant
errors, even in the diagonal capacitance matrix entries,
and should not be used when accurate estimate the small
coupling capacitances are needed.
%
%Specifying zero order expansions without setting
%the partitioning depth with the {\tt -d} option forces {\tt fastcap} to
%partition space into cubes containing at most one 
%panel\footnote{In general an expansion order $l$ results in lowest level
%cubes containing $(1+l)^2$ or fewer panels.}. Many partitioning levels are
%required to accomplish this for larger problems, sometimes resulting
%in excessive memory use. In those cases the number of partitioning levels
%should be set using the {\tt -d} option 
%to some number less than the automatically set amount. Using less than 
%the automatically set partitioning depth usually decreases the fraction
%of the iterate multiplications approximated by
%the multipole algorithm. This typically leads to slightly longer run times and
%more accurate results.
When using zero order expansions, the partitioning depth should be set
manually with the {\tt -d} option, otherwise excessive memory may be used.

%{\sc Something about high memory use for \mbox{\tt -o1,0} with auto levels.}

In cases where accuracy greater than 1\% is required, the tolerance
on the iterative loop may be tightened using the {\tt -t} option.
The default value {\tt 0.01} keeps errors to about 1\% of the larger
self capacitances. The command
\begin{quote}\tt
busgen -n4 | fastcap -t0.001 -o3
\end{quote}
results in ten times more accurate solutions to the discretized problem.
Third order expansions are specified with the {\tt -o3} option so that
the accuracy of the iterates is consistent with the
iterative loop tolerance. Also a finer paneling is used
({\tt -n4} option) to bring discretization error down to the level of
the tighter tolerance.

%If calculation time is less important than accuracy, {\tt fastcap} can be
%forced to compute iterates without using the multipole approximation. 
%The command
%\begin{quote}\tt
%busgen | fastcap -d0 
%\end{quote}
%runs {\tt fastcap} on the $2\times 2$ bus crossing example with no
%spatial partitioning,
%causing the multipole algorithm
%to calculate all charge interactions exactly.  The iteration is still
%stopped when 1\% tolerance is reached but the iterates are calculated
%explicitly.
%In practice, the resulting capacitance matrix is only slightly more
%accurate than that obtained using the default parameters, but can take much
%longer to compute, especially for larger problems.

\subsection{Changing the Dielectric Permittivity Factor}

The {\tt -p} option can be used to specify a permittivity factor
other than the default value of {\tt 1.0}.  The permittivity factor
multiplies all the permittivities in the problem.  
The default permittivity is the free-space value
$8.8542\times 10^{-12}$ F/m but
the relative permittivity may be changed\footnote{MKSA units are assumed throughout. In
particular, conductor dimensions are assumed to be in meters.}.
The command line
\begin{quote}\tt
fastcap -p3.5 sphere2.neu
\end{quote}
runs {\tt fastcap} on the spherical capacitor example embedded in
a material with relative permittivity $3.5$. 

The {\tt -p} option is
typically not used with {\tt list file} input since all the permittivities
are scaled by the same factor. For example
\begin{verbatim}
fastcap -p2.0 -lcoated_sph.lst
\end{verbatim}
figures the capacitance of a sphere with a permittivity {\tt 4.0} coating
embedded in a permittivity {\tt 2.0} dielectric medium (see
Section~\ref{listif} for a description of \verb~coated_sph.lst~).

\subsection{Removing Conductors}

The {\tt -rs} and {\tt -ri} options remove conductors from the solve
and input lists respectively.
The {\tt -rs} option is used to specify a list of conductors that are
to be held at zero potential for all the calculations. For example,
the problem of a $2\times 2$ bus crossing over a ground plane can
be analyzed with the commands
\begin{quote}
\begin{verbatim}
busgen > bus.qui
cubegen -p -t -n20 -xo-1 -yo-1 -zo-1 -xh7 -yh7 -naGND \
  | fastcap - bus.qui -rsG
\end{verbatim}
\end{quote}
The ground plane is modeled using a {\tt cubegen}-generated square sheet 
with conductor name {\tt GND}. The {\tt -rsG} argument tells {\tt fastcap}
to hold the conductor {\tt GND} at zero volts for all the computations.
Thus the problem of {\tt GND} at one volt, the four bus wires grounded
need not be solved. In general, if the self capacitance of a conductor
is not required, it can be removed from the solve list using
the {\tt -rs} option. The conductor list argument {\tt <cond list>} is
a comma separated list of conductor names of the form
\begin{quote}\tt
[<name>],[<name>],\ldots,[<name>]
\end{quote}
A {\tt <name>} token need only be enough of the conductor name's leading
characters to specify it uniquely.

The {\tt -ri} option is used exactly like {\tt -rs}. All the panels
corresponding to conductors in
a {\tt -ri} {\tt <cond list>} are completely removed from the list
of input conductors. This option is useful for examining subproblems
without altering the input. For example, the capacitance of two wires
crossing over a single wire can be examined using
\begin{quote}\tt
busgen | fastcap -ri1
\end{quote}



%\subsection{Command Line Arguments}
%
%The first command line argument, the expansion order {\tt l} $\geq 0$, 
%specifies the accuracy of the multipole algorithm.
%Higher order expansions give more accurate results at the expense of
%more computation. It is recommended that
%second order expansions be used
%since the corresponding level of accuracy is in keeping with the
%default 1\% iterative loop tolerance\footnote{To change the tolerance,
%change {\tt ABSTOL} in {\tt mulGlobal.h}.}. If 1\% accuracy is not
%required, zero or first order expansions may be used to decrease computation.
%
%The second argument is
%the number of levels, {\tt L} $\geq 0$, used in the
%hierarchical partitioning of the
%smallest cube enclosing all the conductors.
%The cube is recursively
%divided into eight smaller cubes, giving 8^{\tt L} cubes after 
%{\tt L} partitionings.
%A given partitioning of space has cubes on 
%levels $0$, $1$, \ldots, {\tt L},
%which decrease in size as the level number increases. 
%
%The number of
%levels {\tt L} should be chosen so that all the lowest level cubes contain
%roughly $(1 + \mbox{\tt l})^2$  panels, where {\tt l} 
%is the expansion order specified on the command line.
%Thus for second order  expansions, the chosen number of levels should give
%approximately nine panels in each lowest level cube. 
%Coarser
%partitionings than this lead to somewhat higher accuracy at the expense
%of more computation while finer partitionings do not result in any further
%computational savings. 
%As described in the next section,
%Fastcap provides
%several ways to monitor the effectiveness of the chosen partitioning.
%
%The final command line argument is the name of the neutral file specifying
%the problem geometry. Input file generation is described in Section~\ref{}.
%
%%too finely, too coarsely
%%finer discretization => less done directly
%%rule of thumb based on panel size, structure size
%

\subsection{Calculating Capacitances}

The output produced by the default version of {\tt fastcap} is described
here by way of several examples. As documented in the file {\tt mulGlobal.h},
{\tt fastcap} can be configured to output other information as well.

The first part of the output summarizes the information from the 
command line and
the input file(s). The program then constructs multipole expansions for the
potential  at all levels of the spatial partitioning. The adaptive nature
of the algorithm ensures that only multipole expansions that lead to a
reduction in computation are constructed.
A message is printed
if an entire partitioning level has no multipole or local expansions. With
no options specified, {\tt fastcap} automatically sets the partitioning depth,
often
so that the lowest level is free of expansions. Thus a  message of
the form
\begin{quote}\tt
No expansions at level 4 (lowest)
\end{quote}
is a normal condition. The final line of multipole setup information has
the form
\begin{quote}
\begin{verbatim}
Percentage of multiplies done by multipole: 73.1%
\end{verbatim}
\end{quote}
The number of multiplications done in one {\tt fastcap} iteration for
an $n$ panel problem is 
considerably less than the $n^2$
required for an explicit matrix-vector product. 
In this example $73.1$\% of the $n^2$
multiplications are approximated using the multipole algorithm, implying
a significant reduction in computation. 

With the expansions in place, the program begins iterating for the $m$
columns of the capacitance matrix, where $m$ is the number of
conductors taking into account those removed from the computation
with the {\tt -rs} and {\tt -ri} options. 
When the iterations are complete,
the capacitance matrix entries are printed in MKSA units.

Section~\ref{sphrun} presents 
example input discretizations of the unit sphere in neutral
file format. Other examples are generated by the programs
{\tt cubegen} (Section~\ref{cubgen}), 
{\tt capgen} (Section~\ref{parpla}) and {\tt busgen} (Section~\ref{busxin})
in generic file format.  More complicated examples, which combine
simple geometries generated using
the programs {\tt cubegen}, {\tt pipedgen} and {\tt pyragen} with
neutral file geometries
are given in
Sections~\ref{listeg} through \ref{dram}.
The generic file generators are compiled with {\tt fastcap} and are all
based on the rectangle and triangle
discretization functions  {\tt disRect()} and {\tt disTri()} in the
files {\tt disrect.c} and {\tt distri.c} respectively.

\subsubsection{Spherical Capacitor}
\label{sphrun}

A conducting sphere in free space with one meter radius has capacitance 
$
4\pi\epsilon_0\:\mbox{F} \approx 0.111265\:\mbox{nF}
$,
as may be verified using Gauss's Law. The two example
neutral files {\tt sphere1.neu} and {\tt sphere2.neu} describe the two
discretizations of the unit sphere illustrated in Figure~\ref{spheres}.
A third discretization, finer than that in {\tt sphere2.neu},
is given in {\tt sphere3.neu}.

%
% two spheres next to each other
%
% next two lines exist only for fig2ug.awk to grab the files
%%%&^ sphere1.ps
%%%&^ sphere2.ps
\begin{figure}
\centerline{
\psfig{figure=\figuredir/sphere1.ps,width=2.8in}\rule{0.4in}{0in}\psfig{figure=\figuredir/sphere2.ps,width=2.8in}
}
\caption{The {\tt sphere1.neu} (left) and {\tt sphere2.neu} discretizations of the unit sphere.}
\label{spheres}
\end{figure}

All three discretizations are obtained by meshing a unit sphere
primitive with quadrilaterals using the {\tt ISO} option in {\tt PATRAN}. 
The {\tt sphere1.neu}
case corresponds to $\mbox{\tt ELEM LEN} = \mbox{\tt 0.5}$ 
while the {\tt sphere2.neu} and {\tt sphere3.neu} cases
result when $\mbox{\tt ELEM LEN} = \mbox{\tt 0.2}$ and 
$\mbox{\tt ELEM LEN} = \mbox{\tt 0.15}$
respectively.  
The panels in the  {\tt PATRAN} discretizations are nearly all nonplanar
quadrilaterals. Since {\tt fastcap} requires planar panels, each 
nonplanar quadrilateral
in the input file is automatically broken into two planar triangles, resulting
in the discretizations of Figure~\ref{spheres}.


The command
\begin{quote}\tt
fastcap sphere1.neu
\end{quote}
produces the output below.
\begin{verbatim}
Running fastcap 2.0 (25May92)
  Input: sphere1.neu
  Input surfaces:
   GROUP1
    sphere1.neu, conductor
      title: `SPHERE, RADIUS 1, ELEM LEN = 0.5'
      outer permittivity: 1
      number of panels: 104
      number of extra evaluation points: 0
      translation: (0 0 0)
  Date: Mon May 25 15:22:40 1992
  Host: hilbert

INPUT SUMMARY
  Expansion order: 2
  Number of partitioning levels: 3
  Overall permittivity factor: 1
  Total number of panels: 104
    Number of conductor panels: 104
    Number of dielectric interface panels: 0
    Number of thin conductor on dielectric interface panels: 0
  Number of conductors: 1
No expansions at level 3 (lowest)
No expansions at level 2
Percentage of multiplies done by multipole: 0%
Warning: no multipole acceleration

ITERATION DATA
Starting on column 1 (SPHERE%GROUP1)
1 2 3

CAPACITANCE MATRIX, nanofarads
                        1
SPHERE%GROUP1 1     0.1063
\end{verbatim}
This output illustrates how {\tt fastcap} automatically calculates
iterates explicitly for problems that are
too small to warrant using multipole expansions. For such problems the
multipole approximation of iterates is actually more costly  than the
exact calculation.  In this case the coarseness of the discretization
also leads to 4\% error in the sphere's self-capacitance.  Full rated
accuracy  (default 1\%) can only be realized if the discretization error
is negligible.

This suggests using a finer discretization of the sphere.
The command
\begin{quote}\tt
fastcap sphere2.neu
\end{quote}
produces the output below.
\begin{verbatim}
Running fastcap 2.0 (25May92)
  Input: sphere2.neu
  Input surfaces:
   GROUP1
    sphere2.neu, conductor
      title: `SPHERE, RADIUS 1, ELEM LEN = 0.2'
      outer permittivity: 1
      number of panels: 768
      number of extra evaluation points: 0
      translation: (0 0 0)
  Date: Mon May 25 15:24:37 1992
  Host: hilbert

INPUT SUMMARY
  Expansion order: 2
  Number of partitioning levels: 3
  Overall permittivity factor: 1
  Total number of panels: 768
    Number of conductor panels: 768
    Number of dielectric interface panels: 0
    Number of thin conductor on dielectric interface panels: 0
  Number of conductors: 1
No expansions at level 3 (lowest)
Percentage of multiplies done by multipole: 76.1%

ITERATION DATA
Starting on column 1 (SPHERE%GROUP1)
1 2

CAPACITANCE MATRIX, nanofarads
                        1
SPHERE%GROUP1 1     0.1105
\end{verbatim}
Compared to the analytic value, the capacitance calculated by {\tt fastcap}
now has 0.7\% error.

In practice, however, the lack of analytic capacitance values
requires further refinement of the discretization until the capacitance
values stabilize. 
The command
\begin{quote}\tt
fastcap sphere3.neu
\end{quote}
produces the output below.
\begin{verbatim}
Running ../bin/fastcap 2.0 (25May92)
  Input: sphere3.neu
  Input surfaces:
   GROUP1
    sphere3.neu, conductor
      title: `SPHERE, RADIUS 1, ELEM LEN = 0.15'
      outer permittivity: 1
      number of panels: 1200
      number of extra evaluation points: 0
      translation: (0 0 0)
  Date: Mon May 25 15:32:28 1992
  Host: hilbert

INPUT SUMMARY
  Expansion order: 2
  Number of partitioning levels: 3
  Overall permittivity factor: 1
  Total number of panels: 1200
    Number of conductor panels: 1200
    Number of dielectric interface panels: 0
    Number of thin conductor on dielectric interface panels: 0
  Number of conductors: 1
Percentage of multiplies done by multipole: 77.8%

ITERATION DATA
Starting on column 1 (SPHERE%GROUP1)
1 2

CAPACITANCE MATRIX, nanofarads
                        1
SPHERE%GROUP1 1     0.1108
\end{verbatim}
The finer discretization in {\tt sphere3.neu} contains nearly twice as
many panels as the {\tt sphere2.neu} case, leading to multipole
expansions on the lowest level (level 3).

Compared to the analytic value, the capacitance is correct
to better than 1\%, as for the {\tt sphere2.neu} discretization.
If the analytic capacitance were not known, it would still be possible to
conclude that the sphere's capacitance is within 1\% of $0.1108$nF, since
the discretization refinement changes the capacitance only slightly.

Using a discretization refinement to check the capacitance values is
not as costly using {\tt fastcap} as for other capacitance extraction
programs. Doubling the number of panels when using a
Gaussian elimination based algorithm leads to an eight-fold increase in run
time.
The {\tt fastcap} run time for {\tt sphere3.neu}, however, 
is only about twice that required for {\tt sphere2.neu}.

\subsubsection{Cubic Capacitor}
\label{cubgen}

The capacitance of a unit cube in free space may be investigated using
the {\tt fastcap} input file generator {\tt cubegen}. The
discretization, illustrated on the left in Figure~\ref{cubes}, uses five panels
along each edge, with edge panel widths equal to 10\% of inner panel
widths\footnote{A.\ E.\ Ruehli and P.\ A.\ Brennan,
``Efficient Capacitance Calculations for Three-Dimensional
                Multiconductor Systems,''
{\em IEEE Transactions on Microwave Theory and Techniques}, 21(2):76--82, 
February 1973.}. 
A {\tt fastcap} generic format
input file containing such a discretization is written to {\tt stdout}
using the command
\begin{quote}\tt
cubegen -n5
\end{quote}
The {\tt cubegen} program has several options which are summarized
in Section~\ref{quiref}. 
%
% two cubes next to each other
%
% for fig2ug.awk
%%%&^ cubes.eps
\begin{figure}
\centerline{
\psfig{figure=\figuredir/cubes.eps}
}
\caption{The {\tt cubegen -n5} (left) and {\tt cubegen -n7} discretizations of the unit cube.}
\label{cubes}
\end{figure}

The command
\begin{quote}\tt
cubegen -n5 | fastcap 
\end{quote}
produces the output below.
\begin{verbatim}
Running fastcap 2.0 (25May92)
  Input: stdin
  Input surfaces:
   GROUP1
    stdin, conductor
      title: `1mX1mX1m cube (n=5 e=0.1)'
      outer permittivity: 1
      number of panels: 150
      number of extra evaluation points: 0
      translation: (0 0 0)
  Date: Mon May 25 15:37:48 1992
  Host: hilbert

INPUT SUMMARY
  Expansion order: 2
  Number of partitioning levels: 3
  Overall permittivity factor: 1
  Total number of panels: 150
    Number of conductor panels: 150
    Number of dielectric interface panels: 0
    Number of thin conductor on dielectric interface panels: 0
  Number of conductors: 1
No expansions at level 3 (lowest)
Percentage of multiplies done by multipole: 72.6%

ITERATION DATA
Starting on column 1 (1%GROUP1)
1 2 3 4

CAPACITANCE MATRIX, picofarads
                   1
1%GROUP1 1      73.28
\end{verbatim}
To check the fidelity of the capacitance value, the finer discretization 
pictured on the right in 
Figure~\ref{cubes},
with seven panels along each edge, is used. The
corresponding command
\begin{quote}\tt
cubegen -n7 | fastcap 
\end{quote}
produces the output below.
\begin{verbatim}
Running fastcap 2.0 (25May92)
  Input: stdin
  Input surfaces:
   GROUP1
    stdin, conductor
      title: `1mX1mX1m cube (n=7 e=0.1)'
      outer permittivity: 1
      number of panels: 294
      number of extra evaluation points: 0
      translation: (0 0 0)
  Date: Mon May 25 15:39:15 1992
  Host: hilbert

INPUT SUMMARY
  Expansion order: 2
  Number of partitioning levels: 3
  Overall permittivity factor: 1
  Total number of panels: 294
    Number of conductor panels: 294
    Number of dielectric interface panels: 0
    Number of thin conductor on dielectric interface panels: 0
  Number of conductors: 1
Percentage of multiplies done by multipole: 50%

ITERATION DATA
Starting on column 1 (1%GROUP1)
1 2 3 4 5 6

CAPACITANCE MATRIX, picofarads
                   1
1%GROUP1 1       73.4
\end{verbatim}
The small change in the capacitance value indicates that the self-capacitance
of the cube is $73.4$pF to within 1\%. This compares well with 
other numerical
calculations giving values of $73.5$pF and
$73.4$pF\footnote{M.\ A.\ Jaswon and G.\ T.\ Symm,
{\em Integral Equation Methods in Potential Theory and Elastostatics},
Academic Press, London, 1977.}.

\subsubsection{Parallel Plate Capacitor}
\label{parpla}

%
% two ||-plate capacitors next to each other
%
% for fig2ug.awk
%%%&^ pltcap.eps
\begin{figure}
\centerline{
\psfig{figure=\figuredir/pltcap.eps}
}
\caption{The {\tt capgen -n20} (left) and {\tt capgen -n25} discretizations of a parallel plate capacitor.}
\label{pltcap}
\end{figure}

The capacitance of a parallel plate capacitor 
can be investigated using the {\tt capgen} input file generator.
The command
\begin{quote}\tt
capgen -n20 | fastcap
\end{quote}
runs {\tt fastcap} on the discretization pictured on the left
in Figure~\ref{pltcap}.
The parallel plates are infinitesimally thin, one meter on a side and
$0.1$ meters apart. Each plate is discretized in the same manner as
the cube faces in the previous example, although twenty
panels per edge are used to better approximate the stronger
singularity in the charge density on a thin plate. The {\tt capgen}
command line options are summarized in Section~\ref{quiref}. The output
is given below.
\begin{verbatim}
Running fastcap 2.0 (25May92)
  Input: stdin
  Input surfaces:
   GROUP1
    stdin, conductor
      title: `1mX1m 2 || plate capacitor with 0.1m separation (n=20 e=0.1)'
      outer permittivity: 1
      number of panels: 800
      number of extra evaluation points: 0
      translation: (0 0 0)
  Date: Mon May 25 15:52:45 1992
  Host: hilbert

INPUT SUMMARY
  Expansion order: 2
  Number of partitioning levels: 4
  Overall permittivity factor: 1
  Total number of panels: 800
    Number of conductor panels: 800
    Number of dielectric interface panels: 0
    Number of thin conductor on dielectric interface panels: 0
  Number of conductors: 2
No expansions at level 4 (lowest)
Percentage of multiplies done by multipole: 85.7%

ITERATION DATA
Starting on column 1 (1%GROUP1)
1 2 3
Starting on column 2 (2%GROUP1)
1 2 3 4

CAPACITANCE MATRIX, nanofarads
                   1          2
1%GROUP1 1     0.1265    -0.1038
2%GROUP1 2    -0.1038     0.1265
\end{verbatim}

The finer discretization illustrated  on the right in
Figure~\ref{pltcap} is input
to {\tt fastcap} using the command
\begin{quote}\tt
capgen -n25 | fastcap
\end{quote}
and produces the output below.
\begin{verbatim}
Running fastcap 2.0 (25May92)
  Input: stdin
  Input surfaces:
   GROUP1
    stdin, conductor
      title: `1mX1m 2 || plate capacitor with 0.1m separation (n=25 e=0.1)'
      outer permittivity: 1
      number of panels: 1250
      number of extra evaluation points: 0
      translation: (0 0 0)
  Date: Mon May 25 15:55:06 1992
  Host: hilbert

INPUT SUMMARY
  Expansion order: 2
  Number of partitioning levels: 4
  Overall permittivity factor: 1
  Total number of panels: 1250
    Number of conductor panels: 1250
    Number of dielectric interface panels: 0
    Number of thin conductor on dielectric interface panels: 0
  Number of conductors: 2
No expansions at level 4 (lowest)
Percentage of multiplies done by multipole: 88.9%

ITERATION DATA
Starting on column 1 (1%GROUP1)
1 2 3 4
Starting on column 2 (2%GROUP1)
1 2 3 4

CAPACITANCE MATRIX, nanofarads
                   1          2
1%GROUP1 1     0.1267     -0.104
2%GROUP1 2     -0.104     0.1267
\end{verbatim}
By the same argument used in previous examples, these capacitances
are accurate to 1\%.
%\end{document}
%

The entries in the capacitance matrix indicate that when a
one volt\footnote{All potentials are referenced to the potential at infinity.}
 source 
is connected across the capacitor, $0.127$nC of charge accumulate on
the plate at one volt and $-0.104$nC accumulate on the grounded plate.
A small part of the charge on the positive plate, $0.127-0.104=0.023$nC,
is induced to support stray electric field lines 
terminating at infinity, while the remaining $0.104$nC of charge
is induced by field lines starting on the positive plate
and ending on the grounded plate. 
%Thus the positive plate
%may be thought of as a single conductor
%capacitor with capacitance $0.127$nF whose,
%relatively large charge capacity is due to the close proximity of the
%grounded plate\footnote{The capacitance of a single plate is only $0.041$nF,
%as may be verified with the command {\tt capgen -n25 -p1 | fastcap}}.
If the grounded plate's potential is changed, then an electric field between
it and infinity is produced, giving its charge a stray field
component. Thus, the most accurate model for the structure in an electric
circuit consists of three ideal capacitors: a $0.023$nF capacitor from
each plate terminal to ground to model stray field effects,
and a $0.104$nF capacitor between the two
terminals. In contrast, the model obtained without {\tt fastcap}
is a single capacitor between the terminals with value proportional
to the inverse of the plate separation distance, 
$\epsilon_0/0.1\mbox{F}=0.089$nF. This model completely neglects stray field
effects.

\subsubsection{Bus Crossing Problem}
\label{busxin}

The final simple
example applies {\tt fastcap} to the bus crossing geometry illustrated
in Figure~\ref{busses}. The illustrated discretization is generated
in generic file format by the program {\tt busgen}, 
whose options are summarized
in Section~\ref{quiref}.  Using a common paneling strategy,
%{\tt ref to Ruehli}
{\tt busgen} first breaks the conductors into cubic sections
(see Figure~\ref{sections}) and then breaks the exposed faces of each section
in the same way as {\tt cubegen} (see Section~\ref{cubgen}).
%
% bus crossing problem with fine lines
%
% for fig2ug.awk
%%%&^ busses.eps
\begin{figure}
\centerline{
\psfig{figure=\figuredir/busses.eps,width=4in}
}
\caption{The {\tt busgen -n5} discretization of the $2\times 2$ bus crossing problem.}
\label{busses}
\end{figure}
%
% bus crossing problem with sections indicated
%
% for fig2ug.awk
%%%&^ sections.eps
\begin{figure}
\centerline{
\psfig{figure=\figuredir/sections.eps,width=4in}
}
\caption{The conductors are broken into sections based on where they overlap.}
\label{sections}
\end{figure}

The discretization of Figure~\ref{busses} is input to {\tt fastcap}
with the command
\begin{quote}\tt
busgen -n5 | fastcap
\end{quote}
and produces the output below.
\begin{verbatim}
Running fastcap 2.0 (25May92)
  Input: stdin
  Input surfaces:
   GROUP1
    stdin, conductor
      title: `2X2 bus crossing problem with 1m wires (n=5 e=0.1)'
      outer permittivity: 1
      number of panels: 2200
      number of extra evaluation points: 0
      translation: (0 0 0)
  Date: Mon May 25 15:57:35 1992
  Host: hilbert

INPUT SUMMARY
  Expansion order: 2
  Number of partitioning levels: 4
  Overall permittivity factor: 1
  Total number of panels: 2200
    Number of conductor panels: 2200
    Number of dielectric interface panels: 0
    Number of thin conductor on dielectric interface panels: 0
  Number of conductors: 4
Percentage of multiplies done by multipole: 89.3%

ITERATION DATA
Starting on column 1 (1%GROUP1)
1 2 3 4 5 6
Starting on column 2 (2%GROUP1)
1 2 3 4 5 6
Starting on column 3 (3%GROUP1)
1 2 3 4 5 6
Starting on column 4 (4%GROUP1)
1 2 3 4 5 6

CAPACITANCE MATRIX, picofarads
                   1          2          3          4
1%GROUP1 1      247.8     -85.01     -48.53     -48.53
2%GROUP1 2     -85.01      247.8     -48.53     -48.54
3%GROUP1 3     -48.53     -48.53      247.9     -84.99
4%GROUP1 4     -48.53     -48.54     -84.99      247.9
\end{verbatim}
To check the capacitance values, the discretization of Figure~\ref{busses}
is refined so that each section has seven panels on a side.
The new discretization is input to {\tt fastcap} using
\begin{quote}\tt
busgen -n7 | fastcap
\end{quote}
which gives the output below.
\begin{verbatim}
Running fastcap 2.0 (25May92)
  Input: stdin
  Input surfaces:
   GROUP1
    stdin, conductor
      title: `2X2 bus crossing problem with 1m wires (n=7 e=0.1)'
      outer permittivity: 1
      number of panels: 4312
      number of extra evaluation points: 0
      translation: (0 0 0)
  Date: Mon May 25 16:05:35 1992
  Host: leibnitz

INPUT SUMMARY
  Expansion order: 2
  Number of partitioning levels: 5
  Overall permittivity factor: 1
  Total number of panels: 4312
    Number of conductor panels: 4312
    Number of dielectric interface panels: 0
    Number of thin conductor on dielectric interface panels: 0
  Number of conductors: 4
Percentage of multiplies done by multipole: 95.3%

ITERATION DATA
Starting on column 1 (1%GROUP1)
1 2 3 4 5 6 7 8 9 10 11
Starting on column 2 (2%GROUP1)
1 2 3 4 5 6 7 8 9 10 11
Starting on column 3 (3%GROUP1)
1 2 3 4 5 6 7 8 9 10 11
Starting on column 4 (4%GROUP1)
1 2 3 4 5 6 7 8 9 10 11

CAPACITANCE MATRIX, picofarads
                   1          2          3          4
1%GROUP1 1      248.2     -85.18     -48.61     -48.61
2%GROUP1 2     -85.18      248.1     -48.61      -48.6
3%GROUP1 3     -48.61     -48.61      248.3      -85.2
4%GROUP1 4     -48.61      -48.6      -85.2      248.2
\end{verbatim}
The small change in the capacitances after the refinement of the 
discretization indicates that these values are accurate to 1\%.

\subsubsection{Coated Spherical Capacitor}

The list file \verb#coated_sph.lst# described in Section~\ref{listif}
represents
a unit radius sphere coated with a unit thickness dielectric as
illustrated in Figure~\ref{cut_sph}.
If the coating has relative permittivity 2 and the surrounding
region is free space, the structure has capacitance
$148.35$pF by Gauss's Law. The command
\begin{quote}
\verb#fastcap -lcoated_sph.lst#
\end{quote}
generates the output
\begin{verbatim}
Running fastcap 2.0 (25May92)
  Input: coated_sph.lst
  Input surfaces:
   GROUP1
    sphere2.neu, conductor
      title: `SPHERE, RADIUS 1, ELEM LEN = 0.2'
      outer permittivity: 2
      number of panels: 768
      number of extra evaluation points: 0
      translation: (0 0 0)
   GROUP2
    big_sphere1.neu, dielectric interface
      title: `SPHERE, RADIUS 2, ELEM LEN = 0.5'
      permittivities: 2 (inner) 1 (outer)
      number of panels: 432
      number of extra evaluation points: 864
      translation: (0 0 0)
  Date: Mon May 25 16:02:13 1992
  Host: hilbert

INPUT SUMMARY
  Expansion order: 2
  Number of partitioning levels: 4
  Overall permittivity factor: 1
  Total number of panels: 1200
    Number of conductor panels: 768
    Number of dielectric interface panels: 432
    Number of thin conductor on dielectric interface panels: 0
  Number of conductors: 1
No expansions at level 4 (lowest)
Percentage of multiplies done by multipole: 90.7%

ITERATION DATA
Starting on column 1 (SPHERE%GROUP1)
1 2 3

CAPACITANCE MATRIX, nanofarads
                        1
SPHERE%GROUP1 1     0.1484
\end{verbatim}
The computed capacitance is well within $1\%$ of the analytic value.

\subsubsection{A Simple List File Example}
\label{listeg}

A simple example that uses a list file is generated with the command
\begin{quote}\tt
testgen.sh
\end{quote}
The script {\tt testgen.sh} uses the generic format
input file generators {\tt cubegen}, {\tt busgen}
and {\tt pyragen} to create input files that are
linked together by a list file {\tt test.lst}, 
\begin{verbatim}
*
* test geometry to show how interface works
* - a pyramid, an L-shape and a 1x1 bus xing btwn gnd planes
* - two dielectric slabs between ground planes
*
**   pyramid (single conductor in its own file)
C ./pyramid.qui 1  0 3 0
**   1x1 bus crossing (single file with multiple conductors)
G BUS
C ./bus.qui 1  0 0 0
**   L-shaped conductor (single object built from 3 separate files)
C ./lcntr.qui 1  0 0 0 +
C ./lleft.qui 1  1 0 0 +
C ./lright.qui 1  0 1 0
**   ground planes (use of same file repeatedly)
C ./plane.qui 1  -1 -1 -1 +
C ./plane.qui 3  -1 -1 5
**   dielectric interface (same file as ground planes)
D ./plane.qui 3 1  -1 -1 4  0 0 5
D ./skirt.qui 1 3  -1 -1 4  0 0 4.5 -
\end{verbatim}
The problem consists of a pyramid next to an L-shaped conductor and
a $1\times 1$ bus crossing. The three conductors are between two horizontal
ground planes. The upper ground plane has a dielectric coating
with relative permittivity 3. The geometry is illustrated in 
Figure~\ref{testpic}.
%
% test problem
%
% for fig2ug.awk
%%%&^ testpic.ps
\begin{figure}
\centerline{
\psfig{figure=\figuredir/testpic.ps,width=6.0in}
}
\caption{The simple test problem.}
\label{testpic}
\end{figure}


The command 
\begin{quote}\tt
fastcap -ltest.lst
\end{quote}
produces the output
\begin{verbatim}
Running fastcap 2.0 (25May92)
  Input: ./test.lst
  Input surfaces:
   GROUP1
    pyramid.qui, conductor
      title: `1mX1mX1m pyramid (n=3 e=0.1)'
      outer permittivity: 1
      number of panels: 37
      number of extra evaluation points: 0
      translation: (0 3 0)
   BUS
    bus.qui, conductor
      title: `1X1 bus crossing problem with 1m wires (n=3 e=0.1)'
      outer permittivity: 1
      number of panels: 252
      number of extra evaluation points: 0
      translation: (0 0 0)
   GROUP3
    lcntr.qui, conductor
      title: `1mX1mX1m cube (n=3 e=0.1)'
      outer permittivity: 1
      number of panels: 36
      number of extra evaluation points: 0
      translation: (0 0 0)
    lleft.qui, conductor
      title: `1mX1mX1m cube (n=3 e=0.1)'
      outer permittivity: 1
      number of panels: 45
      number of extra evaluation points: 0
      translation: (1 0 0)
    lright.qui, conductor
      title: `1mX1mX1m cube (n=3 e=0.1)'
      outer permittivity: 1
      number of panels: 45
      number of extra evaluation points: 0
      translation: (0 1 0)
   GROUP4
    plane.qui, conductor
      title: `7mX6mX1m cube (n=10 e=0)'
      outer permittivity: 1
      number of panels: 110
      number of extra evaluation points: 0
      translation: (-1 -1 -1)
    plane.qui, conductor
      title: `7mX6mX1m cube (n=10 e=0)'
      outer permittivity: 3
      number of panels: 110
      number of extra evaluation points: 0
      translation: (-1 -1 5)
   GROUP5
    plane.qui, dielectric interface
      title: `7mX6mX1m cube (n=10 e=0)'
      permittivities: 1 (inner) 3 (outer)
      number of panels: 110
      number of extra evaluation points: 220
      translation: (-1 -1 4)
   GROUP6
    skirt.qui, dielectric interface
      title: `7mX6mX1m cube (n=2 e=0)'
      permittivities: 3 (inner) 1 (outer)
      number of panels: 104
      number of extra evaluation points: 208
      translation: (-1 -1 4)
  Date: Mon May 25 16:38:47 1992
  Host: hilbert

INPUT SUMMARY
  Expansion order: 2
  Number of partitioning levels: 4
  Overall permittivity factor: 1
  Total number of panels: 849
    Number of conductor panels: 635
    Number of dielectric interface panels: 214
    Number of thin conductor on dielectric interface panels: 0
  Number of conductors: 5
No expansions at level 4 (lowest)
Percentage of multiplies done by multipole: 83.2%

ITERATION DATA
Starting on column 1 (PYRAMID%GROUP1)
1 2 3 4 5 6 7 8
Starting on column 2 (1%BUS)
1 2 3 4 5 6 7 8
Starting on column 3 (2%BUS)
1 2 3 4 5 6 7 8 9
Starting on column 4 (LSHAPE%GROUP3)
1 2 3 4 5 6 7 8 9
Starting on column 5 (GND%GROUP4)
1 2 3 4 5 6 7 8 9

CAPACITANCE MATRIX, picofarads
                         1          2          3          4          5
PYRAMID%GROUP1 1      67.54     -4.323      -3.63      -11.3     -43.27
1%BUS          2     -4.323      189.5     -44.44     -17.04       -114
2%BUS          3      -3.63     -44.44      177.3     -11.73     -108.2
LSHAPE%GROUP3  4      -11.3     -17.04     -11.73      171.5     -117.6
GND%GROUP4     5     -43.27       -114     -108.2     -117.6      960.4
\end{verbatim}
This example illustrates how the ``{\tt +}'' at the end of a 
list file line can be used to group separated files together. In this
case both {\tt GROUP3} and {\tt GROUP4} are formed this way, producing
the conductors {\tt LSHAPE\%GROUP3} and {\tt GND\%GROUP4} respectively.
In general all panels with the same conductor name in any group are
considered to be part of the same conductor. Although this
example does not show this, a group may contain
more than one conductor.

The group name is always appended to the conductor name read from the
input to guarantee that {\tt fastcap}'s internal names are unique. 
Here the second group has been renamed {\tt BUS}, rather than
the default {\tt GROUP2}, using the {\tt G} line in the list file.
It is the user's responsibility to rename groups in such a way
that all the composite names are unique.

The script {\tt testrun.sh} gives further examples using this problem.

\subsubsection{Backplane Connector}

The capacitances associated with several sizes of backplane conductors
may be investigated using the shell scripts {\tt connector2.sh},
{\tt connector3.sh} and {\tt connector4.sh}\footnote{
The $4\times 4$ conductor connector is discussed in more detail
in K.\ Nabors and J.\ White,
   ``Multipole-accelerated 3-D capacitance extraction
algorithms for structures with conformal
dielectrics.'' {\em Proceedings of the 29th ACM/IEEE
Design Automation Conference}, June 1992.
}.  
Each script generates
a set of generic format files using {\tt cubegen} and {\tt pyragen}
together with a list file that links the surface files  together to produce
the final model.  The geometries analyzed are pictured in Figure~\ref{connect}.
%
% the three connector pictures
%
% for fig2ug.awk
%%%&^ connector2.ps
%%%&^ connector3.ps
%%%&^ connector4.ps
\begin{figure}
\centerline{
\psfig{figure=\figuredir/connector2.ps,width=1.9in}\rule{0.1in}{0in}\psfig{figure=\figuredir/connector3.ps,width=1.9in}\rule{0.1in}{0in}\psfig{figure=\figuredir/connector4.ps,width=1.9in}
}
\caption{The three backplane connector problems generated by the scripts
{\tt connector2.sh} (left), {\tt connector3.sh} (center) and
{\tt connector4.sh} (right). The actual discretizations used are
finer than those pictured.}
\label{connect}
\end{figure}
To analyze the smallest connector, first generate the discretization
using the command
\begin{quote} \tt
connector2.sh 3.5 -n3
\end{quote}
The first argument specifies relative permittivity 3.5 for the U-shaped
plastic connector body. The second argument is passed directly to
{\tt cubegen} and {\tt pyragen} and sets the fineness of the discretization
(see also Section~\ref{quiref} for more on {\tt cubegen} and {\tt pyragen}
options). With the discretization generated, run {\tt fastcap} with
the command
\begin{quote}\tt
fastcap -lconnector23.5-n3.lst
\end{quote}
which gives the output
\begin{verbatim}
Running fastcap 2.0 (25May92)
  Input: connector23.5-d.lst
  Input surfaces:
   GROUP1
    pin_end10.00.00.03.52-d.qui, conductor
      title: `1.3mX1.3mX1m pyramid (n=3 e=0.1)'
      outer permittivity: 1
      number of panels: 4
      number of extra evaluation points: 0
      translation: (0 0 0)
    pin_long0.00.00.03.52-d.qui, conductor
      title: `1.3mX1.3mX15m cube (n=3 e=0.1)'
      outer permittivity: 1
      number of panels: 4
      number of extra evaluation points: 0
      translation: (0 0 0)
    pin_sleeve0.00.00.03.52-d.qui, conductor
      title: `1.3mX1.3mX5.2m cube (n=3 e=0.1)'
      outer permittivity: 3.5
      number of panels: 4
      number of extra evaluation points: 0
      translation: (0 0 0)
    pin_short0.00.00.03.52-d.qui, conductor
      title: `1.3mX1.3mX7m cube (n=3 e=0.1)'
      outer permittivity: 1
      number of panels: 4
      number of extra evaluation points: 0
      translation: (0 0 0)
    pin_end20.00.00.03.52-d.qui, conductor
      title: `1.3mX1.3mX-1m pyramid (n=3 e=0.1)'
      outer permittivity: 1
      number of panels: 4
      number of extra evaluation points: 0
      translation: (0 0 0)
\end{verbatim}
\noindent (Some surface information removed for brevity.)
\begin{verbatim}
   GROUP25
    back23.5-d.qui, dielectric interface
      title: `13mX13mX5.2m cube (n=4 e=0)'
      permittivities: 3.5 (inner) 1 (outer)
      number of panels: 2
      number of extra evaluation points: 4
      translation: (0 0 0)
  Date: Tue May 26 01:02:47 1992
  Host: hilbert

INPUT SUMMARY
  Expansion order: 2
  Number of partitioning levels: 3
  Overall permittivity factor: 1
  Total number of panels: 132
    Number of conductor panels: 80
    Number of dielectric interface panels: 52
    Number of thin conductor on dielectric interface panels: 0
  Number of conductors: 4
No multipole expansions at level 3 (lowest)
Percentage of multiplies done by multipole: 64.9%

ITERATION DATA
Starting on column 1 (1%GROUP1)
1 2 3 4 5 
Starting on column 2 (1%GROUP6)
1 2 3 4 5 
Starting on column 3 (1%GROUP11)
1 2 3 4 5 
Starting on column 4 (1%GROUP16)
1 2 3 4 5 

CAPACITANCE MATRIX, nanofarads
                    1          2          3          4 
1%GROUP1  1     0.8842    -0.2647    -0.2519    -0.1014
1%GROUP6  2    -0.2647     0.8825    -0.1005    -0.2521
1%GROUP11 3    -0.2519    -0.1005     0.8825     -0.265
1%GROUP16 4    -0.1014    -0.2521     -0.265     0.8818
\end{verbatim}
This output further illustrates how {\tt fastcap} groups parts of conductors
that are chained together in the list file by means of the {\tt +}
option (see Section~\ref{listif}). In this particular case, groups
1, 6, 11 and 16, consisting of panels which are all associated with
the conductor name ``{\tt 1}''
in the input files, have their corresponding conductors renamed
to {\tt 1\%GROUP1}, {\tt 1\%GROUP6}, {\tt 1\%GROUP11} and
{\tt 1\%GROUP16} to avoid any ambiguity. 

The input file dimensions are assumed to be in meters, giving
capacitances in the nanofarad range. A more realistic connector
would have dimensions on the order of millimeters, meaning that
the reported capacitances should be interpreted as picofarads.


The example problems
for the larger conductors are generated and run in an analogous
fashion.

\subsubsection{DRAM Cells}
\label{dram}

The final example uses the shell script {\tt ramgen.sh} to generate
a list file that combines generic format files and neutral files
to produce a discretization of three adjoining DRAM cells as
illustrated in Figure~\ref{ramcell}\footnote{
This example is discussed in more detail in
K.\ Nabors and J.\ White,
`` An improved approach to including conformal dielectrics in
        multipole-accelerated three-dimensional capacitance extraction.''
To appear in {\em Proceedings of the Workshop on Numerical Modeling
of Processes and Devices for Integrated Circuits: NUPAD IV}, May-June
1992.
}. 
%
% ramcell description and figure callouts
%
%
% dram cell figure
%
% for fig2ug.awk
%%%&^ side_view.ps
%%%&^ nosini.ps
%\documentstyle{article}
%%
% psfigTeX macros: combined TeX/LaTeX source
%
% Previous copyright below applies.  University of Maryland changes are
% Copyright (c) 1989 University of Maryland
% Department of Computer Science.  All rights reserved.
% Permission to copy these changes for any purpose is hereby granted
% so long as this copyright notice remains intact.
%
% psfig.tex and psfig.sty are derived automatically from this file,
% by including the appropriate one of either the TeX or LaTeX specific
% lines (marked with `%tex:' and `%latex').
%
% -------------------------------------------
%
% All software, documentation, and related files in this distribution of
% psfig/tex are Copyright (c) 1987 Trevor J. Darrell
%
% Permission is granted for use and non-profit distribution of psfig/tex
% providing that this notice be clearly maintained, but the right to
% distribute any portion of psfig/tex for profit or as part of any commercial
% product is specifically reserved for the author.
%
% Psfig/tex version 1.1
%
% $Header: /usr/src/local/tex/local/mctex/psfig/RCS/psfig.src,v 3.1 89/08/30 03:42:01 chris Exp $
% based on: psfig.tex,v 1.8 87/07/25 13:21:09 trevor Exp
%

%
%
%

\typeout{psfig: version 1.1 (MC-TeX)}

\newdimen\ps@dim

%\def\psglobal#1{\ps@typeout
%  {psfig: including #1 globally}\immediate\special{ps:plotfile #1 global}}
%\def\psfiginit{\ps@typeout{psfiginit}\psglobal{/usr/lib/ps/figtex.pro}}

%
%
%
%
%
%
%
%
%
%
%
%
%
%
%
%

% Latex is not supposed to change any more, so we can use its innards.
\let\@psdo=\@for

\def\psdraft{\def\@psdraft{0}}
\def\psfull{\def\@psdraft{100}}
\psfull

\def\ps@eat#1{}
\def\pssilent{\let\ps@typeout\ps@eat}
\def\psverbose{\let\ps@typeout\typeout}
\psverbose

\newif\if@prologfile
\newif\if@postlogfile
%%% These are for the option list.
%%% A specification of the form a = b maps to calling \@p@@sa{b}
\newif\if@bbllx
\newif\if@bblly
\newif\if@bburx
\newif\if@bbury
\newif\if@height
\newif\if@width
\newif\if@rheight
\newif\if@rwidth
\newif\if@clip
\def\@p@@sclip#1{\@cliptrue}
\def\@p@@sfile#1{\def\@p@sfile{#1}}
\def\@p@@sfigure#1{\def\@p@sfile{#1}}
\def\@p@@sbbllx#1{%\ps@typeout{bbllx is #1}
	\@bbllxtrue
	\ps@dim=#1
	\edef\@p@sbbllx{\number\ps@dim}
}
\def\@p@@sbblly#1{
	%\ps@typeout{bblly is #1}
	\@bbllytrue
	\ps@dim=#1
	\edef\@p@sbblly{\number\ps@dim}
}
\def\@p@@sbburx#1{
	%\ps@typeout{bburx is #1}
	\@bburxtrue
	\ps@dim=#1
	\edef\@p@sbburx{\number\ps@dim}
}
\def\@p@@sbbury#1{
	%\ps@typeout{bbury is #1}
	\@bburytrue
	\ps@dim=#1
	\edef\@p@sbbury{\number\ps@dim}
}
\def\@p@@sheight#1{
	\@heighttrue
	\ps@dim=#1
	\edef\@p@sheight{\number\ps@dim}
	%\ps@typeout{Height is \@p@sheight}
}
\def\@p@@swidth#1{
	%\ps@typeout{Width is #1}
	\@widthtrue
	\ps@dim=#1
	\edef\@p@swidth{\number\ps@dim}
}
\def\@p@@srheight#1{
	%\ps@typeout{Reserved height is #1}
	\@rheighttrue
	\ps@dim=#1
	\edef\@p@srheight{\number\ps@dim}
}
\def\@p@@srwidth#1{
	%\ps@typeout{Reserved width is #1}
	\@rwidthtrue
	\ps@dim=#1
	\edef\@p@srwidth{\number\ps@dim}
}
\def\@p@@sprolog#1{\@prologfiletrue\def\@prologfileval{#1}}
\def\@p@@spostlog#1{\@postlogfiletrue\def\@postlogfileval{#1}}
\def\@p@@ssilent#1{\pssilent}

\def\@cs@name#1{\csname #1\endcsname}
\def\@setparms#1=#2,{\@cs@name{@p@@s#1}{#2}}
%
% initialize the defaults (size the size of the figure)
%
\def\ps@init@parms{
	\@bbllxfalse \@bbllyfalse
	\@bburxfalse \@bburyfalse
	\@heightfalse \@widthfalse
	\@rheightfalse \@rwidthfalse
	\def\@p@sbbllx{}\def\@p@sbblly{}
	\def\@p@sbburx{}\def\@p@sbbury{}
	\def\@p@sheight{}\def\@p@swidth{}
	\def\@p@srheight{}\def\@p@srwidth{}
	\def\@p@sfile{}
	\def\@p@scost{10}
	\def\@sc{}
	\@prologfilefalse
	\@postlogfilefalse
	\@clipfalse
}
%
% Go through the options setting things up.
%
\def\parse@ps@parms#1{\@psdo\@psfiga:=#1\do{\expandafter\@setparms\@psfiga,}}

%
% Compute bb height and width
%
\newif\ifno@bb
\newif\ifnot@eof
\newread\ps@stream
\def\bb@missing{
	\ps@typeout{psfig: searching \@p@sfile \space  for bounding box}
	\openin\ps@stream=\@p@sfile
	\no@bbtrue
	\not@eoftrue
	\catcode`\%=12
	\loop
		\read\ps@stream to \line@in
		\global\toks200=\expandafter{\line@in}
		\ifeof\ps@stream \not@eoffalse \fi
		%\ps@typeout{looking at :: \the\toks200}
		\@bbtest{\toks200}
		\if@bbmatch\not@eoffalse\expandafter\bb@cull\the\toks200\fi
	\ifnot@eof \repeat
	\catcode`\%=14
}
\catcode`\%=12
\newif\if@bbmatch
\def\@bbtest#1{\expandafter\@a@\the#1%%BoundingBox:\@bbtest\@a@}
\long\def\@a@#1%%BoundingBox:#2#3\@a@{\ifx\@bbtest#2\@bbmatchfalse\else\@bbmatchtrue\fi}
\long\def\bb@cull#1 #2 #3 #4 #5 {
	\ps@dim=#2 bp\edef\@p@sbbllx{\number\ps@dim}
	\ps@dim=#3 bp\edef\@p@sbblly{\number\ps@dim}
	\ps@dim=#4 bp\edef\@p@sbburx{\number\ps@dim}
	\ps@dim=#5 bp\edef\@p@sbbury{\number\ps@dim}
	\no@bbfalse
}
\catcode`\%=14
%
\def\compute@bb{
	\no@bbfalse
	\if@bbllx \else \no@bbtrue \fi
	\if@bblly \else \no@bbtrue \fi
	\if@bburx \else \no@bbtrue \fi
	\if@bbury \else \no@bbtrue \fi
	\ifno@bb \bb@missing \fi
	\ifno@bb \typeout{FATAL ERROR: no bb supplied or found}
		\no-bb-error
	\fi
	%
	\count203=\@p@sbburx
	\count204=\@p@sbbury
	\advance\count203 by -\@p@sbbllx
	\advance\count204 by -\@p@sbblly
	\edef\@bbw{\number\count203}
	\edef\@bbh{\number\count204}
	%\ps@typeout{bbh = \@bbh, bbw = \@bbw}
}
%
% \in@hundreds performs #1 * (#2 / #3) correct to the hundreds,
%	then leaves the result in @result
%
\def\in@hundreds#1#2#3{\count240=#2 \count241=#3
	\count100=\count240	% 100 is first digit #2/#3
	\divide\count100 by \count241
	\count101=\count100
	\multiply\count101 by \count241
	\advance\count240 by -\count101
	\multiply\count240 by 10
	\count101=\count240	%101 is second digit of #2/#3
	\divide\count101 by \count241
	\count102=\count101
	\multiply\count102 by \count241
	\advance\count240 by -\count102
	\multiply\count240 by 10
	\count102=\count240	% 102 is the third digit
	\divide\count102 by \count241
	\count200=#1\count205=0
	\count201=\count200
		\multiply\count201 by \count100
	 	\advance\count205 by \count201
	\count201=\count200
		\divide\count201 by 10
		\multiply\count201 by \count101
		\advance\count205 by \count201
		%
	\count201=\count200
		\divide\count201 by 100
		\multiply\count201 by \count102
		\advance\count205 by \count201
		%
	\edef\@result{\number\count205}
}
\def\compute@wfromh{
	% computing : width = height * (bbw / bbh)
	\in@hundreds{\@p@sheight}{\@bbw}{\@bbh}
	%\ps@typeout{ \@p@sheight * \@bbw / \@bbh, = \@result }
	\edef\@p@swidth{\@result}
	%\ps@typeout{w from h: width is \@p@swidth}
}
\def\compute@hfromw{
	% computing : height = width * (bbh / bbw)
	\in@hundreds{\@p@swidth}{\@bbh}{\@bbw}
	%\ps@typeout{ \@p@swidth * \@bbh / \@bbw = \@result }
	\edef\@p@sheight{\@result}
	%\ps@typeout{h from w: height is \@p@sheight}
}
\def\compute@handw{
	\if@height \if@width \else \compute@wfromh \fi
	\else \if@width \compute@hfromw \else
		\edef\@p@sheight{\@bbh}
		\edef\@p@swidth{\@bbw} \fi
	\fi
}
\def\compute@resv{
	\if@rheight \else \edef\@p@srheight{\@p@sheight} \fi
	\if@rwidth \else \edef\@p@srwidth{\@p@swidth} \fi
}
%
% Compute any missing values
\def\compute@sizes{
	\compute@bb
	\compute@handw
	\compute@resv
}
%
% \psfig
% usage : \psfig{file=, height=, width=, bbllx=, bblly=, bburx=, bbury=,
%			rheight=, rwidth=, clip=}
%
% "clip=" is a switch and takes no value, but the `=' must be preset.
\def\psfig#1{\vbox{
    \ps@init@parms
    \parse@ps@parms{#1}
    \compute@sizes
    %
    \ifnum\@p@scost<\@psdraft
	\ps@typeout{psfig: including \@p@sfile}
	%
	\special{ps::[begin] \@p@swidth \space \@p@sheight \space
		\@p@sbbllx \space \@p@sbblly \space
		\@p@sbburx \space \@p@sbbury \space
		startTexFig \space}
	\if@clip
		\ps@typeout{(clip)}
		\special{ps:: \@p@sbbllx \space \@p@sbblly \space
			\@p@sbburx \space \@p@sbbury \space
			doclip \space}
	\fi
	\if@prologfile \special{ps: plotfile \@prologfileval \space} \fi
	\special{ps: plotfile \@p@sfile \space}
	\if@postlogfile \special{ps: plotfile \@postlogfileval \space} \fi
	\special{ps::[end] endTexFig \space}
	% Create the vbox to reserve the space for the figure
	\vbox to\@p@srheight true sp{\hbox to\@p@srwidth true sp{\hfil}\vfil}
    \else
	% draft figure, just reserve the space and print the
	% path name.
	\vbox to\@p@srheight true sp{\vss
	    \hbox to\@p@srwidth true sp{\hss\@p@sfile\hss}\vss}
    \fi
}}

%


%\begin{document}
\newcommand{\ramcellcap}{The complete DRAM model, (a), and with the dielectric interfaces removed for clarity, (b).}
\setlength{\unitlength}{1.0in}
\begin{figure}
\vbox{
\centerline{
\psfig{figure=\figuredir/side_view.ps,width=6.5in}
}
\vspace{0.25in}
\centerline{
\psfig{figure=\figuredir/nosini.ps,width=6.5in}
}
}
\begin{picture}(6.4,0)
\thicklines
\put(2.35,0.0){(b)}
\put(2.35,4){(a)}
\put(0.2,0.2){\begin{tabular}{l} vias and \\ depletion regions\end{tabular}}
\put(3.1,0.1){word lines}
\put(4.4,0.1){cell plate}
\put(5.2,3.3){bit lines}
\put(5.5,6.5){\begin{tabular}{l}dielectric \\ interfaces\end{tabular}}
% via and depletion... 
\put(1.1,0.25){\vector(1,2){0.1}}
% word lines
\put(3.0,0.15){\vector(-1,1){0.4}}
\put(3.3,0.25){\vector(-1,2){0.7}}
\put(3.5,0.25){\vector(3,4){1.2}}
\put(3.9,0.15){\vector(1,1){0.7}}
% cell plate
\put(5.1,0.15){\vector(1,1){0.25}}
% bit lines
\put(5.3,3.2){\vector(0,-1){0.4}}
\put(5.5,3.2){\vector(0,-1){0.85}}
\put(5.7,3.2){\vector(0,-1){1.5}}
% dielectric interfaces
\put(5.8,6.35){\vector(0,-1){0.15}}
\put(6.05,6.35){\vector(0,-1){0.55}}
\end{picture}
\caption{\ramcellcap}
\label{ramcell}
\end{figure}
%\end{document}

The 
structure is a simplified model
of three adjacent DRAM cells.
Each cell consists of a polycide
bit line running across the cell, terminating
in a via which connects to the drain of a transistor formed by
a polycide line crossing the substrate at right angles to the
bit line. The source of the transistor
is connected to a  wide polycide plate through a heavily doped
region to enhance capacitance. Three more word lines pass through the
cells: a polycide line, and two
aluminum lines above that line.
The aluminum word lines are covered with
a silicon nitride (permittivity 7.0) passivation layer, as represented
by the two dielectric interfaces illustrated in 
Figure~\ref{ramcell}a.  The material above the top interface is air
while below the aluminum word lines silicon dioxide
(permittivity 3.9) is assumed.
In Figure~\ref{ramcell}b, the dielectric interfaces are removed to show
the conductors more clearly.


The command 
\begin{quote}\tt
ramgen.sh
\end{quote}
produces a discretization 
summarized in the list file {\tt ramcell.lst}. 
The 
script calls the generic file generator programs {\tt pipedgen}
and {\tt cubegen} whose options are itemized in
Section~\ref{quiref}. Typing
\begin{quote}\tt
fastcap -lramcell.lst
\end{quote}
produces the output below given in the file {\tt ramcell.out}. 

%The argument {\tt -rsGND} causes the
%conductor \verb$GND_PLANE%GROUP15$, the ground plane, to be held
%at zero volts for all the calculations. Thus the column of the capacitance
%matrix corresponding to the ground plane is not calculated explicitly,
%although {\tt fastcap} reports ground-plane coupling capacitances that
%are calculated by solving for the other columns.

As in the connector case the input scale is meters, while the true
scale is much smaller.  The DRAM cell dimensions are on the order
of microns, so the reported values must be scaled by $10^{-6}$ to
obtain the actual capacitances of the DRAM geometry.

\subsection{Producing Pictures}
\label{genpic}

The complete command line format of the FastCap program is
\begin{verbatim}
fastcap [-o<expansion order>] [-d<partitioning depth>] [<input file>]
        [-p<permittivity factor>] [-rs<cond list>] [-ri<cond list>]
        [-] [-l<list file>] [-t<iter tol>] [-a<azimuth>] [-e<elevation>]
        [-r<rotation>] [-h<distance>] [-s<scale>] [-w<linewidth>]
        [-u<upaxis>] [-q<cond list>] [-rc<cond list>] [-x<axeslength>]
        [-b<.figfile>] [-m] [-rk] [-rd] [-dc] [-c] [-v] [-n] [-f] [-g]
\end{verbatim}
The first nine basic options 
are described in previous sections. The remaining options are used
to produce postscript format pictures of the geometry under analysis.
Section~\ref{quiref} itemizes the options and this section gives
several examples of their use.
The pictures are two-dimensional renderings of the three-dimensional
objects under analysis, and may be line drawings or shaded plots
indicating the charge density or total charge on each panel.  The
later plots are useful for gauging the effectiveness of a particular
discretization. 

\subsubsection{Line Drawings}

When {\tt fastcap} is given the {\tt -m} option it changes from
calculation to line-drawing mode.  Usually the default settings
produce an acceptable plot for checking
a discretization, but many adjustments are possible,
the most important being the  view angles (relative to a coordinate
system parallel to the input coordinates and centered on the 
center of the object) which are adjusted using the
{\tt -a} and {\tt -e} options.  Other options control view distance
({\tt -h}), two-dimensional projection scale ({\tt -s}), rotation
({\tt -r} and {\tt -u}) and line width ({\tt -w}). Axes of any length
may be included  with the {\tt -x} option and lines, arrows and dots may be
added using {\tt -b}. Finally, {\tt -rd} and {\tt -rc} are useful
for removing dielectric-interface and conductor surfaces, respectively,
that block the desired view.
Table~\ref{dashmuse} gives the commands used to produce the line
drawings in this guide as examples. 
\begin{table}
\begin{center}
\begin{tabular}{clp{2.0in}}\hline
\multicolumn{1}{c}{Figure} & \multicolumn{1}{c}{{\tt fasctap} Usage} &
\multicolumn{1}{c}{Comments} \\ \hline
\protect\ref{1x1bus} & {\tt fastcap -m -x3 cond\_air\_1x1.qui}& left figure \\
& {\tt fastcap -m -x3 cond\_dielec\_1x1.qui} & center figure \\
& {\tt fastcap -m -x3 -ltemp.lst} & right figure; {\tt temp.lst} is 
{\tt 1x1bus.lst} with all the ``{\tt C}'' lines commented out\\
\hline
\protect\ref{spheres} & {\tt fastcap -m sphere1.neu} & left figure \\
& {\tt fastcap -m sphere2.neu} & right figure \\
\hline
\protect\ref{cubes} & {\tt cubegen -n5 | fastcap -m} & right figure \\
& {\tt cubegen -n7 | fastcap -m} & left figure \\
\hline
\protect\ref{testpic} & {\tt fastcap -ltest.lst -e75 -m} & \\
\hline
\protect\ref{connect} & {\tt fastcap -m -lconnector23.5-d.lst -s0.5} & left figure; input generated using  {\tt connector2.sh 3.5 -d}\\
& {\tt fastcap -m -lconnector33.5-d.lst -s0.75} & center figure; input generated using  {\tt connector3.sh 3.5 -d}\\
& {\tt fastcap -m -lconnector43.5-d.lst} & left figure; input
generated using {\tt connector4.sh 3.5 -d}\\
\hline
\protect\ref{ramcell} & {\tt fastcap -m -lramcell-d.lst -a0 -e87}
& figure (a); input generated using {\tt ramgen.sh -d}\\
& {\tt fastcap -m -rd -lramcell-d.lst -a0 -e70}
& figure (b); input generated using {\tt ramgen.sh -d}\\
\hline
\protect\ref{sphsha} & {\tt fastcap -q sphere1.neu} & use {\tt -dc} to display
total panel charges rather that panel charge densities \\
\hline
\protect\ref{cubeeg} & {\tt cubegen -d | fastcap -m -x2 -bcubeeg.fig} &
dots are in {\tt cubeeg.fig}; titling done in latex\\
\hline
\end{tabular}
\caption{Commands used to generate some representative figures in this guide.}
\label{dashmuse}
\end{center}
\end{table}

\subsubsection{Charge Density and Total Charge Pictures}
\label{chgplt}

The {\tt -q} option provides a means of checking the charge distribution
calculated by {\tt fastcap}. The command
\begin{quote}\tt
fastcap -q sphere1.neu
\end{quote}
causes {\tt fastcap} to write the postscript format file {\tt sphere11.ps}
containing the picture in Figure~\ref{sphsha}.
%
% shaded sphere showing  charge dens
%
% for fig2ug.awk
%%%&^ sphere1_q.ps
\begin{figure}
\centerline{
\psfig{figure=\figuredir/sphere1_q.ps,width=5.5in}
}
\caption{The file {\tt sphere11.ps} resulting from the command {\tt fastcap -q sphere1.neu}, showing the charge density on each panel after the capacitance calculation.}
\label{sphsha}
\end{figure}
A similar plot can be made for total panel charges using the same
command with the {\tt -dc} option added. 

For problems with more than
one conductor, {\tt fastcap -q} writes a postscript file corresponding
to the charge density present after each column calculation is completed.
The results of any particular subset of the column calculations may be
written to postscript files by specifying lists of columns to include
with the {\tt -q} option.  For example, in the five conductor problem
summarized by the list file {\tt test.lst} (see Section~\ref{listeg}),
{\tt fastcap -q2 -ltest.lst} produces only one  postscript file,
the one corresponding to conductor {\tt 2\%BUS}, and 
{\tt fastcap -qPY,LS,2 -ltest.lst}
writes a picture file for all but conductors {\tt 1\%BUS} and 
{\tt GND\%GROUP4}.




\subsection{Warnings and Errors}

The errors and warnings most often encountered when using {\tt fastcap}
are listed here in order of frequency with the most frequent appearing
first.

\begin{verbatim}
Warning: no multipole acceleration
\end{verbatim}\vspace{-\topsep}
The multipole approximation is not being used to compute iterates.
As explained in Section~\ref{sphrun}, this warning occurs most often when the 
input problem is too small to warrant the use of the multipole approximation.
The  message also appears when {\tt fastcap} is forced to calculate
the iterates without the multipole approximation using the {\tt -d0} 
command-line option. 
%Since the iterates are calculated using explicit
%multiplications when this message appears, the resulting capacitance values
%are actually more accurate than usual.

\begin{verbatim}
resolve_kill_lists: a conductor removed with -ri is in the -rs list
resolve_kill_lists: a conductor removed with -ri or -rs is in the -q list
resolve_kill_lists: all conductors either in -ri or -rs list
\end{verbatim}\vspace{-\topsep}
These errors are caused by asking for plots or computation involving
a conductor that has
been removed from the problem (with {\tt -ri}) or is not having its
column of the capacitance matrix calculated (in the
{\tt -rs} list).

\begin{verbatim}
placeq: Warning, removing identical quadrilateral panel
  rmved ctr = (-0.4375 3.43333 1.2) surf = `via.neu' trans = (0 0 0)
  saved ctr = (-0.4375 3.43333 1.2) surf = `via.neu' trans = (0 0 0)
\end{verbatim}\vspace{-\topsep}
Two panels with the same center coordinates have been given as input.
The most common cause of this error is extra panels in neutral files
due to careless meshing of patches. The example file {\tt via.neu}
contains six duplicate panels. The program removes the extra panels
and continues with the calculation.

\begin{verbatim}
is1stFaceDeeper: Warning, face ordering test failure
  alpha_fac, face 33 = 2.3498 alpha_facref, face 2 = 2.3489
...
\end{verbatim}\vspace{-\topsep}
This error is usually caused by attempting 
to produce a picture of an illegal discretization with
superimposed panels, as can happen when using a list file with
incorrect translation vectors.  Often the postscript file picture
is still correct.

\begin{verbatim}
read_panels: generic format file
  `../examples/cond_air_1x1.qui'
read after neutral file(s) in same group---reorder list file entries
\end{verbatim}\vspace{-\topsep}
A generic format file occurs after a neutral file in the {\tt list file}
and the files are grouped together using the ``{\tt +}'' option
(see Section~\ref{listif}). This condition is not allowed due to
the conductor renaming that can occur in generic format files.

\begin{verbatim}
name_data: conductor 'CONDUCTOR NAME'
  has no patch - redo naming so that one is included
assign_names: no conductor names specified
assign_names: 2 conductors have no names
assign_names: 2 names given for 4 conductors
\end{verbatim}\vspace{-\topsep}
The input {\tt PATRAN} neutral file contains conductors that are not
named properly. As mentioned in Section~\ref{neuif}, {\tt fastcap}'s 
neutral file
interface numbers the input conductors in the order they appear in
the input file. Since the user normally does not know this order, 
a component containing at least one patch that is part of each conductor
must be named using {\tt PATRAN}'s {\tt NAME} directive.

\begin{verbatim}
fastcap: out of memory in file `mulSetup.c' at line 135
 (NULL pointer on 1024 byte request)
Total Memory Allocated: 65135 kilobytes (brk = 0x13fff2d8)
placeq: 7 levels set up
\end{verbatim}\vspace{-\topsep}
Virtual memory has been exhausted. As explained in Section~\ref{trdoff}
the usual cause of this problem is running {\tt fastcap} with low order
expansions on moderate to large problems. In those cases the number of
partitioning levels should be set on the command line with the
{\tt -d} option to the amount successfully allocated by 
the multipole initialization subroutine {\tt placeq}. In
this example the option {\tt -d7} should be used. 
If the error
did not occur during the multipole initialization, then the 
line \verb~placeq: 7 levels set up~ is not printed. In that case the
number of partitioning levels should be set to one less than that
reported in the \verb~INPUT SUMMARY~ output section.
% cube size = 192 bytes, charge size = 160 bytes

\begin{verbatim}
mksCapDump: Warning - capacitance matrix has non-negative off-diagonals
  row 3 col 1
mksCapDump: Warning - capacitance matrix is not strictly diagonally dominant
  due to row 3
\end{verbatim}\vspace{-\topsep}
The capacitance matrix does not have the correct structure. The 
warnings are usually encountered when {\tt fastcap} is forced to use lower
order expansions. As discussed in Section~\ref{trdoff}, 
results obtained using lower order 
expansions are only reliable for the diagonal capacitance-matrix entries.
Inaccuracies in the off-diagonal matrix entries can produce these warnings,
even though the diagonal capacitances are accurate enough to be useful.
The diagonal dominance  error is also caused by
illegal discretizations with superimposed
panels, poorly discretized dielectric regions, particularly with
high permittivities, and poorly discretized conductor surfaces, especially
when different conductors are very close together.


%\begin{verbatim}
%\begin{quote}
%calcp: Warning - negative potential coefficient = -0.204
%\end{quote}
%\end{verbatim}
%The coefficient matrix for the linear system corresponding to the 
%capacitance problem, which must be a positive matrix, has a negative entry.
%As explained in Section~\ref{sphrun}, {\tt fastcap} attempts to
%break all nonplanar
%quadrilaterals into two triangles. Quadrilateral panels that are almost
%planar are occasionally not broken up by {\tt fastcap}

\begin{verbatim}
NONCONVERGENCE AFTER 2456 ITERATIONS
\end{verbatim}\vspace{-\topsep}
The iteration for the capacitance matrix has failed to converge. The
default version of {\tt fastcap} halts after $n$ iterations for an
$n$ panel problem if convergence has not occurred. 
This error rarely occurs for practical problems since
{\tt fastcap} uses a preconditioner.

%tiles too big relative to cubes

%
% should be an appendix
%
\subsection{Bugs}

This section attempts to itemize the major shortcomings of {\tt fastcap}
and the input file generator programs. 
The user is encouraged to report any other bugs by sending mail
to \verb~fastcap-bug@rle-vlsi.mit.edu~.

\subsubsection*{{\tt fastcap}}

\begin{enumerate}
\item List file provides for only translations, not rotations.
\item Translations of panels input via {\tt stdin} or {\tt input file}
are not possible.  Use a list file when translations are required.
\item All input dimensions are assumed in meters. There is currently
no way to scale the problem on the command line. The capacitance
of the true geometry is equal to the calculated capacitance multiplied
by the factor required to convert the input length units (always meters)
to the desired units.
\item Dielectric interfaces can only be input with a list file
(only conductors can be input through {\tt stdin} and {\tt input file}).
Use a list file for all dielectric interfaces.
\item The  dielectric surrounding all conductors
input through {\tt stdin} and {\tt input file} always has the 
permittivity of free space
multiplied by the permittivity factor specified with {\tt -p} (1.0 by
default).
Use a list file if other values are required.
\item Dielectric interfaces have no names and therefore cannot be
removed selectively from the input or charge distribution plots
with command line options.
To remove dielectric interfaces selectively from the input, comment out
lines in the list file.  To remove all dielectric interfaces from
charge plots use {\tt -rd}.
\item Infinitesimally thin conductors on dielectric interfaces (included
using lines starting with `{\tt B}' in list files), are not supported
in this release.
\item All surfaces are listed in the {\tt Input surfaces} part of
the output even if all the conductors in some of them have been
removed using {\tt -ri}.
\item The PATRAN neutral file interface has very little error checking.
If {\tt fastcap} is given an incorrect format neutral file it either
hangs or dumps core.
\item The actual postscript filenames used by {\tt fastcap -m} or 
{\tt fastcap -q<cond list>} are derived from the input file names and cannot
be directly changed by the user.
\item Postscript files produced with {\tt fastcap -m} or 
{\tt fastcap -q<cond list>} produce panel ordering errors (and
usually print warning messages) when panels intersect. However, such
discretizations are illegal.
\item Occasionally the postscript pictures are incorrect even
for legal discretizations. There are two known bugs leading
to this problem.  The first is often avoided by changing
the view angle slightly. The second is caused by problems with
dimensions on the order of $10^{-4}$ or less. This problem can
only be avoided by rescaling the input. 
%
% This error is due to two faces intersecting only at corners; ie
% the line(s) of one pass(passes) through the corner(s) of the other
% - no easy fix; right now all corner hits are not considered intersections
% - very strong function of view angle, however, so can avoid (sa comp book)
\item The panel sorting algorithm used to write out the postscript files
takes time proportional to the number of panels squared.  This limits
its usefulness to problems with less than a few thousand panels. For
larger problems the paneling should be done hierarchically so that
a coarser discretization can be used for pictures. For problems built
using the generic input file generators, the {\tt -d} option provides
this function.
\item When axes are included in postscript files using the
{\tt -x<axes length>} option, the axes' two-dimensional projections
appear in the postscript file before the panel fills. This means
that the object should be between the view point and the axes, otherwise
the axes can be obscured strangely. 
Thus
the option works best when viewing objects that lie entirely in
the positive orthant from a view point in the positive orthant.
\item The bounding box in postscript files produced with {\tt -q} is
incorrect if the shading key is included. The bounding box printed
is that of the shaded image without the key.

\end{enumerate}

\subsubsection*{\tt cubegen}

All the functions of the cube input file generator {\tt cubegen}
can be performed by {\tt pipedgen}, although {\tt cubegen}'s interface
is often more convenient.

\subsubsection*{\tt pyragen}

The pyramid input file generator {\tt pyragen} is currently
only capable of the default discretization, with three panels on a
side (also results from {\tt -n3}), and no discretization (using {\tt -d}).

\subsubsection*{\tt capgen}

All the functions of the 
parallel plate capacitor input file generator {\tt capgen}
can be performed by both {\tt cubegen} and {\tt pipedgen}, unless more
than two plates are required.



\subsection{Quick Reference}
\label{quiref}

%\subsubsection*{{\tt fastcap}: Capacitance Extractor}
\subsubsection*{\tt fastcap}

Run multipole accelerated capacitance extraction algorithm on specified 
conductor geometry, optionally produce a postscript file picture of
the charge distribution used in the calculation, or produce a postscript 
file line drawing of the input geometry. Arguments can be specified in
any order.
\begin{verbatim}
fastcap [-o<expansion order>] [-d<partitioning depth>] [<input file>]
        [-p<permittivity factor>] [-rs<cond list>] [-ri<cond list>]
        [-] [-l<list file>] [-t<iter tol>] [-a<azimuth>] [-e<elevation>]
        [-r<rotation>] [-h<distance>] [-s<scale>] [-w<linewidth>]
        [-u<upaxis>] [-q<cond list>] [-rc<cond list>] [-x<axeslength>]
        [-b<.figfile>] [-m] [-rk] [-rd] [-dc] [-c] [-v] [-n] [-f] [-g]
\end{verbatim}

\begin{center}
\begin{tabular}{ccll}
\multicolumn{4}{c}{\mbox{\tt fastcap} Options}\\\hline
\multicolumn{1}{c}{Option}&\multicolumn{1}{c}{Default}&
\multicolumn{1}{c}{Range}&
\multicolumn{1}{c}{Function}\\\hline
\mbox{\tt -o}&\mbox{\tt 2}& \mbox{\tt 0, 1\ldots}
&Specifies expansion order.\\
\mbox{\tt -d}&auto&\mbox{\tt 0, 1\ldots}
&Specifies spatial partitioning depth.\\
\mbox{\tt -p}&\mbox{\tt 1.0}&$>$ \mbox{\tt 0.0}
&Specifies relative permittivity factor.\\
\mbox{\tt -rs}& \ddag & \ddag
&Specifies conductors to remove from solve list.\\
\mbox{\tt -ri}& \ddag & \ddag
&Specifies conductors to remove from input.\\
\mbox{\tt -}& --- & ---
&Forces conductor surface read from \mbox{\tt stdin}.\\
\mbox{\tt -l}& --- & ---
&Specifies surface {\tt list file}.\\
\mbox{\tt -t}& {\tt 0.01} & $>$ {\tt 0.0}
&Specifies iterative loop tolerance on $\|residual\|_2$.\\
\hline
\mbox{\tt -a}& \mbox{\tt 50.0} & \dag
&Specifies azimuth view angle in degrees.\\
\mbox{\tt -e}& \mbox{\tt 50.0} & \dag
&Specifies elevation view angle in degrees.\\
\mbox{\tt -r}& \mbox{\tt 0.0} & \dag
&Specifies final rotation of 2-D image in degrees.\\
\mbox{\tt -h}& \mbox{\tt 2.0} & $\geq$ \mbox{\tt 0.0}
&Specifies distance from surface of object in object radii.\\
\mbox{\tt -s}& \mbox{\tt 1.0} & $>$ \mbox{\tt 0.0}
&Specifies final scaling of 2-D image.\\
\mbox{\tt -w}& \mbox{\tt 1.0} & $\geq$ \mbox{\tt 0.0}
&Specifies postscript file line width.\\
\mbox{\tt -u}& \mbox{\tt z} & \mbox{\tt x}, \mbox{\tt y}, or \mbox{\tt z}
&Specifies which 3-D axis is mapped to y-axis in 2-D image.\\
\mbox{\tt -q}& \ddag & \ddag
&Causes charge distribution postscript files to be written.\\
\mbox{\tt -rc}& \ddag & \ddag
&Specifies conductors to remove from all pictures.\\
\mbox{\tt -x}& \mbox{\tt 1.0} & $>$ \mbox{\tt 0.0}
&Includes axes of length \mbox{\tt axeslength} in picture.\\
\mbox{\tt -b}& $\diamond $ & $\diamond $
&Specifies lines, dots and arrows to superimpose on picture.\\
\mbox{\tt -m}& --- & ---
&Causes postscript-file line drawing write.\\
\mbox{\tt -rk}& --- & ---
&Removes the shading key in charge distribution pictures.\\
\mbox{\tt -rd}& --- & ---
&Removes all dielectric surfaces from postscript file pictures.\\
\mbox{\tt -dc}& --- & ---
&Makes total charge, rather than charge density, {\tt -q} pictures.\\
\mbox{\tt -c}& --- & ---
&Puts the command line in postscript file pictures.\\
\mbox{\tt -v}& --- & ---
&Removes {\tt showpage} from postscript file.\\
\mbox{\tt -n}& --- & ---
&Numbers faces in order input.\\
\mbox{\tt -f}& --- & ---
&Suppresses hidden line removal.\\
\mbox{\tt -g}& --- & ---
&Prints the graph used to order the panels in postscript file.\\
\hline
\end{tabular}
\end{center}
\noindent\dag Range is unrestricted.

\noindent\ddag Conductor list format: {\tt [<name>],...,[<name>]}. Name
strings need only be specified with enough leading characters to be unique.
Specifying {\tt -q} with no list is the same as giving a list of all the
conductors.

\noindent$\diamond $ See function {\tt readLines()} in {\tt src/zbufInOut.c}
for a description of the {\tt .fig} file format.

\subsubsection*{\tt cubegen}

Generate a cubic capacitor problem for input to {\tt fastcap}. Arguments
may be specified in any order.
\begin{verbatim}
cubegen [-xo<originx>] [-yo<originy>] [-zo<originz>]
        [-xh<heightx>] [-yh<heighty>] [-zh<heightz>]
        [-n<num panels/side>] [-e<rel edge panel width>] [-na<name>]
        [-o] [-t] [-b] [-p] [-pfl] [-pfr] [-pbl] [-pbr] [-d]
\end{verbatim}

\begin{center}
\begin{tabular}{ccll}
\multicolumn{4}{c}{\mbox{\tt cubegen} Options}\\\hline
\multicolumn{1}{c}{Option}&\multicolumn{1}{c}{Default}&
\multicolumn{1}{c}{Range}&
\multicolumn{1}{c}{Function}\\\hline
\mbox{\tt -xo}&{\tt 0.0}& \dag
&Specifies cube origin x coordinate in meters.\\
\mbox{\tt -yo}&{\tt 0.0}& \dag
&Specifies cube origin y coordinate in meters.\\
\mbox{\tt -zo}&{\tt 0.0}& \dag
&Specifies cube origin z coordinate in meters.\\
\mbox{\tt -xh}&{\tt 1.0}& \dag
&Specifies cube excursion from origin in x direction, meters.\\
\mbox{\tt -yh}&{\tt 1.0}& \dag
&Specifies cube excursion from origin in y direction, meters.\\
\mbox{\tt -zh}&{\tt 1.0}& \dag
&Specifies cube excursion from origin in z direction, meters.\\
\mbox{\tt -n}&{\tt 3}&\mbox{\tt 1, 2\ldots}
&Specifies number of panels per cube edge.\\
\mbox{\tt -e}&{\tt 0.1}&$\geq$ \mbox{\tt 0.0}
&Specifies edge to inner panel width ratio.\\
{\tt -na} & `{\tt 1}' & \ddag
&Specifies conductor name.\\
{\tt -o} & --- & ---
&Causes cube to be centered on the point (0 0 0).\\
{\tt -t} & --- & ---
&Removes panels on the top face, $z =$ {\tt zo + zh} plane$\diamond $.\\
{\tt -b} & --- & ---
&Removes panels on the bottom face, $z =$ {\tt zo} plane$\diamond $.\\
{\tt -p} & --- & ---
&Equivalent to {\tt -pfl -pfr -pbl -pbr}.\\
{\tt -pfl} & --- & ---
&Removes panels on the front left face, $x =$ {\tt xo + xh} plane$\diamond $.\\
{\tt -pfr} & --- & ---
&Removes panels on the front right face, $y =$ {\tt yo + yh} plane$\diamond $.\\
{\tt -pbl} & --- & ---
&Removes panels on the back left face, $y =$ {\tt yo} plane$\diamond $.\\
{\tt -pbr} & --- & ---
&Removes panels on the back right face, $x =$ {\tt xo} plane$\diamond $.\\
{\tt -d} & --- & ---
&Disables discretization of cube faces.\\
\hline
\end{tabular}
\end{center}
\noindent\dag Range is unrestricted.

\noindent\ddag Any string with no spaces is allowed.

\noindent$\diamond $ Face names are based on viewing the cube
from the positive orthant. See also Figure~\ref{cubeeg}.
%
% the cube sides and their reference points
%
% for fig2ug.awk
%%%&^ cubeeg.ps
\setlength{\unitlength}{1.0in}
\begin{figure}
\centerline{
\psfig{figure=\figuredir/cubeeg.ps,height=3.0in}
}
\begin{picture}(6.25,0)
\thicklines
\put(3.4,2.3){({\tt xo yo zo+zh}) = (0 0 1)}
\put(3.9,1.1){({\tt xo yo+yh zo}) = (0 1 0)}
\put(1.7,0.3){({\tt xo+xh yo zo}) = (1 0 0)}
\put(2.5,0.5){\vector(0,1){0.6}}
\put(3.1,1.8){\tt t}
\put(2.7,1.25){\tt fl}
\put(3.45,1.25){\tt fr}
\end{picture}
\caption{The discretization generated by the command {\tt cubegen -d}, showing
how the faces are identified.  The axes are two units long.}
\label{cubeeg}
\end{figure}


%\subsubsection*{{\tt capgen}: Parallel Plate Capacitor Discretization Generator}
\subsubsection*{\tt capgen}

Generate a parallel plate capacitor problem for input to {\tt fastcap}.
Arguments may be specified in any order.
\begin{verbatim}
capgen [-s<plate sep>] [-w<plate width>] [-p<num plates>]
       [-n<num panels/plate width>] [-e<rel edge panel width>]
       [-na<cond name>] [-d]
\end{verbatim}

\begin{center}
\begin{tabular}{ccll}
\multicolumn{4}{c}{\mbox{\tt capgen} Options}\\\hline
\multicolumn{1}{c}{Option}&\multicolumn{1}{c}{Default}&
\multicolumn{1}{c}{Range}&
\multicolumn{1}{c}{Function}\\\hline
\mbox{\tt -s}&{\tt 0.1}&$>$ \mbox{\tt 0.0}
&Specifies plate separation in meters.\\
\mbox{\tt -w}&{\tt 1.0}&$>$ \mbox{\tt 0.0}
&Specifies plate width in meters.\\
\mbox{\tt -p}&{\tt 2}&\mbox{\tt 1, 2\ldots}
&Specifies number of parallel plates.\\
\mbox{\tt -n}&{\tt 3}&\mbox{\tt 1, 2\ldots}
&Specifies number of panels per plate width.\\
\mbox{\tt -e}&{\tt 0.1}&$\geq$ \mbox{\tt 0.0}
&Specifies edge to inner panel width ratio.\\
{\tt -na} & \dag & \dag
&Specifies conductor name base.\\
{\tt -d} & --- & ---
&Disables discretization of plates.\\
\hline
\end{tabular}
\end{center}
\noindent\dag Any string with no spaces is allowed. By default
the conductors are given the first $p$ numbers as names, where
$p$ is the number of panels specified with {\tt -p}.

%\subsubsection*{{\tt busgen}: Bus Crossing Discretization Generator}
\subsubsection*{\tt busgen}

Generate a bus crossing problem for input to {\tt fastcap}. Arguments
may be specified in any order.
\begin{verbatim}
busgen [-xo<originx>] [-yo<originy>] [-zo<originz>]
       [-c<conductors/bus>] [-w<wire width>]
       [-n<num panels/wire width>] [-e<rel edge panel width>]
       [-na<name base>] [-d]
\end{verbatim}

\begin{center}
\begin{tabular}{ccll}
\multicolumn{4}{c}{\mbox{\tt busgen} Options}\\\hline
\multicolumn{1}{c}{Option}&\multicolumn{1}{c}{Default}&
\multicolumn{1}{c}{Range}&
\multicolumn{1}{c}{Function}\\\hline
\mbox{\tt -xo}&{\tt 0.0}& \dag
&Specifies bus origin x coordinate in meters.\\
\mbox{\tt -yo}&{\tt 0.0}& \dag
&Specifies bus origin y coordinate in meters.\\
\mbox{\tt -zo}&{\tt 0.0}& \dag
&Specifies bus origin z coordinate in meters.\\
\mbox{\tt -c}&{\tt 2}&\mbox{\tt 1, 2\ldots}
&Specifies number of conductors per bus.\\
\mbox{\tt -w}&{\tt 1.0}&$>$ \mbox{\tt 0.0}
&Specifies conductor width in meters.\\
\mbox{\tt -n}&{\tt 3}&\mbox{\tt 1, 2\ldots}
&Specifies number of panels per wire width.\\
\mbox{\tt -e}&{\tt 0.1}&$\geq$ \mbox{\tt 0.0}
&Specifies edge to inner panel width ratio.\\
{\tt -na} & \ddag & \ddag
&Specifies conductor name base.\\
{\tt -d} & --- & ---
&Disables discretization of conductor faces.\\
\hline
\end{tabular}
\end{center}
\noindent\dag Range is unrestricted.

\noindent\ddag Any string with no spaces is allowed. By default
the conductors are given the first $2c$ numbers as names, where
$c$ is the number of conductors per bus specified with {\tt -c}.

%\subsubsection*{{\tt pipedgen}: Parallelepiped Discretization Generator}
\subsubsection*{\tt pipedgen}

Generate a parallelepiped problem for input to {\tt fastcap}.
Arguments may be specified in any order.
\begin{verbatim}
pipedgen [-cr <x y z>] [-c1 <x y z>] [-c2 <x y z>] [-c3 <x y z>]
         [-n<num panels/side>] [-e<rel edge panel width>]
         [-na<name>] [-t] [-b] [-p] [-pfl] [-pfr] [-pbl] [-pbr] [-d]
\end{verbatim}

\begin{center}
\begin{tabular}{ccll}
\multicolumn{4}{c}{\mbox{\tt pipedgen} Options}\\\hline
\multicolumn{1}{c}{Option}&\multicolumn{1}{c}{Default}&
\multicolumn{1}{c}{Range}&
\multicolumn{1}{c}{Function}\\\hline
{\tt -cr} & {\tt (0 0 0)} & \dag
&Specifies parallelepiped reference corner.\\
{\tt -c1} & {\tt (1 0 0)} & \dag
&Specifies first parallelepiped corner.\\
{\tt -c2} & {\tt (0 1 0)} & \dag
&Specifies second parallelepiped corner.\\
{\tt -c3} & {\tt (0 0 1)} & \dag
&Specifies third parallelepiped corner.\\
\mbox{\tt -n}&{\tt 3}&\mbox{\tt 1, 2\ldots}
&Specifies number of panels per face edge.\\
\mbox{\tt -e}&{\tt 0.1}&$\geq$ \mbox{\tt 0.0}
&Specifies edge to inner panel width ratio.\\
{\tt -na} & `{\tt 1}' & \ddag
&Specifies conductor name.\\
{\tt -t} & --- & ---
&Removes panels on the top face, {\tt c1-cr-c2} plane$\diamond $.\\
{\tt -b} & --- & ---
&Removes panels on the bottom face, parallel to top$\diamond $.\\
{\tt -p} & --- & ---
&Equivalent to {\tt -pfl -pfr -pbl -pbr}.\\
{\tt -pfl} & --- & ---
&Removes panels on the front left face, {\tt c1-cr-c3}  plane$\diamond $.\\
{\tt -pfr} & --- & ---
&Removes panels on the front right face, {\tt c2-cr-c3} plane$\diamond $.\\
{\tt -pbl} & --- & ---
&Removes panels on the back left face, parallel to front right$\diamond $.\\
{\tt -pbr} & --- & ---
&Removes panels on the back right face, parallel to front left$\diamond $.\\
{\tt -d} & --- & ---
&Disables discretization of parallelepiped faces.\\
\hline
\end{tabular}
\end{center}
\noindent \dag Range is unrestricted.

\noindent\ddag Any string with no spaces is allowed.

\noindent $\diamond $Face names are based on viewing the parallelepiped
from the positive orthant. See also Figure~\ref{pipeeg}.
%
% the parallelepiped and its reference points
%
% for fig2ug.awk
%%%&^ pipeeg.ps
\begin{figure}
\centerline{
\psfig{figure=\figuredir/pipeeg.ps,height=3.0in}
}
\begin{picture}(6.25,0)
\thicklines
\put(1.3,1.8){\makebox(1.0,0.2)[r]{{\tt c1} = (1 0 1)}}
\put(4,1.8){{\tt c2} = (0 1 1)}
\put(4.2,1.4){\vector(-1,0){1.0}}
\put(4.3,1.35){{\tt cr} = (1 1 1)}
\put(3.1,0.4){{\tt c3} = (1 1 0)}
\put(3.1,1.8){\tt t}
\put(2.7,1.25){\tt fl}
\put(3.45,1.25){\tt fr}
\end{picture}
\caption{The discretization generated by the command 
{\tt pipedgen -cr 1 1 1  -c1 1 0 1  -c2 0 1 1  -c3 1 1 0  -d}, showing
how the faces are identified.  The axes are two units long.}
\label{pipeeg}
\end{figure}



\subsubsection*{\tt pyragen}

Generate a pyramid capacitor problem for input to {\tt fastcap}. Arguments
may be specified in any order.
\begin{verbatim}
pyragen [-xo<originx>] [-yo<originy>] [-zo<originz>]
        [-xh<heightx>] [-yh<heighty>] [-zh<heightz>]
        [-n<num panels/side>] [-e<rel edge panel width>]
        [-na<name>] [-b] [-p] [-pfl] [-pfr] [-pbl] [-pbr] [-d]
\end{verbatim}

\begin{center}
\begin{tabular}{ccll}
\multicolumn{4}{c}{\mbox{\tt pyragen} Options}\\\hline
\multicolumn{1}{c}{Option}&\multicolumn{1}{c}{Default}&
\multicolumn{1}{c}{Range}&
\multicolumn{1}{c}{Function}\\\hline
\mbox{\tt -xo}&{\tt 0.0}& \dag
&Specifies pyramid origin x coordinate in meters.\\
\mbox{\tt -yo}&{\tt 0.0}& \dag
&Specifies pyramid origin y coordinate in meters.\\
\mbox{\tt -zo}&{\tt 0.0}& \dag
&Specifies pyramid origin z coordinate in meters.\\
\mbox{\tt -xh}&{\tt 1.0}& \dag
&Specifies pyramid base excursion from origin in x direction, meters.\\
\mbox{\tt -yh}&{\tt 1.0}& \dag
&Specifies pyramid base excursion from origin in y direction, meters.\\
\mbox{\tt -zh}&{\tt 1.0}& \dag
&Specifies pyramid height at apex (z direction), meters.\\
\mbox{\tt -n}&{\tt 3}&\mbox{\tt 1, 2\ldots}
&Specifies number of panels per cube edge.\\
\mbox{\tt -e}&{\tt 0.1}&$\geq$ \mbox{\tt 0.0}
&Specifies edge to inner panel width ratio.\\
{\tt -na} & `{\tt 1}' & \ddag
&Specifies conductor name.\\
{\tt -b} & --- & ---
&Removes panels on the bottom face, $z =$ {\tt zo} plane$\diamond $.\\
{\tt -p} & --- & ---
&Equivalent to {\tt -pfl -pfr -pbl -pbr}.\\
{\tt -pfl} & --- & ---
&Removes panels on the front left face$\diamond $.\\
{\tt -pfr} & --- & ---
&Removes panels on the front right face$\diamond $.\\
{\tt -pbl} & --- & ---
&Removes panels on the back left face$\diamond $.\\
{\tt -pbr} & --- & ---
&Removes panels on the back right face$\diamond $.\\
{\tt -d} & --- & ---
&Disables discretization of cube faces.\\
\hline
\end{tabular}
\end{center}
\noindent\dag Range is unrestricted.

\noindent\ddag Any string with no spaces is allowed.

\noindent$\diamond $ Face names are based on viewing the pyramid
from the positive orthant in a manner analogous to the {\tt cubegen}
case. 


\subsubsection*{Examples}

Run {\tt fastcap} with tighter iterative loop tolerance
on {\tt sphere2.neu} with
dielectric relative permittivity  $3.5$:
\begin{quote}\tt
fastcap -p3.5 sphere2.neu -d3 -t0.001
\end{quote}
Run {\tt fastcap} on a finely discretized
$5m\times 5m\times 5m$ cube with zero order expansions 
for speed and seven partitioning
levels to conserve memory:
\begin{quote}\tt
cubegen -n30 -s5.0 | fastcap -o0 -d7
\end{quote}
Run {\tt fastcap} on the $3\times 3$ bus crossing problem with five uniform
panels along short edges:
\begin{quote}\tt
busgen -e1.0 -n5 -c3 | fastcap 
\end{quote}
Run {\tt fastcap} with order zero expansions to get a rough idea of the
larger capacitances for a 2$\mu$m $4\times 4$ bus crossing embedded in 
silicon dioxide:
\begin{quote}\tt
busgen -c4 -w2E-6 | fastcap -p3.9 -o0
\end{quote}
%
% something with stdin and command line/list file
