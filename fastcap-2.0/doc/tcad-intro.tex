\section{Introduction}

In
\newcommand{\introfoot}{
% Manuscript received \today.
This work was supported by the Defense Advanced Research Projects
Agency contract N00014-87-K-825, the National Science Foundation
and grants from I.B.M.\ and Analog Devices.

The authors are with the  Research Laboratory
of Electronics, Department of Electrical Engineering and Computer
Science, Massachusetts Institute of Technology, Cambridge,
MA, 02139, U.S.A.
}
\footnotetext{\introfoot}
the design of high performance integrated circuits and integrated
circuit packaging, there are many cases where accurate estimates of
the capacitances of complicated three dimensional structures are
important for determining final circuit speeds or functionality.  Two
examples of complicated three-dimensional structures for which capacitance
strongly affects performance are
dynamic memory cells, and the
chip carriers commonly used in high density
packaging.  In these problems, capacitance extraction
is made tractable by assuming the conductors are ideal and are
embedded in a piecewise-constant dielectric medium.  Then to compute
the capacitances, Laplace's equation is solved numerically over the
charge free region with the conductors providing boundary conditions.

Although there are a variety of numerical methods that can be used to
solve Laplace's equation, for three-dimensional capacitance
calculations the usual approach is to apply a boundary-element
technique to the integral form of Laplace's
equation \cite{rue,rao,Nin}.  In these approaches the
surfaces or edges of all the conductors are broken into small panels
or tiles and it is assumed that on each panel $ i $, a charge, $ q_i$, is
uniformly or piece-wise linearly distributed.  The potential on each
panel is then computed by summing the contributions to the potential
from all the panels using Laplace's equation Green's functions.  In
this way, a matrix of potential coefficients, $ P $, relating the set
of $ n $ panel potentials and the set of $ n $ panel charges is
constructed.  The resulting $ n \times n $ system of equations
must be solved to compute capacitances.  Typically,
Gaussian elimination or Cholesky factorization is used to solve the system of
equations, in which case the number of operations is order $ n^3 $.
Clearly, this approach becomes computationally intractable if the
number of panels exceeds several hundred, and this limits the size of
the problem that can be analyzed to one with a few conductors.

An approach to reducing the computation time that is particularly
effective for computing the diagonal terms of the capacitance matrix,
also referred to as the self-capacitances, is to ignore small contributions
to the potential coefficient matrix due to pairs of panels which are separated
by large distances \cite{dew}.
In this paper we present a similar approach, which approximates
small potential coefficients with multipole expansions.
We show that this approach produces an algorithm which accurately computes
both the self and coupling capacitances, and has a computational
complexity of nearly $ mn $, where $ m $ is the number of
conductors.  Our algorithm, which is really the pasting together of
three well-known algorithms  \cite{roh}, is presented in three
sections.  To begin, in the next section one of the standard integral
equation approaches is briefly described, and it is shown that the
algorithm requires the solution of an $ n\times n $ dense nearly
symmetric matrix.  Then, in Section~\ref{gcrsec}, a generalized
conjugate residual algorithm is described, and is shown to reduce
the complexity of the calculation to roughly order $ m n^2 $.  In
Section~\ref{accsec}, it is shown that the major computation of the
conjugate residual algorithm, evaluation of a potential field from a
charge distribution, can be computed in order $ n $ time using a
multipole algorithm.  In Section~\ref{ressec}, we describe some experimental
results and in Section~\ref{consec} we present our conclusions and
acknowledgments. Finally,
some implementation details are presented in an appendix.

