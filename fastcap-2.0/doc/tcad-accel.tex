\section{Accelerating the Matrix-Vector Product}
\label{accsec}

As can be seen from examining Algorithm 1,  assuming the
number of iterations required is small, the major costs of the
GCR algorithm are initially forming the dense matrix $ P $, and in each
iteration computing the matrix-vector product $ P w $, both of which
require order $ n^2 $ operations.  Computing the capacitance matrix
with Algorithm 1 is therefore at least order $ m n^2 $, and may be
higher if the number of GCR iterations increases with the problem size.
Note that if the number of panels per conductor is low, Algorithm~1
may not be much more efficient than using direct factorization.

An approach that avoids forming most of $ P $, and reduces the cost of
computing the matrix-vector product $ P w $, can be derived by
recalling that if $ w $ is thought of as representing charges
distributed on panels, then $ P w $ is a potential due to that
charge distribution.  In addition, if the distance between the centers
of panel $ i $ and panel $ j $ is large compared to the panel sizes, then
$ P_{ij} \approx \frac{1}{\| x_i - x_j \| } $.  That is, for widely separated
panels, the $ j^{th} $ panel charge has the same effect on the
potential at $ x_i $ as would a point charge of value $ w_j $ located at
panel $ j $'s center.

\newcommand{\mulsepcap}{ The evaluation point potentials are approximated with a multipole expansion.}
\fig{\figuredir/mulsep.eps}{mulsep}{\mulsepcap}{216}{htbp}{-73}{-424}{1}{1}
\newcommand{\locsepcap}{ The evaluation point potentials are approximated with a local expansion.}
%
\fig{\figuredir/locsep.eps}{locsep}{\locsepcap}{216}{htbp}{-65}{-424}{1}{1}
%
To see how these observations can help simplify the computation of $ P
w $, consider the situation (depicted in 2-D for simplicity) in
Figs.~\ref{mulsep} and~\ref{locsep}.
In either figure, the obvious
approach to determining the potential at the $ n_1 $ evaluation points
from the $ n_2 $ point-charges involves $ n_1 *
n_2 $ operations; at each of the $n_1$
evaluation points one simply sums the
contribution to the potential from  $ n_2 $ charges.  An accurate
approximation for the potentials for the case of Fig.~\ref{mulsep}
can be computed in many fewer operations using {\it multipole
expansions}, which exploit the fact that $ r >> R $ (defined in
Fig.~\ref{mulsep}).  That is, the details of the distribution of the
charges in the inner circle of radius $R$ in Fig.~\ref{mulsep} do not strongly
effect the potentials at the evaluation points outside the outer
circle of radius $r$.  It
is also possible to compute an accurate approximation for the
potentials at the evaluation points
in the inner circle of Fig.~\ref{locsep} in many fewer
than $ n_1 * n_2 $ operations using {\it local expansions}, which
again exploit the fact that $ r >> R $ (as in Fig.~\ref{locsep}). In
this second case, what can be ignored is the details of the evaluation
point distribution.

\subsection{Multipole Expansions}

\newcommand{\monplecap}{ The charges are replaced by the first multipole expansion coefficient.}
%
\fig{\figuredir/monple.eps}{monple}{\monplecap}{216}{htbp}{-78}{-424}{1}{1}
%
A rough approximation to the effect of the $ n_2 $ charges in the inner
circle in Fig.~\ref{mulsep} can be derived by replacing those charges with a
single charge equal to their sum, placed at the inner circle's center
(See Fig.~\ref{monple}).  
The number of operations to compute the $ n_1 $
potentials given this simplification is then $ n_2 + n_1 $, $ n_2 $
operations to compute the sum of charges, and $
n_1 $ operations to calculate the potentials at the evaluation points.
Note that the accuracy of this approximation improves as the
separation $ r $ between the nearest evaluation point and the
center of the inner circle containing the charges
increases compared to the inner circle's radius.
%circles increases compared to the radius of
%distribution for the evaluation points and for the charges.

In the simplified approach above, the potential due to the charges in
Fig.~\ref{mulsep} is approximated by
\begin{equation}
\frac{\sum_{i=1}^{n_2} q_i}{r_j},
\end{equation}
where $ r_j $ is the distance between the center of the charge circle
and the $j^{th}$ evaluation point.
Such an approximation is referred to as
a monopole approximation, and is the first term in the general 
multipole approximation for charge distributions.
In general, the true potential, $\psi$, due to 
point charges inside a sphere
at a location outside the radius of the sphere  can be approximated
arbitrarily accurately
by a truncated
{\em multipole} expansion,
\begin{equation}
\psi(r_j, \theta_j , \phi_j ) \approx \sum_{n=0}^{l} \sum_{m=-n}^{n}
\frac{M^m_n}{r_j^{n+1}} Y^m_n(\theta_j , \phi_j ),
\label{eq:multi}
\end{equation}
where $ l $ is the order of the expansion, $ r_j $, $\theta_j $ and $ \phi_j$ 
are the spherical coordinates of the $j^{th}$ evaluation location referenced
to the sphere's center. The $ Y^m_n(\theta_j , \phi_j ) $ factors
are called spherical harmonics \cite{gre,jac} and the $ M^m_n$ 
are complex weights
known as the multipole coefficients. 
The coefficients are related to
the charges
by
\begin{equation}
M^m_n = \sum_{i=1}^{n_2} q_i \rho^n_i Y^{-m}_n(\alpha_i, \beta_i),
\label{eq:mulmom}
\end{equation}
where $ \rho_i $, $ \alpha_i $, and $ \beta_i $ are the spherical
coordinates of the $ i^{th} $ charge relative to the sphere's center.
It has been shown that the truncated multipole expansion error is bounded by
\begin{equation}
| \psi(r_j, \theta_j , \phi_j ) - \sum_{n=0}^{l} \sum_{m=-n}^{n}
\frac{M^m_n}{r_j^{n+1}} Y^m_n(\theta_j , \phi_j )| < K_1 
\left( \frac{R}{r_j} \right)^{l+1} \leq K_1 \left( \frac{R}{r} \right)^{l+1}
\label{eq:mbound}
\end{equation}
where
$ K_1 $ is independent of $l$, and
$r$ and
$ R $ are as in Figs.~\ref{mulsep} and~\ref{monple}, 
the distance to the nearest
evaluation point and the
radius of the sphere of charge, respectively \cite{gre}.  If the nearest
evaluation point is well outside the sphere, then  (\ref{eq:mbound}) 
implies that all the evaluation point potentials can be accurately computed
using just a few terms of the multipole expansion.

\subsection{Local Expansions}

Multipole expansions cannot be used to simplify calculating
the potentials for the evaluation points in the smaller circle of
Fig.~\ref{locsep}, as the charges are too widely distributed.
However, it is still possible to compute approximate potentials at the
$ n_1 $ evaluation points due to the $ n_2 $ charges in $ n_1 + n_2 $
operations.  To see this, consider that the potential at any of the $
n_1 $ evaluation points in the smaller circle  is
roughly the same as the potential evaluated at the center of
the circle.  Thus the potential at an evaluation point can be approximated by
\begin{equation}
\sum_{i=1}^{n_2}\frac{q_i}{\rho_i},
\label{0thloc}
\end{equation}
where $\rho_i$ is the distance from the $i^{th}$ charge to the center of
the circle containing the evaluation points.
Estimating the potentials at the $n_1 $ 
evaluation points therefore requires 
$ n_2 $ operations to compute the
potential at the circle's center by   (\ref{0thloc}), 
and $ n_1 $ additional operations to copy
that result to the $ n_1 $ evaluation points.  
Note that the approximation improves as
the separation between the charges and the circle's
center increases
compared to the circle's radius.

Just as in the multipole case, it is possible to improve the
accuracy of the above {\em local} expansion by including the effect of
the distance between an evaluation point and the enclosing
sphere's center.  In general, the truncated local expansion 
approximation for the exact potential
in a sphere due to  charges outside the radius of the sphere is
given by 
\begin{equation}
\psi(r_j, \theta_j , \phi_j ) \approx \sum_{n=0}^{\l} \sum_{m=-n}^{n}
L^m_n  Y^m_n(\theta_j , \phi_j ) r_j^{n},
\label{eq:local}
\end{equation}
where $ l $ is the order of the expansion,
$ r_j $, $\theta_j $ and $ \phi_j $ are the spherical coordinates of
the $j^{th}$ evaluation location with respect to the sphere's center, 
and the $\L^m_n $ factors are the complex local expansion coefficients.  
For a set of $ n_2 $ charges
outside the sphere, the local expansion coefficients are given by
\begin{equation}
L^m_n = \sum_{i=1}^{n_2} \frac{q_i}{\rho^{n+1}_i} Y^{-m}_n(\alpha_i, \beta_i),
\end{equation}
where $ \rho_i $, $ \alpha_i $, and $ \beta_i $ are the spherical
coordinates of the $ i^{th} $ charge relative to the center of the
sphere containing the evaluation points.
As for the multipole expansion, the error 
introduced  by the local expansion is related to a ratio of distances,
\begin{equation}
| \psi(r_j, \theta_j , \phi_j ) - \sum_{n=0}^{\l} \sum_{m=-n}^{n}
L^m_n  Y^m_n(\theta_j , \phi_j ) r_j^{n} | < K_2
\left( \frac{r_j}{r} \right)^{l+1} \leq K_2 \left( \frac{R}{r} \right)^{l+1}
\end{equation}
where
$ K_2 $ is independent of $l$, and 
$r$ and
$ R $ are as in Fig.~\ref{locsep},
the distance to the nearest
charge location and the
radius of the sphere of evaluation points, respectively \cite{gre}.
Therefore, if the charges are well outside the sphere then
the potential inside the sphere can be accurately computed using just a few
terms of the local expansion.

\subsection{The Multipole Algorithm}

Low order multipole and local expansions can be used to accurately compute
the potentials at $ n $ evaluation points due to $ n $ panel charges in
order $ n $ operations, even for general evaluation point and charge 
distributions, but the multipole and local expansions have to be
applied carefully, both to ensure adequate separation, and to avoid excess
calculation.  Below we give a multipole algorithm for computing the
potentials at the $ n $ panel center points due to $ n $ panel charges.  The
algorithm requires $ O(n) $ operations, and was originally presented in
\cite{gre} with variants in \cite{roh,kat,zha}.  The algorithm is reproduced
here, modified to fit the boundary-element  calculations.

To begin, the smallest cube that contains the entire collection of panels
for the problem of interest is determined.  This cube will be referred to
as the level 0, or root, cube. Then, the volume of the cube is broken into
eight equally sized child cubes, referred to as level 1 cubes, and each has
the level 0 cube as its parent. The panels are divided among the child cubes
by associating a panel with a cube if the panel's center point is contained
in the cube.  Each of the level 1 cubes is then subdivided into eight level
2 child cubes and the panels are again distributed based on their center
point locations. The result is a collection of $ 64 $ level 2 cubes and a $
64 $-way partition of the panels.  This process is repeated to produce $ L $
levels of cubes, and $ L $ partitions of panels starting with an $8$-way
partition and ending with an $ 8^L $-way partition.  The number of levels,
$ L $, is chosen so that the maximum number of panels in a finest, or
$L^{th} $, level cube is less than some threshold (four is a typical
default).

The following terms are used to concisely describe the multipole algorithm.
\begin{Definition}
{\it Evaluation Points of a Cube}: The center points of the panels associated
with the cube.
\end{Definition}
\begin{Definition}
{\it Nearest Neighbors of a Cube}: Those cubes which have a corner in
common with the given cube.
\end{Definition}
\begin{Definition}
{\it Second Nearest Neighbors of a Cube}: Those cubes which are not
nearest neighbors but have a corner in common with a nearest neighbor.
\end{Definition}
Note that there are at most $ 124 $ nearest and second nearest neighbors 
of a cube, excluding the cube itself. 

\begin{Definition}
{\it Interaction Cubes of a given cube}: Those cubes which are either
the second nearest neighbors of the given cube's parent, or are
children of the given cube's parent's nearest neighbors, excluding
nearest or second nearest neighbors of the given cube.
\end{Definition}
There are a maximum of $ 189 $ interaction cubes for a given cube,
roughly half are from a level one coarser than the level of the given
cube, the rest are on the same level.  The interaction cubes have two
important properties. When combined with the given cube's nearest and
second nearest neighbors, they entirely cover the same volume as the
given cube's parent and the parent's nearest and second nearest
neighbors. Also, the interaction cubes are such that the distance
between a point in the given cube and a point in the interaction cube
is more than half the distance between the centers of the given and
interaction cubes.  This latter property guarantees that when
multipole expansions are used to approximate the effects of
interaction cubes, and when these multipole expansions are gathered
together in a local expansion for the given cube, the resulting
approximation will converge rapidly with increasing expansion order.

\begin{Remarknonum}
As the charges in this problem are not point charges, but are
distributed on panels, it
is necessary to ensure that each panel is entirely contained in a finest
level cube  in order to ensure that evaluation points in a
cube are well separated from panel charges in an interaction cube.  
This may require breaking a panel up into several panels,
but as the multipole algorithm grows linearly with the number of
panels, this is not a significant computational burden.
\end{Remarknonum}

The structure of the multipole algorithm for computing the $ n $ panel 
potentials from $ n $ panel charges is given below.  The formulas for
various transformations and shifts required are given in the appendix.
%Appendix~\ref{appfor}.
A three letter key for each transformation is given to simplify finding
the corresponding  appendix formula.

%\begin{table}
%\begin{singlespace}
\begin{tabbing}
{\bf Algorithm 2: Multipole Algorithm for Computing $ Pw $ }\\[0.2in]
T \= \kill
\>T \= \kill
\>\>T \= \kill
\>\>\>T \= \kill
\>\>\>\>T \= \kill
\> /* \\
\> THE DIRECT PASS: The potentials due to nearby charges are computed \\
\> directly. \\
\> */\\
\> For each {\em cube} of the $ 8^L $ cubes on the finest level \{ \\
\> \> /* Map panel charge distributions to potentials (Q2P). */ \\
\> \> Compute the potential at all the evaluation points in the $ cube $\\
\> \> due to the charge distributions on all the panels in the $ cube $, in \\
\> \> the $cube$'s nearest neighbors, and in the $cube$'s second nearest \\
\> \> neighbors.\\
\> \} \\
\\
\> /* \\
\> THE UPWARD PASS: Computes a multipole expansion for every cube\\
\> at every level.  The computation is order $ n $ because the multipole \\
\> expansion for any cube at a level coarser than the finest level is \\
\> computed by combining the multipole expansions of its children.   \\
\> */\\
\> For each {\em cube} of the $ 8^L $ cubes on the finest level \{ \\
\> \> /* Map panel charges to multipole coefficients (Q2M). */\\
\> \> Construct a multipole expansion for the charge distributions on all the\\
\> \> panels in the $ cube $, about the $cube$'s center.\\
\> \} \\
\\
\> For each level $ i = L-1 $ to $ 2 $ \{ \\
\> \> For each $ cube $ of the $ 8^i $ cubes on level $ i $ \{ \\
\> \> \> For each of the $ 8 $ children of the $ cube $ \{  \\
\> \> \> \>/* Map multipole coefficients to multipole coefficients (M2M). */\\
\> \> \> \> Shift the multipole expansion about the child cube's center\\
\> \> \> \> to a multipole expansion about the $ cube $'s center and add it\\
\> \> \> \> to the multipole expansion for the $ cube $.\\
\> \> \> \} \\
\> \> \} \\
\> \} \\
\\
\> /*\\
\> THE DOWNWARD PASS: Computes a local expansion for every cube.  \\
\> The local expansion includes the effects of all panel charges not in the\\
\> cube or its nearest and second nearest neighbors. Note that at the finest\\
\> level this includes the effects of all panels that are {\em not} treated in the \\
\> direct pass.\\
\> */\\
\>For each level $ i = 2 $ to $ L $ \{ \\
\> \>For each {\em cube} of the $ 8^i $ cubes on level $ i $ \{ \\
\> \> \>/* Map local coefficients to local coefficients (L2L). */\\
\> \> \>If the $ cube$'s parent has a local expansion, shift that expansion\\
\> \> \>to a local expansion about the $ cube $'s center.\\
\\
\> \> \>For each of the $ cube $'s interaction cubes \{ \\
\> \> \> \>/* Map multipole coefficients to local coefficients (M2L). */\\
\> \> \> \>Convert the multipole expansion about the center of the \\
\> \> \> \>interaction cube to a local expansion about the $ cube $'s\\
\> \> \> \>center and add it to the local expansion for the $ cube $.\\
\> \> \> \} \\
\> \> \} \\
\> \} \\
\\
\>/*\\
\>THE EVALUATION PASS:  Evaluates the local expansions at the finest level.\\
\>*/\\
\> For each {\em cube} of the $ 8^L $ cubes on the lowest level \{ \\
\> \> /* Map local coefficients into potentials (L2P). */\\
\> \> Evaluate the $ cube$'s local expansion for the potential at all the\\
\> \> evaluation points in the $ cube $, and {\it add } those computed \\
\> \> potentials to the evaluation point potentials.\\
\> \} \\
\end{tabbing}
%\end{singlespace}
%\end{table}

  
