This manual describes FastCap, a three-dimensional capacitance
extraction program.  FastCap computes self and mutual capacitances
between ideal conductors of
arbitrary shapes, orientations and sizes.  
The conductors can be embedded in a dielectric
region composed of any number of constant-permittivity regions of any shape and
size.
The algorithm used in
FastCap is an acceleration of the boundary-element technique for
solving the integral equation associated with the multiple-conductor,
multiple-dielectric
capacitance extraction problem.  
The linear system resulting  from the boundary-element discretization
is solved using a
a generalized conjugate
residual algorithm with a fast multipole algorithm to efficiently
compute the iterates.

This manual is divided into two sections.  The first section explains
how to prepare input files for FastCap.  The input files contain the
description of the conductor surface and dielectric interface geometries.  
The second section shows how to run the program.
It mainly explains how to modify the default settings assumed by
FastCap.  Also documented in the section are outputs from several
example runs, including the postscript files FastCap produces as
an aid to problem visualization.

Information on compiling FastCap, obtaining the FastCap source code and 
corresponding about FastCap is given in
Appendix~\ref{comfas}. Further appendices provide background information
about the FastCap algorithm.


%\noindent{\sc Probably be good to put something in here about mailing
%to fastcap@sobolev.mit.edu and fastcap-bugs@sobolev.mit.edu}
